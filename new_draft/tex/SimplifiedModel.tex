

%%%%%%%%%%%%%%%%
%% JS: this new introduction is aimed at helping this document stand alone; I'd imagine in the final white paper a lot of these ideas would be introduced in the overall introduction.

Long-lived particles (LLPs) arise in many well-motivated theories of physics beyond the Standard Model, ranging from such well-established scenarios as the MSSM to newer models such as neutral naturalness and hidden sector dark matter.  Macroscopic decay lengths can arise from many possible considerations, including flavor, cosmology, and the structure of (super-)symmetry breaking, and are a natural and generic prediction in theories of weak-scale hidden sectors \JS{cite matt strassler here}.  A comprehensive search program for LLPs is therefore critical to fully leverage the LHC's immense capability to illuminate the weak scale and the cosmology of our universe. 

LLPs present both major opportunities and major challenges at the LHC.  On one hand, the SM backgrounds for displaced objects are inherently small, in many cases making prospects for displaced signals much better than for otherwise similar prompt signals.  On the other hand, searches for LLPs demand specialized techniques at all stages of the analysis, from triggering to reconstruction to background estimation techniques.  Further, as LLP searches involve aspects of detector response which cannot be reliably simulated with public tools, they are notoriously diffcult for theorists to accurately apply to new models, potentially jeopardizing their future utility.  

This document builds on the exceptional success of the existing LHC program to establish a systematic framework for future LLP searches that helps ensure experimental results are (i) {\em powerful}, covering as much territory as possible; (ii) {\em flexible}, so that they are broadly applicable to different types of models, (iii) {\em efficient}, reducing unneccesary redundancy among searches, and (iv) {\em durable}, providing a common framework for comparing and reinterpreting searches for years to come.
We expect that this common framework will help illuminate gaps in coverage and highlight areas where new searches are needed, \JS{possibly ref to future section in white paper??}.


%New particles with long lifetimes arise in many proposed extensions of the Standard Model.  In some cases, experimental searches have targeted these new particles; in others, proposals have been made for new techniques to reconstruct LLPs in particular models.  It is well-known that LHC searches for new physics, such as models of supersymmetry or dark matter, are sufficiently general that a new dedicated search is typically not necessary in order to cover every new proposed model.  The exceptional success of the existing LHC program informs us that it is desirable to develop search strategies that ensure



%The most commonly adopted framework for accomplishing these goals is the simplified model framework. With LLP signatures, the simplified models approach is even \emph{more} appropriate:~this is because searches tend to have low backgrounds and relatively inclusive searches can be done, each of which has sensitivity to LLPs with many different types of production and decay modes.

The simplified model framework has proven to be a highly successful approach for accomplishing these goals in the case of prompt signatures  \cite{Alves:2011wf}.  Simplified models are so successful because the majority of search sensitivity is driven by a few broad aspects of a given BSM signature (such as a production mode, production rate, and decay topology), while remaining insensitive to properties such as the spin of the particles involved, etc.  Our aim here is to construct an initial basis of simplified models for theories containing one or two LLPs.  The simplified model approach is even more powerful for LLP signatures:~the typically lower backgrounds for displaced signatures allow searches to be more inclusive, thus enabling a single search to have sensitivity to a wide variety of possible models for LLP production and decay.

Because LLPs are often produced and decay (if at all) in physically distinct locations, the production and decay processes can typically be  considered separately.\footnote{In addition to production and decay, a third consideration is the propagation of particles through the detector.  While neutral or only electrically charged particles have straightforward propagation, colored states, \emph{e.g.}, SUSY $R$-hadrons, or particles with exotic charges such as magnetic monopoles or quirks, typically engage in a more complicated and often very uncertain traverse.  The subtleties with colored LLPs will be discussed more in subsection \ref{sec:coloredLLPs}, and exotic, conserved charges will not be discussed here.}  
We can exploit this observation to construct a simple basis of LLP production modes and LLP decay modes, defining an {\bf LLP channel} as the combination of a particular production mode with a particular decay mode, with the lifetime of the LLP taken as a free parameter. We emphasize that the LLP channel defined here is \emph{not} the same as an experimental signature that manifests in the detector:~a single channel can give rise to many different signatures depending on where the LLP decays occur in the detector (or outside), while a single experimental search for a particular signature could cover many LLP channels for particular choices of parameters in the corresponding simplified models. We here  construct a basis of simplified models of LLP channels, drawing on the extensive  literature of theoretical models predicting LLPs at the LHC;
%focus here on the theory definition, which provides some control over the range of models that we expect to see at the LHC, while
in Section \ref{sec:experimentcoverage} we map existing searches into this basis of simplified models to determine the gaps in coverage and make proposals for new searches.

The list of simplified models presented here is  a starting point, rather than a final statement:~the present goal is to provide a set of simplified models that covers many of the best-motivated and simplest models containing singly- and doubly-produced LLPs. Simplified models by design do not include all of the specific details and subtle features that may be found in a given complete model.  Therefore, the provided list is meant to be expandable to cover new or more refined models as the LLP search program develops.  For instance,  extending the simplified model framework to separately treat heavy flavor is of high interest (in analogy to the prompt case, \cite{Essig:2011qg}); see Section \ref{sec:simplified_future} for a discussion of this and other future opportunities.   High-multiplicity signatures such as dark showers or emerging jets present different experimental and theoretical issues, as we discuss in  Section \ref{sec:showers}.

%The simplified model framework is also not ideally suited to high-multiplicity signatures such as dark showers or emerging jets, whose phenomenology depends crucially on model parameters such as strong couplings, hadronization details, and  spectra and lifetime of large numbers of new hidden-sector particles; we therefore focus in this section on production of one or two LLPs, leaving a discussion of future opportunities in Section \ref{sec:simplified_future} and dark shower signatures in Section \ref{sec:showers}.



%%%%%%%%%%%%%%%%%%%%%%%%%%%%%%%
\section{Goals of the Present Simplified Model Framework}
%%%%%%%%%%%%%%%%%%%%%%%%%%%%%%%

The purpose of the simplified model framework is to provide a simple, common language that experimentalists and theorists can use to describe theories of LLPs and the corresponding mapping between models and experimental signatures. We therefore want our simplified model space to:
%
\begin{enumerate}
%
\item  Use a minimal set of models to cover a wide range of the best-motivated theories of LLPs;
\item  Furnish a simple map between models and signatures to enable a clear assessment of existing search coverage and possible gaps; %Map easily between models and signatures so that the current coverage and gaps may be easily identified;
%\item Apply to theories and signatures not yet proposed;
% JS: I commented this out because it read to me as redundant with the next bullet point
\item Expand flexibly when needed to incorporate theories and signatures not yet proposed;
\item Provide a concrete Monte Carlo signal event generation framework;
\item Facilitate the re-interpretation of searches by supplying a sufficiently varied set of standard benchmark models\footnote{Note that  in general more benchmark models will be needed for enabling reliable re-interpretation than are strictly necessary for discovering a new particle, \emph{i.e.,}  it is important to consider both whether two simplified models share a common signature in a search, and also whether they look similar enough to have similar reconstruction efficiencies} for which experimental efficiencies can be provided for validation purposes.
\end{enumerate}
%
In the following sections we will construct a proposal for a minimal basis of simplified models for events with one or two LLPs.
We begin with a discussion of the theories that dominate the predictions for LLPs at the LHC, and identify a set of `umbrella' models that yield LLPs in Section~\ref{sec:motivated_theories}.  We next identify the relevant (simplified) production and decay modes for LLPs in Section~\ref{sec:building_blocks},  emphasizing that each production mode has a characteristic set of predictions for the number and nature of {\em prompt} objects accompanying the LLP.  In Section~\ref{sec:} we combine these production and decay modes into our simplified model basis and highlight how different umbrella models naturally populate the various LLP channels.  Section~\ref{sec:}  presents a framework and instructions for how the best motivated simplified model channels can be simulated in Monte Carlo using a model library that is currently under development. \JS{ probably want to change this wording} \BS{how about this?}

%a simplified model framework for events with one and two LLPs.  Extending this  framework to models with dark showers will be discussed later.

%%%%%%%%%%%%%%%%%%%%%%%%%%%%%%%%%%%%%
\section{Existing Well-Motivated Theories for LLPs}\label{sec:motivated_theories}
%%%%%%%%%%%%%%%%%%%%%%%%%%%%%%%%%%%%%

Here we provide a brief distillation of the bulk of the literature that has to date provided the best motivation for (singly and doubly produced) LLPs into five over-arching umbrella scenarios.  We emphasize that each of these categories is a broad umbrella containing many different models yielding LLPs.
%
\begin{itemize}

\item {\bf SUSY-like theories (SUSY).}~This category contains models
  with multiple new particles carrying SM gauge charges and a variety
  of allowed cascade decays. LLPs can arise thanks to approximate
  symmetries (such as $R$-parity or indeed supersymmetry in the case
  of gauge mediation) or through a hierarchy of mass scales (such as
  highly off-shell intermediaries as in split SUSY, or
  nearly-degenerate multiplets, as in AMSB ).  Our terminology
  classifies any non-SUSY models with heavy SM charged particles, such
  as composite Higgs or extra-dimensional models, under the SUSY-like
  umbrella.  In this category LLP production is typically dominated by
  SM gauge interactions, whether of the LLP itself or of a heavy
  parent particle.

\item {\bf Higgs-portal theories (Higgs).}~In this category, LLPs couple dominantly to the SM-like Higgs boson.  This possibility is well-motivated as the Higgs provides one of the leading low-dimensional portals into the SM, and the experimental characterization of the Higgs boson leaves much scope for couplings to BSM physics.  The most spectacular signatures here are exotic Higgs decays to low-mass particles  (as in many Hidden Valley scenarios), which can arise in models of neutral naturalness and dark matter.  
%Experimentally, this scenario is well motivated by the fact that the Higgs can still have a 30\% branching fraction to exotic final states and theoretically by the fact that new singlet particles can have large couplings to the Higgs. 
The Higgs is also special in that it comes with a rich set of associated production modes in addition to the dominant gluon fusion process, with VBF and Higgstrahlung providing important avenues for trigger strategy and background suppression.

\item {\bf Gauge-portal theories (ZP).}~ This category contains scenarios where new vector mediators can produce LLPs. These are similar to Higgs models, but where the mediator is predominantly produced from $q\bar{q}$-initiated final states without other associated objects. Examples include models where both SM fermions and LLPs carry a charge associated with a new $Z'$, as well as ``dark'' photon or $Z$ models in which the couplings of new vector bosons to the SM are mediated by kinetic mixing. Scenarios with LLPs coupled to new gauge bosons are well motivated by theories of dark matter, particularly models with significant self-interactions and/or sub-weak mass scales [cite various].

\item {\bf Dark-matter theories (DM):}~ Non-SUSY and hidden sector DM scenarios are collected in this category, which encompasses models where dark matter is produced as a final state in the collider process. The main feature distinguishing this category from the Higgs and gauge scenarios above is that dark matter, i.e., missing momentum, is a necessary and irreducible component of each signature.

\item {\bf Heavy neutrinos and friends (RH$\nu$):}  Models where  new weak-scale states are responsible for SM neutrino mass generation typically predict long-lived TeV-scale right-handed neutrinos, and may additionally predict new long-lived Higgs bosons.  Characteristic features of models in this category are singly-produced LLPs and lepton-rich signatures.

\end{itemize}

%
%As we develop our simplified model framework, we will construct maps between the UV models and the simplified model channels to show the highest priority/best motivated combinations of production and decay modes for LLPs. 
In developing our simplified model framework below, we will construct maps between these UV model categories and the simplified model channels to illuminate some of the best-motivated combinations of production and decay modes for LLPs. This will then allow us to focus on the most interesting channels and assess their coverage.
%in particular parts of the model parameter space.


%%%%%%%%%%%%%%%%%%%%%%%%%%%%%%%%%%%%%%%%%%%%%%
\section{The Simplified Model Building Blocks}\label{sec:building_blocks}
%%%%%%%%%%%%%%%%%%%%%%%%%%%%%%%%%%%%%%%%%%%%%%


As discussed above, in LLP searches production and decay can largely be factorized. This allows us to specify the relevant production and decay modes for LLP models separately; we then put them together and map the space of models into the existing motivated theories.

%%%%%%%%%%%%%%%%%%%%%%%%%%%%%%%%%%%
\subsection{Production Modes}
%%%%%%%%%%%%%%%%%%%%%%%%%%%%%%%%%%%

According to the earlier models, we can identify a minimal set of interesting production modes for LLPs.
These production modes determine LLP signal yield both by establishing the overall production cross-section and by determining a boost distribution for the LLPs.   Additionally, a given production mechanism will also  make generic predictions for the number and type of {\em prompt} objects accompanying the LLP(s).  These prompt accompanying objects (AOs) can be important for both triggering on events with LLPs and for background rejection, particularly when the LLP has a low mass. 

%
\begin{itemize}

\item {\bf Direct Pair Production (DPP):}~Here the LLP is dominantly pair-produced non-resonantly from SM initial states.  This is most straightforwardly obtained when the LLP is charged under a SM gauge interaction.  
%, it can be directly produced via the corresponding gauge boson. 
In this case, an irreducible production cross section is then specified by the LLP gauge charge and mass.  DPP can also occur in the presence of a (heavy) $t$-channel mediator (e.g., an initial quark-antiquark pair may exchange a virtual squark to pair produce bino-like neutralinos); in this case the production cross section is a free parameter.

\item {\bf Heavy parent (HP):}~ In this case the LLP can be produced in the decay of a heavy parent particle that is itself pair produced. The production cross section is essentially a free parameter, is indirectly specified by the charge and mass of the heavy parent. Heavy parent production gives very different kinematics for the LLP than direct pair production production, and will often produce additional prompt accompanying objects in the decays.

\item {\bf Higgs (HIG):}~The LLP is produced through its couplings to the SM Higgs boson.  This case has an interesting interplay of possible production modes. The dominant production cross-section is  gluon fusion, which features no associated objects beyond ISR, but thanks to its role in EWSB, the Higgs has associated production modes (VBH, VH) with characteristic features. The best prospects are for LLP masses below $m_h/2$, which can be produced in exotic Higgs decays, but LLPs with heavier masses can still produced via the off-shell Higgs portal, albeit at lower rates. The LLP can be pair-produced or singly produced (in association with MET). The cross section (or Higgs branching fraction) is a free parameter of the model.


\item {\bf Heavy resonance (ZP):}~Here the LLP is produced in the decay of an on-shell resonance, such as a heavy $Z'$ gauge boson initiated by $q\bar{q}$ initial state or a heavy scalar\footnote{The properties of the observed light Higgs suggest that any new scalar would minimally impact EWSB and thus would have at most only a very small rate from VBF and VH processes; for this reason, we place heavy scalars in the heavy resonance model as opposed to the Higgs model above.}, $\Phi$.  Note that production via an off-shell resonance is kinematically similar to the direct production (DPP) above.  As with HIG, the LLP can be pair-produced or singly-produced (in association with MET). Here ISR is the dominant source of accompanying objects.
%\item {\bf  $Z'$ (ZP):}~Similar to the Higgs portal, but the LLP is produced in the decay of an on-shell gauge boson initiated by $q\bar{q}$ initial state. Note that production via off-shell $Z'$ is similar to if the LLP is directly charged under the SM $Z$ and so is included in GP above. As with HIG, the LLP can be pair produced or singly produced (in association with MET).
\item {\bf Charged current (CC):}~In models with weak-scale right-handed neutrinos, the LLP can be produced in the leptonic decays of $W/W'$. Single production is favored.  Prompt leptons are typical accompanying objects.
\end{itemize}
%

%There are other scenarios in which associated objects accompany the LLP production, such as the prompt lepton arising from charged-current production of neutral LLPs or in ``dark $Z$''-type scenarios, ISR jets in DM-motivated scenarios, and many other cases. 
The generic presence of associated prompt objects (such as VBF or central jets, lepton, missing momentum, etc.) in many LLP models indicates that they may offer valuable levers to extend sensitivity to otherwise hard-to-reach parts of parameter space.
%, where possible, LLP searches should be supplemented with signal regions that exploit the presence of the associated objects, extending sensitivity to otherwise hard-to-reach parts of parameter space.

%%%%%%%%%%%%%%%%%%%%%%%%%%
\subsection{Decay Modes}
%%%%%%%%%%%%%%%%%%%%%%%%%%


It is important to note that a given LLP search can frequently be
sensitive to a variety of possible LLP decay modes.  This is in part
because particles that decay far inside the detector offer fewer
avenues for particle identification, (\emph{e.g.,} for an LLP decaying
inside of the calorimeter, all decay products are either missing
energy or a calorimeter deposition). It is also because backgrounds
are often low enough that tight identification and/or reconstruction
cuts typically found in exclusive analyses are not needed to suppress
backgrounds. For example, ATLAS has a displaced vertex search
sensitive to dilepton and multitrack vertices that are relatively
agnostic to other objects originating from near the displaced vertex
(ref). Similarly, CMS has an analysis sensitive to
high-impact-parameter leptons without reconstructing a vertex
(ref). 

However, in some cases the topology of a decay does matter:~for
example, one potentially important factor is distinguishing cases
where the LLP decays into two SM objects vs.~three, because this
determines whether its mass can, in principle, be resonantly reconstructed by looking
for two objects in a decay.  \JS{this point still needs some
  refinement: in many cases there is simply is not enough information
  to reconstruct a resonance.  This only applies for tracker decays
  and I guess $X\to \mu\mu$ going off any where, which should be made
  explicit.  It may be helpful to provide the explicit example of
  lepton pairs in the tracker?} Whether or not this is possible in practice depends on the details
  of the decay; for example, an LLP decay to a muon pair at any point in the detector
   or a pair of any visible objects in the tracker
  may be reconstructed as a resonance, while an LLP decay to hadrons inside of the calorimeter
  cannot be reconstructed as a resonance. An additional simplified model 
  featuring a non-resonant or 3-body decay of the LLP may be needed depending on the particular
  search, and so we include the possibility of an LLP decays with additional non-reconstructed objects. \BS{Kind of a mouthful; maybe a better way to integrate?}

{\bf We therefore emphasize that the following decay modes are defined
  loosely in the understanding that they will also provide good
  acceptance for similar and/or related decay modes, for example
  $2j+invisible$ is also a proxy for $3j$ because searches for
  non-resonant hadronic LLP decays can be sensitive to both.} It
should also be emphasized that we are not recommending searches to be
optimized to the exact, exclusive decay mode because that could
suppress sensitivity to related but slightly more complicated models.

\JS{Q: provide at least one familiar example for each of the following
  bullets?}

\begin{itemize}
\item {\bf Diphoton decays:}~The LLP can decay resonantly to
  $\gamma\gamma$ (like in Higgs-portal models) or to
  $\gamma\gamma+\mathrm{invisible}$ (in dark matter models). This
  latter mode stands as a proxy for other $\gamma\gamma+X$ decays
  where the third object is not explicitly reconstructed.

\item {\bf Single photon decays:}~The LLP decays to
  $\gamma+\mathrm{invisible}$ (like a bino in models with
  gauge-mediated SUSY-breaking).  While the SUSY signal mandates a
  near-massless invisible particle, a more generalsignature allows for
  a heavy invisible particle, as can arise in theories of dark matter
  (ref).
%dark matter framework can come with a heavy final state particle.

\item {\bf Hadronic decays:}~The LLP can decay into two jets ($jj$)
  (like in Higgs and gauge portal models, or RPV SUSY), $jj$ +
  invisible (SUSY, dark matter, or neutrino models), or $j$ +
  invisible (SUSY). Here, jet ($j$) means either a light-quark jet,
  gluon, or $b$-quark jet.

\item {\bf Semileptonic decays:}~The LLP can decay into a lepton + 1
  or 2 jets (like in SUSY or neutrino models).

\item {\bf Leptonic decays:}~The LLP can decay into
  $\ell^+\ell^-(+\mathrm{invisible})$, or
  $\ell^\pm+\mathrm{invisible}$ (as in Higgs-portal, gauge-portal,
  SUSY, or neutrino models). $\ell$ may be any flavor of charged
  lepton, but the decays are lepton flavor-universal and (for
  $\ell^+\ell^-$ decays) flavor-conserving.

\item {\bf Flavored leptonic decays:}~The LLP can decay into
  $\ell_\alpha+\mathrm{invisible}$, $\ell_\alpha^+\ell_\beta^-$ or
  $\ell_\alpha^+\ell_\beta^-+\mathrm{invisible}$ where flavors
  $\alpha\neq\beta$ (as in SUSY or neutrino models).
\end{itemize}

In all cases, both the LLP mass and proper lifetime are free
parameters.  Therefore, stable particle searches are also covered by
taking the $c\tau\rightarrow\infty$ limit of any decay mode.  We
emphasize that, depending on the location of the LLP within the
detector, these decay modes may or may not be individually
distinguishable; a displaced dijet decay will look very different from
a displaced diphoton decay in the tracker, but nearly identical if the
decay occurs in the calorimeter.  We are identifying promising
channels here, as distinct from detector signatures. 

As an example of how the above listed decay modes cover the most important
experimental signatures, we consider a scenario of an LLP decaying to
top quarks. This scenario is very well-motivated (for instance, with
long-lived stops on SUSY) and might appear to merit its own decay
category. However, the top quark immediately decays to final states
that \emph{are} covered in the above list, giving a semileptonic
decay ($t\rightarrow b\ell^+\nu$) and a hadronic decay ($t\rightarrow
bjj$), and so the above model-independent decay modes cover this
important scenario.

While it would be ideal to have separate experimental searches for
each of the above decay modes (when distinguishable), it is rare for
specific models to allow the LLP to decay in only one manner; instead,
as in the displaced top example above, a number of decay modes are
typically allowed with corresponding predictions for the branching
fractions. For example, if a LLP couples to the SM via mixing with the
SM Higgs boson, then the LLP decays via mass-proportional couplings
giving rise to $b$- and $\tau$-rich signatures. If, instead, the LLP
decays through a kinetic mixing as in the case of dark photons or $Z$
bosons, then the LLP can decay to any particle charged under the weak
interactions, giving rise to a relatively large leptonic branching
fraction in addition to hadronic decay modes. This allows some level
of prioritization of decay modes based on motivated UV-complete
models, although it is ultimately desirable to retain independent
sensitivity to each individual decay mode.  Indeed, for each decay
mode listed above, models exist in the literature for which the given
decay mode would be the discovery channel. \JS{I flag this addition as
  it is a strong statement, but I am sure that this is true.}
\linebreak


\noindent {\bf Invisible Final-State Particles:}~where invisible
particles appear as a product of LLP decay, additional
model-dependence arises from the unknown nature and mass of the
invisible particle. The invisible particle could be a SM neutrino, DM,
a SUSY LSM, or another beyond-SM particle. The phenomenology depends
strongly on the mass splitting $\Delta \equiv M_{\rm LLP}-M_{\rm
  invisible}$. If $\Delta \ll M_{\rm LLP}$ (\emph{i.e.,} $M_{\rm
  LLP}\sim M_{\rm invisible}$, the spectrum is squeezed and the decay
products of the LLP are soft. This could, for instance, lead to
signatures such as disappearing tracks or necessitate the use of ISR
jets to reconstruct the LLP signature. If the mass splitting is large,
$M_{\rm invisible}\ll M_{\rm LLP}$, then the signatures lose their
dependence on the invisible particle mass.

We suggest three possible benchmarks:~a squeezed spectrum with $\Delta
\ll M_{\rm LLP}$; a massless invisible state, $\Delta = M_{\rm LLP}$
(which also includes the case where the invisible particle is a SM
neutrino); and an intermediate splitting corresponding to a democratic
mass hierarchy, $\Delta \approx M_{\rm LLP}/2$.

%%%%%%%%%%%%%%%%%%%%%%%%%%%%%%%%%%%%
\section{Our Simplified Model Proposal}
%%%%%%%%%%%%%%%%%%%%%%%%%%%%%%%%%%%%

We separately consider LLPs that are:~(a)~neutral, (b)~electrically
charged but color neutral, and (c)~color charged. These three
possibilities are considered separately because the relevant
production modes are different (the latter two have irreducible
production through SM gauge interactions) and their signatures can be
different (such as disappearing tracks and hadronized LLPs).

Once again, we emphasize that in spite of the many simplified model
channels, there are a small number of experimental LLP searches that
have excellent coverage for a wide range of channels. Our major goal
is ultimately to identify whether there are other searches that could
have a similarly high impact, and where the gaps are.

%%%%%%
\subsection{Neutral LLPs}
%%%%%%

The simplified model channels for neutral LLPs are shown in Table
\ref{tab:neutral_LLP}. In the first iteration of the simplified
models, it is sufficient to consider as ``jets'' each of the
following:~$j=u,d,s,c,b,g$. However, we comment that $b$-quarks pose
unique challenges and opportunities. Since $b$-quarks are themselves
LLPs, they appear with an additional displacement relative to the LLP
decay location. They also often give rise to soft muons in their
decays, which could in principle lead to additional trigger or
selection possibilities. We discuss this further in Section
\ref{sec:simplified_future}. Similarly, for now we consider $e$,
$\mu$, and $\tau$ to be included in the header of ``leptons'' and
searches should try to be sensitive to each.
 
When multiple production modes are specified in one row (typically one
or more in parentheses), this means that multiple especially
well-motivated production channels give rise to similar
signatures. Typically only one of these production modes will need to
be included when developing a search, but we include the different production modes
to indicate where people's favorite models may lie. $X$ indicates the
LLP. We further discuss the subtleties around our choice of simplified model channels in the final paragraph of this section.  \JS{I feel like our philosophy is not completely consistent
  here.  We have earlier emphasized the desireability for multiple
  models to help recast; here we are providing a really stripped down
  set.  We should be very explicit about this choice and the reasoning
  behind it (discovery only).  A useful avenue for future work would
  be to study what additional simplified models might maximally help with recastability.}\BS{See new last paragraph of section; open to suggestions or edits}
  


In each entry of the table, we indicate which umbrella of well-motivated models
(Section~\ref{sec:motivated_theories}) can predict a particular
$(\mathrm{production})\times(\mathrm{decay})$ mode.  An asterisk on
the umbrella model indicates that MET is required in the decay.
%
\begin{table}
\begin{center}
\begin{tabular}{ |c|c|c|c|c|c|c| } 
 \hline
\backslashbox{Production}{Decay} & $\gamma\gamma(+\mathrm{inv.})$ & $\gamma+\mathrm{inv.}$ & $jj(+\mathrm{inv.})$ & $jj\ell$ & $\ell^+\ell^-(+\mathrm{inv.})$ & $\ell_\alpha^+\ell_{\beta\neq\alpha}^-(+\mathrm{inv.})$\\
\hline\hline
DPP:~sneutrino pair &  & SUSY & SUSY & SUSY & SUSY & SUSY\\
 \hline
 HP:~squark pair, $\tilde{q}\rightarrow jX$ &  & SUSY & SUSY & SUSY & SUSY & SUSY\\
 or gluino pair $\tilde g\rightarrow jjX$ &&&&&&\\
 \hline
HP:~slepton pair, $\tilde{\ell}\rightarrow\ell X$ &  & SUSY & SUSY & SUSY & SUSY & SUSY\\
 or chargino pair, $\tilde{\chi}\rightarrow WX$ &&&&&&\\
 \hline 
% HIG:~$h(h')\rightarrow XX$ & Higgs (DM)  &  & Higgs (DM) &  & Higgs (DM) & \\
 HIG:~$h\rightarrow XX$ & Higgs, DM*  &  & Higgs, DM* &  & Higgs, DM* & \\
  or $\rightarrow XX+\mathrm{inv.}$ &&&&&&\\
 \hline 
 %HIG:~$h(h')\rightarrow X+\mathrm{inv.}$ & DM  &  & DM &  & DM & \\
 HIG:~$h\rightarrow X+\mathrm{inv.}$ & DM*  &  & DM* &  & DM* & \\
  \hline
   ZP:~$Z(Z')\rightarrow XX$ & $Z'$, DM*  &  & $Z'$, DM* &  & $Z'$, DM* & \\
  or $\rightarrow XX+\mathrm{inv.}$ &&&&&&\\
 \hline 
 ZP:~$Z(Z')\rightarrow X+\mathrm{inv.}$ & DM  &  & DM &  & DM & \\
  \hline
   CC:~$W(W')\rightarrow \ell X$ &   &  & RH$\nu$* & RH$\nu$ & RH$\nu$* & RH$\nu$* \\
  \hline
\end{tabular}
%
\end{center}
\caption{{\bf Simplified model channels for neutral LLPs.} The LLP is indicated by $X$. Each row shows a separate production mode and each column shows a separate possible decay mode, and therefore every cell in the table corresponds to a different simplified model channel of (production)$\times$(decay). We have cross-referenced the UV models from Section \ref{sec:motivated_theories} with cells in the table to show how the most common signatures of complete models populate the simplified models. When two production modes are provided (with an ``or''), both simplified models can be used to cover the same experimental signatures. Parentheses in the decay mode indicate the presence of additional $\slashed{E}_{\rm T}$ in some models. The asterisk (*) shows that the model definitively predicts missing momentum in the LLP decay. }\label{tab:neutral_LLP}
\end{table}
%
Again, we emphasize that the production modes listed in Table
\ref{tab:neutral_LLP} encompass also the associated production of
characteristic prompt objects. For example, the Higgs production modes
not only proceed through gluon fusion, but also through vector boson
fusion and $VH$ production, both of which result in associated prompt
objects such as forward tagging jets, leptons, or missing
momentum. All of the production modes listed in Table
\ref{tab:neutral_LLP} could be accompanied by ISR jets that aid in
triggering or identifying signal events. It is therefore important
that searches are designed to exploit such associated prompt objects
whenever they can improve signal sensitivity.

To demonstrate how to map full models onto the list of simplified
models (and vice-versa), we consider a few concrete cases. For
instance, if we consider a model of neutral naturalness where $X$ is a
long-lived scalar that decays via Higgs mixing (for instance, $X$
could be the lightest quasi-stable glueball), then the process where
the SM Higgs $h$ decays to $h\rightarrow XX$, $X\rightarrow b\bar{b}$
would be covered with the Higgs production mechanism and a dijet
decay. Entirely unrelated models, such as the case where $X$ is a bino-like
neutralino in $h\rightarrow XX$, $X\rightarrow j jj $ would be covered
with the same simplified model. Similarly, a hidden sector model with
a dark photon, $A'$, produced in $h\rightarrow A'A'$,
$A'\rightarrow f\bar{f}$ would also give rise to the dijet signature
when $f$ is a quark, whereas it would populate the $\ell^+\ell^-$
column if $f$ is a lepton. Finally, a scenario with multiple hidden
sector states $X_1$ and $X_2$, in which $X_2$ is an LLP and $X_1$ is a
stable, invisible particle, could give rise to signatures like
$h\rightarrow X_2 X_2$, $X_2\rightarrow X_1jj$ that would be covered by the
same Higgs production, hadronic decay simplified model; however, we
see how $\slashed{E}_{\rm T}$ can easily appear in the final state,
and that a dijet pair does not always reconstruct a
resonance. Therefore, the simplified models in Table
\ref{tab:neutral_LLP} can cover an incredibly broad range of
signatures, but only if searches are not overly optimized to
particular features such as $\slashed{E}_{\rm T}$ and
resonances\footnote{This is, of course, not to say that searches
  shouldn't be done that exploit these features, but only that
  experiments should bear in mind the range of topologies and models
  covered by each cell in Table \ref{tab:neutral_LLP} when designing
  searches.}.
  
    We reiterate that a goal of this section is to develop a compact set of simplified model channels that, in broad brush strokes, covers the space of theoretical models in order to motivate new experimental searches. Such a minimal, compact set may not be optimal for reinterpretation of results (where variations on our listed production and decay modes may influence signal efficiencies and cross section sensitivities), but rather provides a convenient characterization of possible signals to ensure that no \emph{major} discovery mode is missed. We comment further on the sensitivity of current and proposed searches on the simplified model proposals and their variants, as well as discussing challenges surrounding reinterpretation of LHC results, in Sections \ref{sec:experimentcoverage} and \ref{sec:recommendations}, respectively. \BS{An attempt at giving the precision that Jessie recommends above}.

\subsection{Electrically Charged LLPs:~$|Q|=1$}

Here, we need to consider \emph{far fewer} production modes because of the irreducible gauge production associated with the electric charge. We still consider one heavy parent scenario where the heavy parent has a QCD charge, as this could potentially dominate the production cross section. Note we lump all resonant production into the $Z'$ simplified model.  The reason is that the SM Higgs cannot decay into two charged particles due to the model-independent limits from LEP on charged particles masses $M\gtrsim75$ GeV.   Similarly, there are fewer decay modes because of the requirement of charge conservation.  %It is true that a heavy scalar could lead to LLP pair production, but due to the irreducible gauge production cross section, we do not need to rely on associated production modes and so the scalar and $Z'$ simplified models can be combined for simplicity.

For concreteness, we recommend using $Q=1$ as a benchmark for charged LLPs for the purpose of determining allowed decay modes. 
%Because other values of $Q$ are also possible, it is worth still treating the production cross section as a free parameter. 
Although other values of $Q$ are possible, these typically result in cosmologically stable charged relics or necessitate different decay paths than those listed here.   
We note that there are dedicated searches for heavy quasi-stable charged particles with either $Q\gg1$ or $Q\ll1$; because those searches are by construction not intended to be sensitive to the decays of the LLP, the existing models are sufficient for characterizing these signatures and they do not need to be additionally included in our framework.

\begin{table}
\begin{center}
\begin{tabular}{ |c|c|c|c|c|} 
 \hline
\backslashbox{Production}{Decay} & $\ell+\mathrm{inv.}$ &  $jj(+\mathrm{inv.})$ & $jj\ell$ & $\ell\gamma$ \\
\hline\hline
DPP:~chargino pair & SUSY & SUSY & SUSY & \\
or slepton pair & & & &\\
\hline
HP:~$\tilde{q}\rightarrow j X$ & SUSY & SUSY & SUSY & \\
\hline
%ZP:~$Z'\rightarrow XX$ & Higgs/Z'/DM & Higgs/Z'/DM & Higgs/Z'/DM \\
ZP:~$Z'\rightarrow XX$ & Z', DM*& Z', DM* & Z'  & \\
\hline
CC:~$W'\rightarrow X+\mathrm{inv.}$ & DM* & DM* &  &\\
\hline
\end{tabular}
\end{center}
\caption{{\bf Simplified model channels for electrically charged LLPs, $|Q|=1$.} The LLP is indicated by $X$. Each row shows a separate production mode and each column shows a separate possible decay mode, and therefore every cell in the table corresponds to a different simplified model channel of (production)$\times$(decay). We have cross-referenced the ``well-motivated'' UV models from Section \ref{sec:motivated_theories} with cells in the table to show how the most common signatures complete models can be linked to the simplified models. When two production modes are provided (with an ``or''), both production simplified models can be used to cover the same experimental signatures. Parentheses in the decay mode indicate the presence of additional $\slashed{E}_{\rm T}$ in some models. The asterisk (*) shows that the model definitively predicts missing momentum in the LLP decay. }\label{tab:charged_LLP}
\end{table}

\subsection{LLPs with Color Charge}
\label{sec:coloredLLPs}

Because QCD is a non-Abelian group, the gauge pair production cross section of the LLP is specified by the LLP mass and its representation under $\mathrm{SU}(3)$. 

A complication of the QCD-charged LLP is that the LLP hadronizes prior to the decay. While the hadronization will not affect any hard kinematic features of its decay, it can result in interesting phenomena such as stopping in the detector, charge flipping, etc. These have been greatly explored in the context of $R$-hadrons in SUSY and we refer those interested in performing such searches to the relevant literature.

\begin{table}
\begin{center}
\begin{tabular}{ |c|c|c|c|c|} 
 \hline
\backslashbox{Production}{Decay} & $j+\mathrm{inv.}$ &  $jj(+\mathrm{inv.})$ & $j\ell$ & $j\gamma$ \\
\hline\hline
DPP:~squark pair & SUSY & SUSY & SUSY & \\
or gluino pair & & & &\\
\hline
\end{tabular}
\end{center}
\caption{{\bf Simplified model channels for LLPs with color charge.} The LLP is indicated by $X$. Each row shows a separate production mode and each column shows a separate possible decay mode, and therefore every cell in the table corresponds to a different simplified model channel of (production)$\times$(decay). We have cross-referenced the ``well-motivated'' UV models from Section \ref{sec:motivated_theories} with cells in the table to show how the most common signatures complete models can be linked to the simplified models. When two production modes are provided (with an ``or''), both production simplified models can be used to cover the same experimental signatures. Parentheses in the decay mode indicate the presence of additional $\slashed{E}_{\rm T}$ in some models. }\label{tab:color_LLP}
\end{table}

{\bf Add in paragraph about effects of hadronization}

\section{A Simplified Model Library}

The simplified models outlined in the above sections provide a common language for theorist and experimentalists to study the sensitivity of existing searches, propose new search ideas, and interpret results in terms of UV models. However, the above activities demand a simple framework for the simulation of signal events that can be used to evaluate signal efficiencies of different search strategies and map these back onto model parameters. Requiring individual users to create their own MC models for each simplified model is impractical, inefficient, and invites the introduction of errors into the comparison of results by different individuals.

In this document, we propose and provide a draft version of a \emph{simplified model library} consisting of model files and Monte Carlo (MC) generator cards that can be used to generate events for various simplified models in a straightforward fashion. Because each experiment uses slightly different MC generators and settings, this allows each collaboration (as well as theorists) to generate events for each simplified model based on the provided files. Depending on how the LLP program expands and develops over the next few years, it may become expedient to go a step further and add to the simplified model library sets of events in a standard format (such as the Les Houches format) that can be directly fed into parton shower and detector simulation programs; given the factorization of production and decay of LLPs that is valid for all but QCD-charged LLPs, this could involve two mini-libraries:~a set of production events for LLPs, a set of decays for LLPs, along with a protocol for ``stitching'' the events together.

The current version of the library can be found here:~{\bf [provide link]}

We provide model files in the popular Universal Feynrules Output (UFO) format, which is designed to interface easily with parton-level simulation programs such as \texttt{MadGraph5\_aMC@}\texttt{NLO}. The goal is to cover as many of the simplified models of Section \ref{sec:building_blocks} with as few UFO models as possible; this limits the amount of upkeep needed to maintain the library and develops familiarity with the few UFO models needed to simulate the LLP simplified models. We then give specific instructions for how to simulate each simplified LLP channel using the UFO models. {\bf NOTE:~as this document is still in draft form, the library is not yet complete. If you need a simplified model that has not yet been filled in, or would like to contribute to completion of the library, please contact the organizers listed in the simplified model repository.}

\subsection{Base Models for Library}

In order to reproduce the simplified model channels of Section \ref{sec:building_blocks}, we need a collection of models that
%
\begin{itemize}
\item Includes additional gauge bosons and scalars to allow vector and scalar portal production of LLPs;
\item Includes new gauge-charged fermions and scalars to cover direct and simple cascade production modes of LLPs;
\item Includes a right-handed-neutrino-like state with couplings to SM neutrinos and leptons;
\item Allows for the decays of the LLP particle through all of the decay modes listed in Section \ref{sec:building_blocks}, either through renormalizable or high-dimensional couplings.
\end{itemize}

Fortunately, an extensive set of UFO models is already available for simulating the production of beyond-SM particles. We note that extensions or generalizations of only three already-available UFO models are needed at the present time:

\begin{enumerate}

\item {\bf The Minimally Supersymmetric SM (MSSM)}  contains a whole host of new particles with various gauge charges and spins. Therefore, an MSSM-based model allows for the simulation of many of the simplified model channels. In particular, we note that existing UFO variants of the MSSM that include gauge-mediated supersymmetry breaking (GMSB) couplings (including decays to light gravitinos) and $R$-parity violation (to allow for the decay of otherwise stable lightest SUSY particles) already cover most of the SUSY-motivated LLP scenarios.


\item{\bf The Hidden Abelian Higgs Model (HAHM)} consists the SM supplemented by a ``hidden sector'' consisting of a new $\mathrm{U}(1)$ gauge boson and corresponding Higgs field. The physical gauge and Higgs bosons couple to the SM via kinetic and mass mixing, respectively, and can be easily supplemented with new matter fields. The HAHM allows for straightforward simulation of exotic decays of the SM Higgs, as well as $Z'$ models and many hidden sector scenarios.

\item {\bf The Left-Right Symmetric Model (LR)} supplements the SM by an additional $\mathrm{SU}(2)_{\rm R}$ symmetry, which leads to additional charged and neutral gauge bosons. The model also contains a right-handed neutrino which is the typical LLP candidate, which can be produced via SM $W$, $Z$, or the new gauge bosons. 

\end{enumerate}

As outlined in the next section, nearly all of the simplified model channels can be simulated with minor variants of the above models. Note that this may require some funny-seeming adjustment of parameters:~for example, in the MSSM model which does not contain a $Z'$, the effects of a $Z'$ on signal production can be included by re-scaling the mass of the SM $Z$ boson. Because the libraries are mostly needed for simulation of signal events and not total production cross sections, this is acceptable within the simplified models framework. Similarly, the spins of the particles are generally not important:~replacing the direct production of a fermion with the direct production of a scalar will not substantially change the signature, so as long as results are expressed in terms of sensitivity to cross sections and not couplings, the results can be easily applied to any similar production mode regardless of spin.

\subsection{Instructions for Simulating Simplified Model Channels}

We refer here extensively to the simplified models in Tables  \ref{tab:neutral_LLP}-\ref{tab:color_LLP}. Because it is already quite an extensive task to come up with simplified models for so many (production)$\times$(decay) modes, we for now restrict ourselves to the ``filled'' entries in Tables  \ref{tab:neutral_LLP}-\ref{tab:color_LLP}. If you are interested in performing an experimental search or developing a simplified model library entry for one of the ``unfilled'' entries, please contact the conveners of the simplified model library found at the library link:~{\bf [provide link]}.

Note that in all of the simplified model proposals below, any particles \emph{not} present in the production or decay chain should have their masses set to a very large value ($M\gtrsim5$ TeV) to ensure they are sufficiently decoupled from direct production at the LHC.

\subsubsection{Neutral LLPs}

The instructions for simulating the simplified model channels for neutral LLPs are given as follows:~Double Pair Production (DPP) in Table \ref{tab:DPP_neutral_library}, Heavy Parent (HP, QCD-charged parent) in Table \ref{tab:HP_QCD_neutral_library}.

 We then proceed to the Higgs (HIG) production modes in Tables \ref{tab:Higgs_neutral_library}-\ref{tab:Higgs_single_neutral_library}.

For the $Z/Z'$ (ZP) production modes, we run into an issue:~many of the fermion LLPs and decay modes are implemented in the MSSM, but the vanilla MSSM model does not contain a $Z'$. It does, however, contain a $Z$:~by modifying the $Z$ mass to whatever value is needed for the $Z'$, the MSSM model can be re-purposed for many of the ZP modes. At this point, it is worth reiterating that these strategies are for \emph{simulating signal events alone}, from which cross section sensitivities can be estimated based on backgrounds, efficiencies, and luminosity. The cross sections from the ``hacked'' ZP event generation processes themselves should not be used. In the future, it may be preferable to create or use an extended version of the MSSM which explicitly includes an additional U(1) gauge boson, which would obviate the need for any such manipulation of the SM Z mass. We provide instructions for the $Z/Z'$ (ZP) production modes in Tables \ref{tab:Zp_neutral_library}-\ref{tab:Zp_single_neutral_library}.

Finally, we provide instructions for the charged-current (CC) production modes in Table \ref{tab:CC_neutral_library}. This production mode is most easily simulated using a left-right symmetric model or other right-handed-neutrino model.

\begin{table}
\begin{center}
\begin{tabular}{ |c|l|} 
 \hline
Decay Mode & Simplified Model Library Process \\
\hline\hline
$X\rightarrow \gamma+$inv. & MSSM+GMSB. LLP is a bino $(\tilde\chi^0)$ produced due to $pp\rightarrow \tilde{\chi}^0\tilde{\chi}^0$ via $t$-channel squark   \\
&   exchange ($M_{\tilde q}>5$ TeV). Bino decays to photon + gravitino, $\tilde\chi^0\rightarrow \gamma+\tilde{G}$. \\
\hline
$X\rightarrow jj$& MSSM+RPV. LLP is sneutrino LSP $(\tilde\nu)$ that is pair-produced via weak gauge interactions.  \\
& $\tilde\nu \rightarrow q\bar q$  via the $QLd^{\rm c}$ operator.\\
\hline
%$X\rightarrow jj$+inv.& MSSM+split SUSY. LLP is wino LSP $\tilde\chi^0$ that is pair-produced via weak gauge interactions.  \\
%&  $\tilde\chi^0\rightarrow q\bar{q}\nu$  via an off-shell sfermion and the  $QLd^{\rm c}$ operator.\\
$X\rightarrow jj$+inv.& MSSM. LLP is second neutralino (wino) LSP $\tilde\chi_2^0$ that is pair-produced via   \\
&  weak gauge interactions. $\tilde\chi_2^0\rightarrow q\bar{q}\tilde\chi_1^0$  via an off-shell sfermion, and the $\tilde\chi_1^0$ is invisible  \\
& with arbitrary mass.\\
\hline
$X\rightarrow jjj$ & MSSM+RPV. While this is partially covered by $jj+\mathrm{inv.}$ in the case where the additional  \\
& quark is not  reconstructed, we include it here for completeness. LLP is wino LSP $(\tilde\chi^0)$     \\
&  that is pair-produced via weak interactions. $\tilde\chi^0\rightarrow q_\alpha q_\alpha q_\beta$ via an off-shell sfermion and  \\
& the $u_\alpha^{\rm c}d_\alpha^{\rm c}d_\beta^{\rm c}$ operator.\\
\hline
$X\rightarrow jj \ell_\alpha$ & MSSM+RPV. LLP is wino LSP ($\tilde\chi^0$) that is pair-produced via weak interactions.  \\
&  $\tilde\chi^0\rightarrow \ell_\alpha q\bar q$ via an off-shell sfermion and $L_\alpha Qd^{\rm c}$ operator.\\
\hline
$X\rightarrow \ell_\alpha^+\ell_\alpha^-$ & MSSM+RPV. LLP is sneutrino $\tilde \nu_\beta$ of flavor $\beta$ that is pair-produced via weak  interactions. \\
& $\nu_\beta\rightarrow \ell_\alpha^+\ell_\alpha^-$ via the $L_\alpha L_\beta E_\alpha^{\rm c}$ operator. \\
\hline
$X\rightarrow \ell_\alpha^+\ell_\alpha^-$(+inv.) & MSSM. LLP is second neutralino $\tilde \chi_2^0$ that is pair-produced via weak  \\
&  interactions. $\tilde\chi_2^0\rightarrow\tilde\chi_1^0\ell_\alpha^+\ell_\alpha^-$ via an off-shell slepton.\\
\hline
$X\rightarrow \ell_\alpha^+\ell_\beta^-$(+inv.) & MSSM+RPV. LLP is sneutrino $\tilde \nu_\alpha$ of flavor $\alpha$ that is pair-produced via weak   interactions.  \\
& $\nu_\alpha\rightarrow \ell_\alpha^+\ell_\beta^-$ via the $L_\alpha L_\beta E_\alpha^{\rm c}$ operator. An additional massless invisible final state can be  \\
&  obtained with a wino LLP decaying into $\ell_\alpha^+\ell_\beta^-\nu_\alpha$ through the same operator and an \\
& off-shell slepton. The massive invisible case is less  motivated for $\alpha\neq\beta$.\\
\hline
\end{tabular}
\end{center}
\caption{Simplified model library process proposals for Double Pair Production (DPP) production mode. Where a ``wino'' LSP is specified, an admixture of Higgsino is required to lead to direct pair production of the neutral wino component. As an alternative, one could have $pp\rightarrow \tilde\chi^\pm \tilde\chi^0$, $\tilde\chi^\pm \rightarrow {W^\pm}^* \tilde\chi^0$ promptly, and take the $\tilde\chi^\pm$ to be degenerate with $\tilde\chi^0$ such that the additional charged decay products are essentially unobservable.  }\label{tab:DPP_neutral_library}
\end{table}

\begin{table}
\begin{center}
\begin{tabular}{ |c|l|} 
 \hline
Decay Mode & Simplified Model Library Process \\
\hline\hline
$X\rightarrow \gamma+$inv. & MSSM+GMSB. LLP is a bino $(\tilde\chi)$ produced via $pp\rightarrow \tilde{q}\tilde{q}^*$, $\tilde{q}\rightarrow \tilde\chi+q$. Bino decays to   \\
&     photon+ gravitino, $\tilde\chi\rightarrow \gamma+\tilde{G}$. \\
\hline
$X\rightarrow jj$& MSSM+RPV. LLP is squark LSP $(\tilde q)$ produced via gluino pair production, $pp\rightarrow \tilde g\tilde g$.    \\
& The gluino decays via $\tilde g\rightarrow \tilde{q} \bar q$ and $\tilde{q}\rightarrow qq$ via the $u^{\rm c}_\alpha d^{\rm c}_\alpha d^{\rm c}_\beta$ operator.\\
\hline
$X\rightarrow jj$+inv.& MSSM. LLP is wino LSP $\tilde\chi_2^0$ that is produced via $pp\rightarrow \tilde q\tilde q^*$, $\tilde{q}\rightarrow q\tilde{\chi}_2^0$.    \\
&  Then,  $\tilde\chi_2^0\rightarrow q\bar{q}\tilde\chi_1^0$ via an   off-shell quark.\\
\hline
$X\rightarrow jjj$ & MSSM+RPV. LLP is bino LSP $\tilde\chi$ that is produced via $pp\rightarrow \tilde q\tilde q^*$, $\tilde{q}\rightarrow q\tilde{\chi}$. Then,   \\
& $\tilde\chi\rightarrow q_\alpha q_\alpha q_\beta$  via the $u_\alpha^{\rm c}d_\alpha^{\rm c}d_\beta^{\rm c}$ operator.\\
\hline
$X\rightarrow jj \ell_\alpha$ & MSSM+RPV. LLP is bino LSP ($\tilde\chi$) that is produced via $pp\rightarrow \tilde q\tilde q^*$, $\tilde q\rightarrow q\tilde \chi$. $\tilde\chi\rightarrow \ell_\alpha q\bar q$  \\
&   via an off-shell sfermion and $L_\alpha Qd^{\rm c}$ operator.\\
\hline
$X\rightarrow \ell_\alpha^+\ell_\alpha^-$ or $\ell_\alpha^+\ell_\beta^-$ & MSSM+RPV. LLP is sneutrino ($\tilde \nu$) that is produced via $pp\rightarrow \tilde{g}\tilde g$, $\tilde g\rightarrow jj \tilde\chi$, $\tilde\chi\rightarrow \tilde\nu \bar\nu$. \\
&  Then, $\nu_\alpha\rightarrow \ell_\alpha^+\ell_\beta^-$ or $\nu_\beta\rightarrow \ell_\alpha^+\ell_\alpha^-$ via the $L_\alpha L_\beta E_\alpha^{\rm c}$ operator. \\
\hline
$X\rightarrow \ell_\alpha^+\ell_\alpha^-$+inv.& MSSM. LLP is second neutralino ($\tilde \chi_2^0$) that is produced via $pp\rightarrow \tilde{q}\tilde q^*$,   \\
 &  $\tilde q\rightarrow q\tilde\chi_2^0$. Then, $\tilde\chi_2^0\rightarrow \ell_\alpha^+\ell_\alpha^- \tilde\chi_1^0$.  \\
 \hline
 $X\rightarrow \ell_\alpha^+\ell_\beta^-$+inv.& MSSM+RPV. LLP is bino ($\tilde \chi^0$) that is produced via $pp\rightarrow \tilde{q}\tilde q^*$, $\tilde q\rightarrow q\tilde\chi^0$. Then,  \\
 &  $\tilde\chi^0\rightarrow \ell_\alpha^+\ell_\beta^- \nu_\alpha$ via $L_\alpha L_\beta E^{\rm c}_\alpha$ operator and off-shell slepton (massless invisible only).  \\
\hline
\end{tabular}
\end{center}
\caption{Simplified model library process proposals for Heavy Parent (HP) production mode where the parent particle carries a QCD charge. In most of the above cases, a squark parent can be replaced by a gluino parent with an additional jet in its decay.  }\label{tab:HP_QCD_neutral_library}
\end{table}

\begin{table}
\begin{center}
\begin{tabular}{ |c|l|} 
 \hline
Decay Mode & Simplified Model Library Process \\
\hline\hline
$X\rightarrow \gamma+$inv. & MSSM+GMSB. LLP is a bino $(\tilde\chi^0)$ produced via $pp\rightarrow \tilde{\chi}^+\tilde{\chi}^-$, $\tilde \chi^+ \rightarrow W^+\tilde\chi^0$ ($\tilde\chi^+$ is a     \\
&   wino). Bino decays to  photon+ gravitino, $\tilde\chi\rightarrow \gamma+\tilde{G}$. \\
\hline
$X\rightarrow jj$& MSSM+RPV. LLP is sneutrino LSP $(\tilde \nu)$ produced via $pp\rightarrow \tilde\chi^+\tilde\chi^-$, $\tilde\chi^+\rightarrow \tilde\nu \ell^+$. The    \\
&   sneutrino decays via $\tilde \nu\rightarrow q\bar{q}$  via the $u^{\rm c}_\alpha d^{\rm c}_\alpha d^{\rm c}_\beta$ operator.\\
\hline
$X\rightarrow jj$+inv.& MSSM. LLP is wino $\tilde\chi_2^0$ that is produced via $pp\rightarrow\tilde\chi^+\tilde\chi^-$, $\tilde\chi_2^+\rightarrow W^+\tilde\chi_2^0$.   \\
&  Then,   $\tilde\chi_2^0\rightarrow q\bar{q}\tilde\chi_1^0$  via an  off-shell  squark.\\
\hline
$X\rightarrow jjj$ & MSSM+RPV. LLP is bino LSP ($\tilde\chi^0$) that is produced via $pp\rightarrow \tilde\chi^+\tilde\chi^-$, $\tilde \chi^+\rightarrow W^+\tilde\chi^0$.  \\
&  Then, $\tilde\chi^0\rightarrow qqq$ via  an off-shell sfermion and the $u^{\rm c}d^{\rm c}d^{\rm c}$ operator.\\
\hline
$X\rightarrow jj \ell_\alpha$ & MSSM+RPV. LLP is bino LSP ($\tilde\chi^0$) that is produced via $pp\rightarrow \tilde\chi^+\tilde\chi^-$, $\tilde \chi^+\rightarrow W^+\tilde\chi^0$.  \\
&  Then, $\tilde\chi^0\rightarrow qq'\ell_\alpha$ via  an off-shell sfermion and the $Qd^{\rm c}L_\alpha$ operator.\\
\hline
$X\rightarrow \ell_\alpha^+\ell_\alpha^-$ or $\ell_\alpha^+\ell_\beta^-$ & MSSM+RPV. LLP is sneutrino ($\tilde \nu$) that is produced via $pp\rightarrow \tilde\chi^+\tilde\chi^-$, $\tilde\chi^+\rightarrow \ell^+\tilde \nu$. Then, \\
& $\nu_\alpha\rightarrow \ell_\alpha^+\ell_\beta^-$ or $\nu_\beta\rightarrow \ell_\alpha^+\ell_\alpha^-$ via the $L_\alpha L_\beta E_\alpha^{\rm c}$ operator. \\
\hline
$X\rightarrow \ell_\alpha^+\ell_\alpha^-$+inv.& MSSM. LLP is second neutralino ($\tilde \chi_2^0$) that is produced via $pp\rightarrow \tilde\chi^+\tilde\chi^-$,  \\
 & $\tilde \chi^+\rightarrow W^+\tilde\chi_2^0$. Then,  $\tilde\chi_2^0\rightarrow \ell_\alpha^+\ell_\alpha^- \tilde\chi_1^0$ via an off-shell slepton and the $L_\alpha L_\beta E_\alpha^{\rm c}$ operator. \\
\hline
$X\rightarrow \ell_\alpha^+\ell_\beta^-$+inv. & MSSM+RPV. LLP is bino ($\tilde \chi^0$) that is produced via $pp\rightarrow \tilde\chi^+\tilde\chi^-$, $\tilde \chi^+\rightarrow W^+\tilde\chi^0$. Then,  \\
 & $\tilde\chi^0\rightarrow \ell_\alpha^+\ell_\alpha^- \nu_\beta$ or $\tilde\chi^0\rightarrow \ell_\alpha^+\ell_\beta^-\nu_\alpha$ via an off-shell slepton and the $L_\alpha L_\beta E_\alpha^{\rm c}$ operator. \\
\hline
\end{tabular}
\end{center}
\caption{Simplified model library process proposals for Heavy Parent (HP) production mode where the parent particle carries electroweak charge. }\label{tab:HP_EW_neutral_library}
\end{table}

\begin{table}
\begin{center}
\begin{tabular}{ |c|l|} 
 \hline
Decay Mode & Simplified Model Library Process \\
\hline\hline
$X\rightarrow \gamma\gamma$ & HAHM. LLP is the singlet scalar Higgs ($h_{\rm D}$)   produced via $pp\rightarrow h$, $h\rightarrow h_{\rm D}h_{\rm D}$. Then,    \\
&  $h_{\rm D}\rightarrow\gamma\gamma$ via mixing with the SM Higgs. \\
\hline
$X\rightarrow \gamma\gamma$+inv. & MSSM. LLP is the second neutralino  ($\tilde\chi_2^0$)   produced via $pp\rightarrow h$, $h\rightarrow \tilde\chi_2^0\tilde\chi_2^0$.    \\
&  Then,  $\tilde\chi_2^0\rightarrow\tilde\chi_1^0\gamma\gamma$ via an off-shell SM Higgs. \\
\hline
$X\rightarrow jj$& HAHM. LLP is singlet scalar Higgs $(h_{\rm D})$ produced via  $pp\rightarrow h$, $h\rightarrow h_{\rm D}h_{\rm D}$. Then,      \\
&   $h_{\rm D}\rightarrow j j$ via mixing with the SM Higgs.\\
\hline
$X\rightarrow jj$+inv.& MSSM. LLP is wino $\tilde\chi_2^0$ that is produced via $pp\rightarrow h $, $h\rightarrow \tilde\chi_2^0\tilde\chi_2^0$.  Then, \\
&  $\tilde\chi_2^0\rightarrow jj\tilde\chi_1^0$  via an  off-shell  squark.\\
\hline
$X\rightarrow \ell_\alpha^+\ell_\alpha^-$ & HAHM. LLP is hidden gauge boson ($Z_{\rm D}$) that is produced via $pp\rightarrow h$, $h\rightarrow Z_{\rm D}Z_{\rm D}$.  \\
&  Then, $Z_{\rm D}\rightarrow \ell_\alpha^+\ell_\alpha^-$ via  mixing with the SM gauge bosons.\\
\hline
$X\rightarrow \ell_\alpha^+\ell_\alpha^-$+inv. & MSSM.  LLP is the second neutralino  ($\tilde\chi_2^0$)   produced via $pp\rightarrow h$, $h\rightarrow \tilde\chi_2^0\tilde\chi_2^0$.   \\
&  Then, $\tilde\chi_2^0\rightarrow\ell_\alpha^+\ell_\alpha^-\tilde\chi^0_1$ via  an off-shell slepton.\\
\hline
\end{tabular}
\end{center}
\caption{Simplified model library process proposals for Higgs (HIG) production mode where the Higgs decays to two LLPs. These modes are particularly important because they can come in association with forward jets (VBF) or leptons and $\slashed{E}_{\rm T}$ (VH). Note that, in cases of $M_X>M_h/2$, the same production modes could occur with an off-shell SM Higgs. }\label{tab:Higgs_neutral_library}
\end{table}

\begin{table}
\begin{center}
\begin{tabular}{ |c|l|} 
 \hline
Decay Mode & Simplified Model Library Process \\
\hline\hline
$X\rightarrow \gamma\gamma$ & (N)MSSM. LLP is a pseudoscalar or singlino ($a$)   produced via $pp\rightarrow h$, $h\rightarrow \tilde\chi_2^0\tilde\chi_2^0$,    \\
& $\tilde\chi_2^0\rightarrow\tilde\chi_1^0a$. Finally, $a\rightarrow\gamma\gamma$. \\
\hline
$X\rightarrow \gamma\gamma$+inv. & MSSM. LLP is the second neutralino  ($\tilde\chi_2^0$)   produced via $pp\rightarrow h$, $h\rightarrow \tilde\nu\tilde\nu^*$,    \\
&  $\tilde\nu\rightarrow\tilde\chi_2^0\nu$. Then,  $\tilde\chi_2^0\rightarrow\tilde\chi_1^0\gamma\gamma$ via an off-shell SM Higgs. \\
\hline
$X\rightarrow jj$& MSSM+RPV. LLP is a sneutrino $(\tilde\nu)$ produced via  $pp\rightarrow h$, $h\rightarrow \tilde\chi_1^0\tilde\chi_1^0$, $\tilde\chi_1^0\rightarrow \tilde\nu\bar\nu$.      \\
&   Then, $\tilde\nu\rightarrow jj$  via the RPV operator $LQd^{\rm c}$.\\
\hline
$X\rightarrow jj$+inv.& MSSM. LLP is the second neutralino ($\tilde\chi_2^0$) that is produced via $pp\rightarrow h $,  \\
& $h\rightarrow \tilde\nu\tilde\nu^*$, $\tilde\nu\rightarrow \nu\tilde\chi_2^0$.  Then, $\tilde\chi_2^0\rightarrow jj\tilde\chi_1^0$  via an  off-shell  squark.\\
\hline
$X\rightarrow \ell_\alpha^+\ell_\alpha^-$ &  MSSM+RPV. LLP is a sneutrino $(\tilde\nu_\beta)$ produced via  $pp\rightarrow h$, $h\rightarrow \tilde\chi_1^0\tilde\chi_1^0$, $\tilde\chi_1^0\rightarrow \tilde\nu_\beta\bar\nu_\beta$.     \\
&    Then, $\tilde\nu_\beta\rightarrow \ell_\alpha^+\ell_\alpha^-$  via the RPV operator $L_\alpha L_\beta E^{\rm c}_\alpha$.\\
\hline
$X\rightarrow \ell_\alpha^+\ell_\alpha^-$+inv. & MSSM. LLP is the second neutralino ($\tilde\chi_2^0$) that is produced via $pp\rightarrow h $,  \\
& $h\rightarrow \tilde\nu\tilde\nu^*$, $\tilde\nu\rightarrow \nu\tilde\chi_2^0$.  Then, $\tilde\chi_2^0\rightarrow \ell_\alpha^+\ell_\alpha^-\tilde\chi_1^0$  via an  off-shell  slepton.\\

\hline
\end{tabular}
\end{center}
\caption{Simplified model library process proposals for Higgs (HIG) production mode where the Higgs decays to two LLPs plus invisible. These modes are particularly important because they can come in association with forward jets (VBF) or leptons and $\slashed{E}_{\rm T}$ (VH). Note that, in cases of $M_X>M_h/2$, the same production modes could occur with an off-shell SM Higgs. }\label{tab:Higgs_inv_neutral_library}
\end{table}

\begin{table}
\begin{center}
\begin{tabular}{ |c|l|} 
 \hline
Decay Mode & Simplified Model Library Process \\
\hline\hline
$X\rightarrow \gamma\gamma$+inv. & MSSM. LLP is the second neutralino  ($\tilde\chi_2^0$)   produced via $pp\rightarrow h$, $h\rightarrow\tilde\chi_2^0\tilde\chi_1^0$.  Then,  \\
&    $\tilde\chi_2^0\rightarrow\tilde\chi_1^0\gamma\gamma$ via an off-shell SM Higgs. \\
\hline
$X\rightarrow jj$+inv.& MSSM. LLP is the second neutralino ($\tilde\chi_2^0$) that is produced via $pp\rightarrow h $,  \\
& $h\rightarrow \tilde\chi_2^0\tilde\chi_1^0$. Then, $\tilde\chi_2^0\rightarrow jj\tilde\chi_1^0$  via an  off-shell  squark.\\
\hline
$X\rightarrow \ell_\alpha^+\ell_\alpha^-$+inv. & MSSM. LLP is the second neutralino ($\tilde\chi_2^0$) that is produced via $pp\rightarrow h $,  \\
& $h\rightarrow \tilde\chi_2^0\tilde\chi_1^0$.  Then, $\tilde\chi_2^0\rightarrow \ell_\alpha^+\ell_\alpha^-\tilde\chi_1^0$  via an  off-shell  slepton.\\

\hline
\end{tabular}
\end{center}
\caption{Simplified model library process proposals for Higgs (HIG) production mode where the Higgs decays to single LLP plus invisible. These modes are particularly important because they can come in association with forward jets (VBF) or leptons and $\slashed{E}_{\rm T}$ (VH). Note that, in cases of $M_X>M_h/2$, the same production modes could occur with an off-shell SM Higgs. }\label{tab:Higgs_single_neutral_library}
\end{table}

\begin{table}
\begin{center}
\begin{tabular}{ |c|l|} 
 \hline
Decay Mode & Simplified Model Library Process \\
\hline\hline
$X\rightarrow \gamma\gamma$ & (N)MSSM. LLP is a degenerate pseudoscalar $a$ and scalar $s$   produced via $pp\rightarrow Z$,     \\
&  $Z\rightarrow as$. Finally, $a\rightarrow\gamma\gamma$ and $s\rightarrow \gamma\gamma$. Modify $Z$ mass to desired $Z'$ mass. \\
\hline
$X\rightarrow \gamma\gamma$+inv. & MSSM. LLP is a mixed Higgsino/bino  ($\tilde\chi_2^0$)   produced via $pp\rightarrow Z$, $Z\rightarrow \tilde\chi_2^0\tilde\chi_2^0$.    \\
&  Then,  $\tilde\chi_2^0\rightarrow\tilde\chi_1^0\gamma\gamma$ via an off-shell SM Higgs. \\
\hline
$X\rightarrow jj$& MSSM+RPV. LLP is a sneutrino $(\tilde\nu)$ produced via  $pp\rightarrow Z$, $Z\rightarrow \tilde\nu\tilde\nu^*$.  Then, $\tilde\nu\rightarrow jj$    \\
&    via the RPV operator $LQd^{\rm c}$.\\
\hline
$X\rightarrow jj$+inv.& MSSM. LLP is a mixed Higgsino/bino ($\tilde\chi_2^0$) that is produced via $pp\rightarrow Z $,  $Z\rightarrow \tilde\chi_2^0\tilde\chi_2^0$.\\
&   Then, $\tilde\chi_2^0\rightarrow jj\tilde\chi_1^0$  via an  off-shell  squark.\\
\hline
$X\rightarrow \ell^+\ell$ &  MSSM. sneutrino $(\tilde\nu)$ produced via  $pp\rightarrow Z$, $Z\rightarrow \tilde\nu\tilde\nu^*$.  Then, $\tilde\nu_\beta\rightarrow \ell_\alpha^+\ell_\alpha^-$ or \\
&    $\tilde\nu_\beta\rightarrow \ell_\alpha^+\ell_\alpha^-$ (depending on flavor structure)  via the RPV operator $L_\alpha L_\beta E^{\rm c}_\alpha$.\\
\hline
$X\rightarrow \ell_\alpha^+\ell_\alpha^-$+inv. & MSSM. LLP is a mixed Higgsino/bino ($\tilde\chi_2^0$) that is produced via $pp\rightarrow Z $, $Z\rightarrow\tilde\chi_2^0\tilde\chi_2^0$.  \\
& Then, $\tilde\chi_2^0\rightarrow \ell_\alpha^+\ell_\alpha^-\tilde\chi_1^0$  via an  off-shell  slepton.\\

\hline
\end{tabular}
\end{center}
\caption{Simplified model library process proposals for $Z/Z'$ (ZP) production mode where the $Z'$ decays to two LLPs. For this section, we typically use an MSSM model {\bf with a modified $Z$ mass}, and use this ``$Z$'' as our $Z'$. Note that, in cases of $M_X>M_{Z'}/2$, the same production modes could occur with an off-shell SM Higgs.  }\label{tab:Zp_neutral_library}
\end{table}

\begin{table}
\begin{center}
\begin{tabular}{ |c|l|} 
 \hline
Decay Mode & Simplified Model Library Process \\
\hline\hline
$X\rightarrow \gamma\gamma$ & (N)MSSM.LLP is a pseudoscalar or singlino ($a$)   produced via $pp\rightarrow Z$, $Z\rightarrow \tilde\chi_2^0\tilde\chi_2^0$,    \\
& $\tilde\chi_2^0\rightarrow\tilde\chi_1^0a$. Finally, $a\rightarrow\gamma\gamma$. \\
\hline
$X\rightarrow \gamma\gamma$+inv. & MSSM. LLP is the second neutralino  ($\tilde\chi_2^0$)   produced via $pp\rightarrow Z$, $Z\rightarrow \tilde\nu\tilde\nu^*$,    \\
&  $\tilde\nu\rightarrow\tilde\chi_2^0\nu$. Then,  $\tilde\chi_2^0\rightarrow\tilde\chi_1^0\gamma\gamma$ via an off-shell SM Higgs. \\
\hline
$X\rightarrow jj$& MSSM+RPV. LLP is a sneutrino $(\tilde\nu)$ produced via  $pp\rightarrow Z$, $Z\rightarrow \tilde\chi_1^0\tilde\chi_1^0$, $\tilde\chi_1^0\rightarrow \tilde\nu\bar\nu$.      \\
&   Then, $\tilde\nu\rightarrow jj$  via the RPV operator $LQd^{\rm c}$.\\
\hline
$X\rightarrow jj$+inv.& MSSM. LLP is a second neutralino ($\tilde\chi_2^0$) that is produced via $pp\rightarrow Z $,  \\
& $Z\rightarrow \tilde\nu\tilde\nu^*$, $\tilde\nu\rightarrow \nu\tilde\chi_2^0$.  Then, $\tilde\chi_2^0\rightarrow jj\tilde\chi_1^0$  via an  off-shell  squark.\\

\hline
$X\rightarrow \ell_\alpha^+\ell_\alpha^-$ &  MSSM+RPV. LLP is a sneutrino $(\tilde\nu_\beta)$ produced via  $pp\rightarrow Z$, $Z\rightarrow \tilde\chi_1^0\tilde\chi_1^0$, $\tilde\chi_1^0\rightarrow \tilde\nu_\beta\bar\nu_\beta$.     \\
&    Then, $\tilde\nu_\beta\rightarrow \ell_\alpha^+\ell_\alpha^-$  via the RPV operator $L_\alpha L_\beta E^{\rm c}_\alpha$.\\
\hline
$X\rightarrow \ell_\alpha^+\ell_\alpha^-$+inv. & MSSM. LLP is the second neutralino ($\tilde\chi_2^0$) that is produced via $pp\rightarrow Z $,  \\
& $Z\rightarrow \tilde\nu\tilde\nu^*$, $\tilde\nu\rightarrow \nu\tilde\chi_2^0$.  Then, $\tilde\chi_2^0\rightarrow \ell_\alpha^+\ell_\alpha^-\tilde\chi_1^0$  via an  off-shell  slepton.\\

\hline
\end{tabular}
\end{center}
\caption{Simplified model library process proposals for $Z/Z'$ (ZP) production mode where the $Z'$ decays to two LLPs plus invisible. For this section, we typically use an MSSM model {\bf with a modified $Z$ mass}, and use this ``$Z$'' as our $Z'$. Note that, in cases of $M_X>M_{Z'}/2$, the same production modes could occur with an off-shell SM Higgs.  }\label{tab:Zp_inv_neutral_library}
\end{table}

\begin{table}
\begin{center}
\begin{tabular}{ |c|l|} 
 \hline
Decay Mode & Simplified Model Library Process \\
\hline\hline
$X\rightarrow \gamma\gamma$+inv. & MSSM. LLP is the second neutralino  ($\tilde\chi_2^0$)   produced via $pp\rightarrow Z$, $Z\rightarrow\tilde\chi_2^0\tilde\chi_1^0$.  Then,  \\
&    $\tilde\chi_2^0\rightarrow\tilde\chi_1^0\gamma\gamma$ via an off-shell SM Higgs. \\
\hline
$X\rightarrow jj$+inv.& MSSM. LLP is the second neutralino ($\tilde\chi_2^0$) that is produced via $pp\rightarrow Z $,  \\
& $Z\rightarrow \tilde\chi_2^0\tilde\chi_1^0$. Then, $\tilde\chi_2^0\rightarrow jj\tilde\chi_1^0$  via an  off-shell  squark.\\
\hline
$X\rightarrow \ell_\alpha^+\ell_\alpha^-$+inv. & MSSM. LLP is the second neutralino ($\tilde\chi_2^0$) that is produced via $pp\rightarrow Z$,  \\
& $Z\rightarrow \tilde\chi_2^0\tilde\chi_1^0$.  Then, $\tilde\chi_2^0\rightarrow \ell_\alpha^+\ell_\alpha^-\tilde\chi_1^0$  via an  off-shell  slepton.\\

\hline
\end{tabular}
\end{center}
\caption{Simplified model library process proposals for $Z/Z'$ (ZP) production mode where the $Z'$ decays to single LLP plus invisible. For this section, we typically use an MSSM model {\bf with a modified $Z$ mass}, and use this ``$Z$'' as our $Z'$. Note that, in cases of $M_X>M_{Z'}/2$, the same production modes could occur with an off-shell SM Higgs. }\label{tab:Zp_single_neutral_library}
\end{table}

\begin{table}
\begin{center}
\begin{tabular}{ |c|l|} 
 \hline
Decay Mode & Simplified Model Library Process \\
\hline\hline
$X\rightarrow jj$+inv. & LR. LLP is the right-handed neutrino  ($\nu_{\rm R}$)   produced via $pp\rightarrow W^\pm$, $W^\pm\rightarrow \ell^\pm \nu_{\rm R}$.    \\
& Then,   $\nu_{\rm R}\rightarrow q\bar{q}\nu$ via an off-shell $W$. For massive invisible state, it may be possible   \\
&  to use a cascade $\nu_{\rm R2}\rightarrow q\bar{q}\nu_{\rm R1}$ treating the lightest right-handed neutrino as stable. \\
\hline
$X\rightarrow jj\ell^\pm$& LR. LLP is the right-handed neutrino  ($\nu_{\rm R}$)   produced via $pp\rightarrow W^\pm$, $W^\pm\rightarrow \ell^\pm \nu_{\rm R}$.  \\
& Then,   $\nu_{\rm R}\rightarrow q\bar{q}'\ell^\pm$ via an off-shell $W$.\\
\hline
$X\rightarrow \ell_\alpha^+\ell_\alpha^-$+inv. & LR.LLP is the right-handed neutrino  ($\nu_{\rm R}$)   produced via $pp\rightarrow W^\pm$, $W^\pm\rightarrow \ell^\pm \nu_{\rm R}$.  \\
or $X\rightarrow \ell_\alpha^+\ell_\beta^-$+inv.  & Then,   $\nu_{\rm R}\rightarrow \ell_\alpha^+\ell_\alpha^-\nu_\beta$ or $\nu_{\rm R}\rightarrow \ell_\alpha^+\ell_\beta^-\nu_\alpha$ via an off-shell $W/Z$.\\

\hline
\end{tabular}
\end{center}
\caption{Simplified model library process proposals for charged current (CC) production mode, $W_{\rm SM}^\pm/{W'}^\pm\rightarrow X+\ell^\pm$; these can be simulated using left-right symmetric models using either the $W$ or $W'$ (for simplicity, in the table above we only state explicitly $W$). Right-handed neutrino lifetimes are most naturally long for sub-weak-scale masses. }\label{tab:CC_neutral_library}
\end{table}


\section{Future Opportunities and Challenges}\label{sec:simplified_future}
\begin{itemize}
\item Focused on the simplest and best-motivated models for now. Already many channels to consider! But this can be expanded as more work is done to fill in gaps
\item Focused on low-multiplicity signatures (so does not include dark showers, part of ongoing working group)
\item Cannot cover every single model! Some searches may need to be sensitive to detailed features of a specific model, particularly as we push to be more background-dominated search regions.
\item Tackle increased multiplicities due to $b$ vs.~light-flavor jet, different lepton flavors?
\end{itemize}
