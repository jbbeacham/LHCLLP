

New particles with long-lifetimes arise in many proposed extensions of the standard model.  In some cases, experimental searches have targeted these new particles; in others, proposals have been made for new techniques to reconstruct LLPs in particular models.  
It is well-known that LHC searches for new physics, such as models of supersymmetry or dark matter, are sufficiently general that a new dedicated search is typically not necessary in order to cover every new proposed model.  The exceptional success of the existing LHC program informs us that it is desirable to develop search strategies that ensure experimental results are broadly applicable to different types of models, reduce redundancy among experimental searches, and illuminate gaps in coverage and areas where new searches are needed. 

The most commonly adopted framework for accomplishing these goals is the simplified model framework. The simplified model framework uses the fact that many searches are only sensitive to a few broad aspects of a particle's signature (such as the production mode, production rate, and decay topology) and not precise details such as the spin of the particle, its angular distribution, etc. With LLP signatures, the simplified models approach is even \emph{more} appropriate:~this is because searches tend to have low backgrounds and relatively inclusive searches can be done, each of which has sensitivity to LLPs with many different types of production and decay modes.


Because LLP particles are often produced and decay (if at all) in physically distinct locations, the production and decay processes can be factorized.\footnote{In addition to production and decay, a third consideration is the propagation of particles through the detector.  While neutral or only electrically charged particles have straight-forward propagation, colored states, \emph{e.g.}, SUSY $R$-hadrons, or particles with exotic charges, such as magnetic monopoles or quirks typically engage in a more complicated and often very uncertain traverse.  The subtleties with colored LLPs will be discussed more in subsection \ref{sec:coloredLLPs}, and exotic, conserved charges will not be discussed here.}  
We can exploit this to construct a simple basis of LLP production modes and LLP decay modes; we define an {\bf LLP channel} as a combination of a particular production and decay mode, with the lifetime of the LLP taken as a free parameter. We emphasize that the LLP channel as defined here is \emph{not} the same as an experimental signature that manifests in the detector:~a single simplified model channel could give rise to many, many different signatures depending on where the decays in occur in the detector (or outside), while a single experimental search for a particular signature could cover many simplified model channels for particular choices of parameters. We focus here on the theory definition, which provides some control over the range of models that we expect to see at the LHC, while in Section \ref{sec:experimentcoverage} we map our basis of Simplified Models onto existing searches to determine the gaps in coverage and proposals for new searches.

The list of simplified models presented here is necessarily incomplete:~the present goal is to provide a set of simplified models that covers many of the best motivated and simplest models containing LLPs. However, simplified models cannot include all of the specific details and subtle features found in a complete model. Therefore, the provided list is meant to be expandable to cover new or more complicated models as the LLP search program develops. The simplified model framework is also not ideally suited to high-multiplicity signatures such as dark showers or emerging jets, whose phenomenology depends crucially on model parameters such as strong couplings, hadronization details, and  spectra and lifetime of large numbers of new hidden-sector particles; we therefore focus in this section on production of one or two LLPs, leaving a discussion of future opportunities in Section \ref{sec:simplified_future} and dark shower signatures in Section \ref{sec:showers}.

\section{Goals of the Simplified Model Framework}
The purpose of the simplified model framework is to provide a simple, common language that experimentalists and theorists can use to describe LLP theories and the corresponding mapping between models and experimental signatures. We therefore want our simplified model space to:
%
\begin{enumerate}
%
\item Cover the most interesting theories of LLPs with a minimal set of models;
\item Map easily between models and signatures so that the current coverage and gaps may be easily identified;
\item Apply to theories and signatures not yet proposed;
\item Expand flexibly to incorporate theories and signatures not yet proposed;
\item Provide a concrete Monte Carlo signal event generation framework;
\item Facilitate the re-interpretation of searches by supplying benchmark models\footnote{Note that more benchmark models may be needed for re-interpretation than are strictly necessary for discovering a new particle, \emph{i.e.,} we should be mindful both of whether two simplified models share a common signature in a search, and also whether they look similar enough to have similar reconstruction efficiencies} for which experimental efficiencies can be provided for validation purposes.
\end{enumerate}
%
We begin by discussing a simplified model framework for events with one and two LLPs.  Extending this  framework to models with dark showers will be discussed later.


\section{Existing Well-Motivated Theories for LLPs}\label{sec:motivated_theories}
Here, we categorize a few broad classes of models that have conventionally provided the best motivation for LLPs. We should emphasize that there are many theories beyond these that motivate LLPs; however, many such theories broadly fall under the umbrella of one of the following UV theories and so we can cite them but not give a detailed description.
%
\begin{itemize}
\item {\bf SUSY-like theories (SUSY):}~these are models with multiple new particles carrying SM gauge charges and a variety of allowed cascade decays. Lifetimes can be long due to approximate symmetries (such as $R$-parity or gauge mediation) and decays mediated by highly off-shell intermediaries (as in split SUSY).  Here we classify any non-SUSY models with heavy SM charged particles, such as composite Higgs or extra-dimensional models, under the SUSY-like umbrella.
\item {\bf Higgs-portal theories (Higgs):}~these include scenarios with exotic Higgs decays to low-mass particles  (as in many Hidden Valley scenarios), and is well motivated theoretically by models of Higgs-portal hidden sectors and neutral naturalness. Experimentally, this scenario is well motivated by the fact that the Higgs can still have a 30\% branching fraction to exotic final states and theoretically by the fact that new singlet particles can have large couplings to the Higgs. The Higgs is also special in that it comes with its own set of associated production modes, such as VBF or Higgs-strahlung in addition to the dominant gluon fusion process.
\item {\bf Gauge-portal theories (ZP):}~these include scenarios with new vector mediators decaying to exotic LLPs. These are similar to Higgs models, but where the mediator is predominantly produced from $q\bar{q}$-initiated final states without other associated objects. Examples include models with a new $Z'$ coupled to SM fermions, as well as ``dark'' photon or $Z$ models in which couplings of new vector bosons to the SM are mediated by kinetic mixing between the new gauge bosons and SM gauge bosons. Such models are particularly motivated in light of hidden sectors, dark matter, and their interactions with SM particles.
\item {\bf Dark-matter theories (DM):}~this class focuses on non-SUSY dark matter and hidden-sector scenarios, and encompasses models where dark matter is produced as a final state in the collider process. The main distinguishing feature from the Higgs and gauge scenarios above is that dark matter (missing momentum) is a necessary and irreducible component of each signature.
\item {\bf Heavy neutrinos and friends (RH$\nu$):}~if there exist new weak-scale states responsible for giving SM neutrinos mass, the new particles can typically be long lived. The signatures tend to be lepton-rich due to the connection with SM neutrino masses, and the long-lived states are often singly produced.
\end{itemize}
%
%As we develop our simplified model framework, we will construct maps between the UV models and the simplified model channels to show the highest priority/best motivated combinations of production and decay modes for LLPs. 
In developing our simplified model framework, we will construct maps between the UV models and the simplified model channels to illustrate some of the best motivated combinations of production and decay modes for LLPs. This will then allow us to focus on the most interesting scenarios and determine their coverage in particular parts of the model parameter space.

\section{The Simplified Model Building Blocks}\label{sec:building_blocks}

Recall that in LLP searches production and decay can be factorized. This allows us to specify the relevant production and decay modes for LLP models separately; we then put them together and map the space of models into the existing motivated theories.

\subsection{Production Modes}
According to the earlier models, we can identify a minimal set of interesting production modes for LLPs:
%
\begin{itemize}
\item {\bf Direct Pair Production (DPP):}~If the LLP is charged under a SM gauge interaction, it can be directly produced via the corresponding gauge boson. The production cross section is specified by the LLP gauge charge and mass. This also results from a $t$-channel mediator, and in this case the production cross section is a free parameter.
\item {\bf Heavy parent (HP):}~If the LLP can be produced in the decay of a heavy parent particle that is itself charged under the SM gauge interactions. The production cross section is essentially a free parameter and is indirectly specified by the charge and mass of the heavy parent. The heavy parent production gives a very different kinematics for the LLP than gauge production.
\item {\bf Higgs (HIG):}~The LLP can be produced in decays of the SM Higgs boson.  This case has an interesting interplay of possible production modes. The dominant production cross-section is  gluon fusion, which features no associated objects, but thanks to its role in EWSB, the Higgs has associated production modes (VBH, VH) with characteristic features. Note that if the LLP mass is heavier than $m_h/2$, it can also be produced via the off-shell Higgs portal. The LLP can be pair produced or singly produced (in association with MET). The cross section (or Higgs branching fraction) is a free parameter of the model.



\item {\bf Heavy resonance (ZP):}~Similar to the Higgs portal, but the LLP is produced in the decay of an on-shell resonance, such as a heavy $Z'$ gauge boson initiated by $q\bar{q}$ initial state or a heavy scalar\footnote{The properties of the observed light Higgs suggest that any new scalar would minimally impact EWSB and thus would have at most only a very small rate from VBF and VH processes, for this reason, we place heavy scalars in the heavy resonance model as opposed to the Higgs model above.}, $\Phi$.  Note that production via an off-shell resonance is kinematically similar to the direct production (DPP) above.  As with HIG, the LLP can be pair-produced or singly-produced (in association with MET). 
%\item {\bf  $Z'$ (ZP):}~Similar to the Higgs portal, but the LLP is produced in the decay of an on-shell gauge boson initiated by $q\bar{q}$ initial state. Note that production via off-shell $Z'$ is similar to if the LLP is directly charged under the SM $Z$ and so is included in GP above. As with HIG, the LLP can be pair produced or singly produced (in association with MET).
\item {\bf Charged current (CC):}~In the case of a right-handed neutrino, the LLP can be produced in the leptonic decays of $W/W'$. Single production is favored.
\end{itemize}
%
 As  mentioned above, a special feature of the Higgs-mediated LLP production modes is the presence of associated, prompt objects when the Higgs is produced via VBF or VH. These additional prompt objects can enhance the ability of experiments to trigger on and reconstruct the LLP decays, particularly when the LLP has a low mass. 
 
 There are other scenarios in which associated objects accompany the LLP production, such as the prompt lepton arising from charged-current production of neutral LLPs or in ``dark $Z$''-type scenarios, ISR jets in DM-motivated scenarios, and many other cases. The presence of associated prompt objects (such as a VBF or central jets, lepton, missing momentum, etc.) in many LLP models suggests that, where possible, LLP searches should be supplemented with signal regions that exploit the presence of the associated objects, extending sensitivity to otherwise hard-to-reach parts of parameter space.

\subsection{Decay Modes}
It is important to note that LLP searches are typically fairly inclusive. This is in part due to the fact that particle ID is less possible for decays far in the detector (\emph{e.g.,} for decays inside of the calorimeter, everything looks like a calorimeter deposition). It is also because backgrounds are often low enough that tight cuts typically found in exclusive analyses are not needed to suppress backgrounds. For example, ATLAS has a displaced vertex search sensitive to dilepton and multitrack vertices that are relatively agnostic to other objects originating from near the displaced vertex. Similarly, CMS has an analysis sensitive to high-impact-parameter leptons without reconstructing a vertex.  However, in some cases the topology of a decay does matter:~for example, one potentially important factor is distinguishing cases where the LLP decays into two SM objects vs.~three, because this determines whether its mass is resonantly reconstructed by looking for two objects in a decay. 

{\bf We therefore emphasize that the following decay modes are intended to cover similar/related decay modes, for example $2j+invisible$ is also a proxy for $3j$ because searches for non-resonant hadronic LLP decays can be sensitive to both. It should also be emphasized that searches should not be optimized to the exact, exclusive decay mode because that could suppress sensitivity to related but slightly more complicated models.}

\begin{itemize}
\item {\bf Diphoton decays:}~The LLP can decay resonantly to $\gamma\gamma$ (like in Higgs-portal models) or to $\gamma\gamma+\mathrm{invisible}$ (in dark matter models). This latter mode stands as a proxy for other 3-body decays where you don't explicitly reconstruct the third object.
\item {\bf Single photon decays:}~The LLP decays to $\gamma+\mathrm{invisible}$ (like in SUSY gauge mediation).  While the gauge mediation signal mandates a near massless invisible final state, a more general dark matter framework can come with a heavy final state particle.
\item {\bf Hadronic decays:}~The LLP can decay into two jets ($jj$) (like in Higgs-portal, gauge-portal models, or RPV SUSY), $jj$ + invisible (like in SUSY, dark matter, or neutrino models), or $j$ + invisible (like in SUSY). Here, jet ($j$) means either a light-quark jet, gluon, or $b$-quark jet.
\item {\bf Semileptonic decays:}~The LLP can decay into a lepton + 1 or 2 jets (like in SUSY or neutrino models).
\item {\bf Leptonic decays:}~The LLP can decay into $\ell^+\ell^-(+\mathrm{invisible})$, or $\ell^\pm+\mathrm{invisible}$ (as in Higgs-portal, gauge-portal, SUSY, or neutrino models). $\ell$ is any flavor of charged lepton, but the decays are lepton-flavor conserving.
\item {\bf Flavored leptonic decays:}~The LLP can decay into $\ell_\alpha+\mathrm{invisible}$, $\ell_\alpha^+\ell_\beta^-$ or $\ell_\alpha^+\ell_\beta^-+\mathrm{invisible}$ where flavors $\alpha\neq\beta$ (as in SUSY or neutrino models).
\end{itemize}


In all examples, $c\tau$ and the mass of the LLP is a free parameter. Therefore, stable particle searches are also covered by taking the $c\tau\rightarrow\infty$ limit of any decay mode. 

As an example of how the above decay modes cover the most important experimental signatures, we consider a scenario of an LLP decaying to top quarks. This scenario is very well motivated (for instance, with long-lived stops on SUSY) and it would appear to merit its own decay category. However, the top quark can decay to several final states that \emph{are} covered in the above list, such as a semileptonic decay ($t\rightarrow b\ell^+\nu$) and a hadronic decay ($t\rightarrow bjj$), and so the above model-independent decay modes would cover this important scenario.

While it would be ideal to have experimental searches for each of the above decay modes, it is rare for specific models to allow the LLP to decay in only one manner; instead, a number of LLP decays are allowed with a prediction for the branching fraction. For example, if a LLP couples to the SM via mixing with the SM Higgs boson, then the LLP decays via mass-proportional couplings giving rise to $b$- and $\tau$-rich signatures. If, instead, the LLP decays through a kinetic mixing as in the case of dark photons or $Z$ bosons, then the LLP can decay to any particle charged under the weak interactions, giving rise to a relatively large leptonic branching fraction in addition to hadronic decay modes. This allows some level of prioritization of decay modes based on motivated UV-complete models, although  it is ultimately desirable to retain independent sensitivity to each individual decay mode.\linebreak


\noindent {\bf Invisible Final-State Particles:}~where invisible particles appear as a product of LLP decay, there is an additional model dependence arising from the unknown nature and mass of the invisible particle. The invisible particle could be a SM neutrino, an LSP in SUSY, a dark matter particle, or another beyond-SM particle. The phenomenology depends strongly on the mass splitting, $\Delta \equiv M_{\rm LLP}-M_{\rm invisible}$. If $\Delta \ll M_{\rm LLP}$ (\emph{i.e.,} $M_{\rm LLP}\sim M_{\rm invisible}$, the spectrum is squeezed and the decay products of the LLP are soft. This could, for instance, lead to signatures such as disappearing tracks or necessitate the use of ISR jets to reconstruct the LLP signature. If the mass splitting is large, $M_{\rm invisible}\ll M_{\rm LLP}$, then the signatures lose their dependence on the invisible particle mass. 

We suggest three possible benchmarks:~a squeezed spectrum with $\Delta \ll M_{\rm LLP}$; a massless invisible state, $\Delta = M_{\rm LLP}$ (which also includes the case where the invisible particle is a SM neutrino); and an intermediate splitting corresponding to a democratic mass hierarchy, $\Delta \approx M_{\rm LLP}/2$.

\section{Our Simplified Models Proposals}

We separately consider LLPs that are:~(a)~neutral, (b)~electrically charged but color neutral, and (c)~carry color charge. These are considered separately because the relevant production modes are different (the latter two have irreducible production through SM gauge interactions) and their signatures can be different (such as disappearing tracks and hadronized LLPs).

Once again, we emphasize that in spite of the many simplified model channels, there are a small number experimental LLP searches that have excellent coverage to a wide range of channels. The goal is ultimately to identify whether there are other searches that could have a similarly high impact, and where the gaps are.

\subsection{Neutral LLPs}

The simplified model channels for neutral LLPs are shown in Table \ref{tab:neutral_LLP}. In the first iteration of the simplified models, it is sufficient to consider as ``jets'' each of the following:~$j=u,d,s,c,b,g$. However, we comment that $b$-quarks pose unique challenges and opportunities. Since $b$-quarks are themselves LLPs, they appear with an additional displacement relative to the LLP decay location. They also often give rise to soft muons in their decays, which could in principle lead to additional trigger or selection possibilities. We discuss this further in Section \ref{sec:simplified_future}. Similarly, for now we consider $e$, $\mu$, and $\tau$ to be included  in the header of ``leptons'' and searches should try to be sensitive to each.
 

When multiple production modes are specified in one row (typically one or more in parentheses), this means that multiple especially well motivated production channels give rise to similar signatures. Typically only one of these production modes will need to be included in a search, but we include the different production modes to indicate where people's favorite models may lie. $X$ indicates the LLP.

In each entry of the table, we indicate where a particular $(\mathrm{production})\times(\mathrm{decay})$ mode is predicted in the most well-motivated version of the UV theory class. If the UV model is indicated in parentheses, MET is required in the decay.
%
\begin{table}
\begin{center}
\begin{tabular}{ |c|c|c|c|c|c|c| } 
 \hline
\backslashbox{Production}{Decay} & $\gamma\gamma(+\mathrm{inv.})$ & $\gamma+\mathrm{inv.}$ & $jj(+\mathrm{inv.})$ & $jj\ell$ & $\ell^+\ell^-(+\mathrm{inv.})$ & $\ell_\alpha^+\ell_{\beta\neq\alpha}^-(+\mathrm{inv.})$\\
\hline\hline
DPP:~sneutrino pair &  & SUSY & SUSY & SUSY & SUSY & SUSY\\
 \hline
 HP:~squark pair, $\tilde{q}\rightarrow jX$ &  & SUSY & SUSY & SUSY & SUSY & SUSY\\
 or gluino pair $\tilde g\rightarrow jjX$ &&&&&&\\
 \hline
HP:~slepton pair, $\tilde{\ell}\rightarrow\ell X$ &  & SUSY & SUSY & SUSY & SUSY & SUSY\\
 or chargino pair, $\tilde{\chi}\rightarrow WX$ &&&&&&\\
 \hline 
% HIG:~$h(h')\rightarrow XX$ & Higgs (DM)  &  & Higgs (DM) &  & Higgs (DM) & \\
 HIG:~$h\rightarrow XX$ & Higgs, DM*  &  & Higgs, DM* &  & Higgs, DM* & \\
  or $\rightarrow XX+\mathrm{inv.}$ &&&&&&\\
 \hline 
 %HIG:~$h(h')\rightarrow X+\mathrm{inv.}$ & DM  &  & DM &  & DM & \\
 HIG:~$h\rightarrow X+\mathrm{inv.}$ & DM*  &  & DM* &  & DM* & \\
  \hline
   ZP:~$Z(Z')\rightarrow XX$ & $Z'$, DM*  &  & $Z'$, DM* &  & $Z'$, DM* & \\
  or $\rightarrow XX+\mathrm{inv.}$ &&&&&&\\
 \hline 
 ZP:~$Z(Z')\rightarrow X+\mathrm{inv.}$ & DM  &  & DM &  & DM & \\
  \hline
   CC:~$W(W')\rightarrow \ell X$ &   &  & RH$\nu$* & RH$\nu$ & RH$\nu$* & RH$\nu$* \\
  \hline
\end{tabular}
%
\end{center}
\caption{{\bf Simplified model channels for neutral LLPs.} The LLP is indicated by $X$. Each row shows a separate production mode and each column shows a separate possible decay mode, and therefore every cell in the table corresponds to a different simplified model channel of (production)$\times$(decay). We have cross-referenced the ``well-motivated'' UV models from Section \ref{sec:motivated_theories} with cells in the table to show how the most common signatures complete models can be linked to the simplified models. When two production modes are provided (with an ``or''), both production simplified models can be used to cover the same experimental signatures. Parentheses in the decay mode indicate the presence of additional $\slashed{E}_{\rm T}$ in some models. The asterisk (*) shows that the model definitively predicts missing momentum in the LLP decay. }\label{tab:neutral_LLP}
\end{table}
%
We emphasize that the production modes listed in Table \ref{tab:neutral_LLP} encompass also the associated production of prompt objects. For example, the Higgs production modes not only proceed through gluon fusion, but also through vector-boson fusion and $VH$ production, both of which result in associated prompt objects such as forward tagging jets, leptons, or missing momentum. All of the production modes listed in Table \ref{tab:neutral_LLP} could be accompanied by ISR jets that aid in triggering or identifying signal events. It is therefore important that searches are designed to exploit such associated prompt objects whenever they can improve signal sensitivity.

To demonstrate how to map full models onto the list of simplified models (and vice-versa), we  consider a few concrete cases. For instance, if we consider a model of neutral naturalness where $X$ is a long-lived scalar that decays via Higgs mixing (for instance, $X$ could be the lightest quasi-stable glueball), then the process where the SM Higgs $h$ decays to $h\rightarrow XX$, $X\rightarrow b\bar{b}$ would be covered with the Higgs production mechanism and a dijet decay. Entirely unrelated models, such as SUSY where $X$ is a neutralino  in $h\rightarrow XX$, $X\rightarrow j jj $ would be covered with the same simplified model. Similarly, a hidden sector model with a dark photon, $A'$, tat is produced in $h\rightarrow A'A'$, $A'\rightarrow f\bar{f}$ would also give rise to the dijet signature when $f$ is a quark, whereas it would populate the $\ell^+\ell^-$ column if $f$ is a lepton. Finally, a scenario with multiple hidden sector states $X_1$ and $X_2$, in which $X_2$ is an LLP and $X_1$ is a stable, invisible particle, could give rise to signatures like $h\rightarrow X_2 X_2$, $X_2\rightarrow X_1jj$ would be covered by the same Higgs production, hadronic decay simplified model; however, we see how $\slashed{E}_{\rm T}$ can easily appear in the final state, and that a dijet pair does not always reconstruct a resonance. Therefore, the simplified models in Table \ref{tab:neutral_LLP} can cover an incredibly broad range of signatures, but only if searches are not overly optimized to particular features such as $\slashed{E}_{\rm T}$ and resonances\footnote{This is, of course, not to say that searches shouldn't be done that exploit these features, but only that experiments should bear in mind the range of topologies and models covered by each cell in Table \ref{tab:neutral_LLP} when designing searches.}.

\subsection{Electrically Charged LLPs:~$|Q|=1$}

Here, we need to consider \emph{far fewer} production modes because of the irreducible gauge production associated with the electric charge. We still consider one heavy parent scenario where the heavy parent has a QCD charge, as this could potentially dominate the production cross section. Note we lump all resonant production into the $Z'$ simplified model.  The reason is that the SM Higgs cannot decay into two charged particles due to the model-independent limits from LEP on charged particles masses $M\gtrsim75$ GeV.   Similarly, there are fewer decay modes because of the requirement of charge conservation.  %It is true that a heavy scalar could lead to LLP pair production, but due to the irreducible gauge production cross section, we do not need to rely on associated production modes and so the scalar and $Z'$ simplified models can be combined for simplicity.

For concreteness, we recommend using $Q=1$ as a benchmark for charged LLPs for the purpose of determining allowed decay modes. 
%Because other values of $Q$ are also possible, it is worth still treating the production cross section as a free parameter. 
Although other values of $Q$ are possible, these typically result in cosmologically stable charged relics or necessitate different decay paths than those listed here.   
We note that there are dedicated searches for heavy quasi-stable charged particles with either $Q\gg1$ or $Q\ll1$; because those searches are by construction not intended to be sensitive to the decays of the LLP, the existing models are sufficient for characterizing these signatures and they do not need to be additionally included in our framework.

\begin{table}
\begin{center}
\begin{tabular}{ |c|c|c|c|c|} 
 \hline
\backslashbox{Production}{Decay} & $\ell+\mathrm{inv.}$ &  $jj(+\mathrm{inv.})$ & $jj\ell$ & $\ell\gamma$ \\
\hline\hline
DPP:~chargino pair & SUSY & SUSY & SUSY & \\
or slepton pair & & & &\\
\hline
HP:~$\tilde{q}\rightarrow j X$ & SUSY & SUSY & SUSY & \\
\hline
%ZP:~$Z'\rightarrow XX$ & Higgs/Z'/DM & Higgs/Z'/DM & Higgs/Z'/DM \\
ZP:~$Z'\rightarrow XX$ & Z', DM*& Z', DM* & Z'  & \\
\hline
CC:~$W'\rightarrow X+\mathrm{inv.}$ & DM* & DM* &  &\\
\hline
\end{tabular}
\end{center}
\caption{{\bf Simplified model channels for electrically charged LLPs, $|Q|=1$.} The LLP is indicated by $X$. Each row shows a separate production mode and each column shows a separate possible decay mode, and therefore every cell in the table corresponds to a different simplified model channel of (production)$\times$(decay). We have cross-referenced the ``well-motivated'' UV models from Section \ref{sec:motivated_theories} with cells in the table to show how the most common signatures complete models can be linked to the simplified models. When two production modes are provided (with an ``or''), both production simplified models can be used to cover the same experimental signatures. Parentheses in the decay mode indicate the presence of additional $\slashed{E}_{\rm T}$ in some models. The asterisk (*) shows that the model definitively predicts missing momentum in the LLP decay. }\label{tab:charged_LLP}
\end{table}

\subsection{LLPs with Color Charge}
\label{sec:coloredLLPs}

Because QCD is a non-Abelian group, the gauge pair production cross section of the LLP is specified by the LLP mass and its representation under $\mathrm{SU}(3)$. 

A complication of the QCD-charged LLP is that the LLP hadronizes prior to the decay. While the hadronization will not affect any hard kinematic features of its decay, it can result in interesting phenomena such as stopping in the detector, charge flipping, etc. These have been greatly explored in the context of $R$-hadrons in SUSY and we refer those interested in performing such searches to the relevant literature.

\begin{table}
\begin{center}
\begin{tabular}{ |c|c|c|c|c|} 
 \hline
\backslashbox{Production}{Decay} & $j+\mathrm{inv.}$ &  $jj(+\mathrm{inv.})$ & $j\ell$ & $j\gamma$ \\
\hline\hline
DPP:~squark pair & SUSY & SUSY & SUSY & \\
or gluino pair & & & &\\
\hline
\end{tabular}
\end{center}
\caption{{\bf Simplified model channels for LLPs with color charge.} The LLP is indicated by $X$. Each row shows a separate production mode and each column shows a separate possible decay mode, and therefore every cell in the table corresponds to a different simplified model channel of (production)$\times$(decay). We have cross-referenced the ``well-motivated'' UV models from Section \ref{sec:motivated_theories} with cells in the table to show how the most common signatures complete models can be linked to the simplified models. When two production modes are provided (with an ``or''), both production simplified models can be used to cover the same experimental signatures. Parentheses in the decay mode indicate the presence of additional $\slashed{E}_{\rm T}$ in some models. }\label{tab:color_LLP}
\end{table}

{\bf Add in paragraph about effects of hadronization}

\section{A Simplified Model Library}

The simplified models outlined in the above sections provide a common language for theorist and experimentalists to study the sensitivity of existing searches, propose new search ideas, and interpret results in terms of UV models. However, the above activities demand a simple framework for the simulation of signal events that can be used to evaluate signal efficiencies of different search strategies and map these back onto model parameters. Requiring individual users to create their own MC models for each simplified model is impractical, inefficient, and invites the introduction of errors into the comparison of results by different individuals.

In this document, we propose and provide a draft version of a \emph{simplified model library} consisting of model files and Monte Carlo (MC) generator cards that can be used to generate events for various simplified models in a straightforward fashion. Because each experiment uses slightly different MC generators and settings, this allows each collaboration (as well as theorists) to generate events for each simplified model based on the provided files. Depending on how the LLP program expands and develops over the next few years, it may become expedient to go a step further and add to the simplified model library sets of events in a standard format (such as the Les Houches format) that can be directly fed into parton shower and detector simulation programs; given the factorization of production and decay of LLPs that is valid for all but QCD-charged LLPs, this could involve two mini-libraries:~a set of production events for LLPs, a set of decays for LLPs, along with a protocol for ``stitching'' the events together.

The current version of the library can be found here:~{\bf [provide link]}

We provide model files in the popular Universal Feynrules Output (UFO) format, which is designed to interface easily with parton-level simulation programs such as \texttt{MadGraph5\_aMC@}\texttt{NLO}. The goal is to cover as many of the simplified models of Section \ref{sec:building_blocks} with as few UFO models as possible; this limits the amount of upkeep needed to maintain the library and develops familiarity with the few UFO models needed to simulate the LLP simplified models. We then give specific instructions for how to simulate each simplified LLP channel using the UFO models. {\bf NOTE:~as this document is still in draft form, the library is not yet complete. If you need a simplified model that has not yet been filled in, or would like to contribute to completion of the library, please contact the organizers listed in the simplified model repository.}

\subsection{Base Models for Library}

In order to reproduce the simplified model channels of Section \ref{sec:building_blocks}, we need a collection of models that
%
\begin{itemize}
\item Includes additional gauge bosons and scalars to allow vector and scalar portal production of LLPs;
\item Includes new gauge-charged fermions and scalars to cover direct and simple cascade production modes of LLPs;
\item Includes a right-handed-neutrino-like state with couplings to SM neutrinos and leptons;
\item Allows for the decays of the LLP particle through all of the decay modes listed in Section \ref{sec:building_blocks}, either through renormalizable or high-dimensional couplings.
\end{itemize}

Fortunately, an extensive set of UFO models is already available for simulating the production of beyond-SM particles. We note that extensions or generalizations of only three already-available UFO models are needed at the present time:

\begin{enumerate}

\item {\bf The Minimally Supersymmetric SM (MSSM)}  contains a whole host of new particles with various gauge charges and spins. Therefore, an MSSM-based model allows for the simulation of many of the simplified model channels. In particular, we note that existing UFO variants of the MSSM that include gauge-mediated supersymmetry breaking (GMSB) couplings (including decays to light gravitinos) and $R$-parity violation (to allow for the decay of otherwise stable lightest SUSY particles) already cover most of the SUSY-motivated LLP scenarios.


\item{\bf The Hidden Abelian Higgs Model (HAHM)} consists the SM supplemented by a ``hidden sector'' consisting of a new $\mathrm{U}(1)$ gauge boson and corresponding Higgs field. The physical gauge and Higgs bosons couple to the SM via kinetic and mass mixing, respectively, and can be easily supplemented with new matter fields. The HAHM allows for straightforward simulation of exotic decays of the SM Higgs, as well as $Z'$ models and many hidden sector scenarios.

\item {\bf The left-right symmetric model} supplements the SM by an additional $\mathrm{SU}(2)_{\rm R}$ symmetry, which leads to additional charged and neutral gauge bosons. The model also contains a right-handed neutrino which is the typical LLP candidate, which can be produced via SM $W$, $Z$, or the new gauge bosons. 

\end{enumerate}

As outlined in the next section, nearly all of the simplified model channels can be simulated with minor variants of the above models. Note that this may require some funny-seeming adjustment of parameters:~for example, in the MSSM model which does not contain a $Z'$, the effects of a $Z'$ on signal production can be included by re-scaling the mass of the SM $Z$ boson. Because the libraries are mostly needed for simulation of signal events and not total production cross sections, this is acceptable within the simplified models framework. Similarly, the spins of the particles are generally not important:~replacing the direct production of a fermion with the direct production of a scalar will not substantially change the signature, so as long as results are expressed in terms of sensitivity to cross sections and not couplings, the results can be easily applied to any similar production mode regardless of spin.

\subsection{Instructions for Simulating Simplified Model Channels}


\section{Future Opportunities and Challenges}\label{sec:simplified_future}
\begin{itemize}
\item Focused on the simplest and best-motivated models for now. Already many channels to consider! But this can be expanded as more work is done to fill in gaps
\item Focused on low-multiplicity signatures (so does not include dark showers, part of ongoing working group)
\item Cannot cover every single model! Some searches may need to be sensitive to detailed features of a specific model, particularly as we push to be more background-dominated search regions.
\item Tackle increased multiplicities due to $b$ vs.~light-flavor jet, different lepton flavors?
\end{itemize}
