\section{Recasting Inside the Experimental Collaborations}
\label{sec:ch5-recastingInsideExp}

Reinterpretations performed within the experiments themselves present unique advantages and disadvantages. They allow for thorough and consistent treatment of detector effects and geometry, object reconstruction, and systematic uncertainties in a way which is impossible through external recasting. Groups can share resources and easily communicate all necessary details. On the other hand, they are of course limited to the model(s) chosen for reinterpretation. In the ideal situation, reinterpretation(s) which provide meaningful results can be performed with minimal overhead to a given analysis.

% \textcolor{blue}{SK: To my mind closure tests going from object efficiencies to simplified model efficiencies, efficiency maps for additional simplified models, 
% and additional precise interpretations of interesting models would be of clear benefits; however all this substantially increases the workload 
% for the analysis group unless the process can be automatised. I see the typical use case as a pheno study finding something interesting and 
% the collaboration then doing a more precise interpretation than can be done with public tools.}

\subsection{The RECAST Framework}
{\red{GC: Lukas agreed to prepare written contribution, awaiting input}}