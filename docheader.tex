%%%%%%%%%%%%%
%Compilation options
%%%%%%%%%%%%%

\usepackage{graphicx}
\usepackage[svgnames]{xcolor}
\usepackage{hyperref}
%\usepackage{lineno}
\usepackage{epstopdf}
% \usepackage{appendix}
\usepackage{xspace} %for single top
\usepackage{longtable} %for ttbar 
\usepackage{slashed}
\usepackage{amsfonts}
\usepackage{afterpage}
\usepackage[T1]{fontenc}
%%\usepackage{fontspec}
\usepackage{amsmath,amssymb}
\usepackage{slashed}
%%\usepackage{feynmp}
% begin feynman diagram setup
\usepackage{feynmp-auto}
\usepackage[utf8]{inputenc}
% before setting default graphics include widths, save the default to properly scale feynman diagrams with their labels
\makeatletter
\let\ginnatwidth\Gin@nat@width
\let\ginnatheight\Gin@nat@height
\makeatother
\setkeys{Gin}{width=\linewidth,totalheight=\textheight,keepaspectratio}
%if you don't use the feynmandiagram environment for feynmf/mp figures, and you've changed the Gin keys above, then you need to manually adjust the unit length for feynman diagrams to get ratio of tex labels to graphics right. \unitlength = 1.21mm for tufte \linewidth and \textheight
\DeclareGraphicsRule{*}{mps}{*}{}
\newenvironment{feynmandiagram}[1][]{\setkeys{Gin}{width=\ginnatwidth,totalheight=\ginnatheight}\begin{fmffile}{#1}
\begin{fmfgraph*}(100,70)\fmfpen{thick}}{\end{fmfgraph*}\end{fmffile}\setkeys{Gin}{width=\linewidth,totalheight=\textheight,keepaspectratio}}
% end feynman diagram setup
\usepackage{subfig}
\DeclareGraphicsRule{*}{mps}{*}{}

\usepackage{xhfill}% http://ctan.org/pkg/xhfill
%\newcommand{\ditto}[1][.4pt]{\xrfill{#1}~''~\xrfill{#1}}
\newcommand{\ditto}{~''~}


\setkeys{Gin}{width=\linewidth,totalheight=\textheight,keepaspectratio}
\usepackage{mathtools} % extended mathematics
\usepackage{booktabs} % book-quality tables
\usepackage{units}    % non-stacked fractions and better unit spacing
\usepackage{multicol} % multiple column layout facilities
\usepackage{fancyvrb} % extended verbatim environments
\usepackage{fancyhdr}
\usepackage{refcount}
\usepackage{calc}
\usepackage{lastpage}

\usepackage[firstpage]{draftwatermark}
\SetWatermarkScale{0.8}

% \usepackage{natbib} % for \citep
% create a dummy file (shell command "touch moderntex") to turn on some features that don't work on lxplus
\IfFileExists{moderntex}{
  \usepackage[protrusion=true,expansion=true,tracking=true,kerning=true,spacing=true]{microtype}
}{}
\fvset{fontsize=\normalsize} % default font size for fancy-verbatim environments

\usepackage[marginpar]{todo} % for todo list
\let\nominalTodo\Todo
\renewcommand\Todo[1]{\nominalTodo{\normalfont\footnotesize\sffamily\bf #1}}


%%%Nice tables
\usepackage{booktabs,colortbl, array}
\usepackage{rotating}
%%End nice tables

\newcommand{\red}[1]{\textcolor{red}{#1}}
% Prints an asterisk that takes up no horizontal space.
% Useful in tabular environments.
\newcommand{\hangstar}{\makebox[0pt][l]{*}}
\newcommand{\openepigraph}[2]{%
  %\sffamily\fontsize{14}{16}\selectfont
  \begin{fullwidth}
  \sffamily\large
  \begin{doublespace}
  \noindent\allcaps{#1}\\% epigraph
  \noindent\allcaps{#2}% author
  \end{doublespace}
  \end{fullwidth}
}
\newcommand{\blankpage}{\newpage\hbox{}\thispagestyle{empty}\newpage}

\usepackage{etex}
\usepackage{multirow,mdwlist}

\reserveinserts{20}
%,backref=true,maxcitenames=3,maxbibnames=3,

\usepackage[hyperref=true,url=false,backend=bibtex,style=numeric,backref=false,firstinits=true,doi=false,eprint=true,language=USenglish,sorting=none,style=numeric-comp]{biblatex}
% from http://tex.stackexchange.com/questions/176965/biblatex-sentence-case-and-math-mode-not-working-together
\DeclareFieldFormat{sentencecase}{\MakeSentenceCase{#1}}
\renewbibmacro*{title}{%
  \ifthenelse{\iffieldundef{title}\AND\iffieldundef{subtitle}}
    {}
    {\ifthenelse{\ifentrytype{article}\OR\ifentrytype{inbook}\OR\ifentrytype{report}%
      \OR\ifentrytype{incollection}\OR\ifentrytype{inproceedings}%
      \OR\ifentrytype{inreference}}
      {\printtext[title]{%
        \printfield[sentencecase]{title}%
        \setunit{\subtitlepunct}%
        \printfield[sentencecase]{subtitle}}}%
      {\printtext[title]{%
        \printfield[titlecase]{title}%
        \setunit{\subtitlepunct}%
        \printfield[titlecase]{subtitle}}}%
     \newunit}%
  \printfield{titleaddon}}
% many of the following biblatex commands follow from examples by Ian Brock (ian.brock@cern.ch)
\DeclareFieldFormat[article]{journaltitle}{#1\isdot}
\DeclareFieldFormat[article]{journalsubtitle}{#1\isdot}
\DeclareFieldFormat[article]{volume}{\textbf{#1}\isdot}
\DeclareFieldFormat[article,inbook,incollection,inproceedings,patent,thesis,unpublished]
  {title}{\emph{#1\isdot}}
\errorcontextlines=100
\renewcommand*{\newunitpunct}{\addcomma\space}
\renewbibmacro{in:}{}
\renewcommand{\bibpagespunct}{\space}
\DefineBibliographyStrings{USenglish}{%
  page = {},
  pages = {}
}
\DefineBibliographyStrings{UKenglish}{%
  page = {},
  pages = {}
}