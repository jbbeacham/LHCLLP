Long-lived particle (LLP) searches represent a fantastic opportunity for discovering beyond-the-Standard Model physics at CERN's Large Hadron Collider (LHC).  The LHCb, CMS and ATLAS experiments have already conducted several searches for LLPs in proton-proton collision data in LHC Run 1, at center-of-mass energies of 7 and 8 TeV, and at the beginning of LHC Run 2, at 13 TeV.  Such searches often require significantly customized analysis strategies involving non-standard detector objects, reconstruction methods and triggers.  As a result, they have received modest coordinated attention compared to the overwhelming majority of LHC searches for particles that decay promptly and yield well-defined and well-calibrated detector objects.

With the successful establishment of a wide range of null results from the mainstream searches for prompt objects with the first data taken at 13 TeV --- typically searching for particles and phenomena that benefit most from the sizable increase in center-of-mass energy --- the challenge and goal for the remainder of the LHC research program is to design optimal searches for as wide a range of detector signatures as possible, to ensure that new physics is uncovered wherever it may be hiding.  Given the myriad possible signal scenarios that can feature LLPs (including models involving dark photons, hidden valleys, R-parity violating supersymmetry, dark QCD sectors, heavy neutral leptons, etc.) and the attendant detector signatures, as well as the desirability of presenting results in a largely model-independent fashion, the LHC LLP Community (experimentalists and theorists) here summarizes the current state of such searches to better define LHC sensitivity to LLPs, to identify optimal experimental information to be presented in LLP results to ensure future recasting and reinterpretation, and to discuss prospects for future searches in Runs 2 and 3 and at the High-Luminosity LHC.  Several workshops and conferences have been held in 2015 and 2016 \footnote{The ``Long-Lived Particle Signatures Workshop", at the University of Massachusetts, Amherst, in November of 2015; ``Searching for Exotic Hidden Signatures with ATLAS in LHC Run 2", in Cosenza, Italy, in March of 2016; ``Experimental Challenges for the LHC Run II", at the Kavli Institute for Theoretical Physics, in May of 2016; and the ``LHC Long-Lived Particles Mini-Workshop", at CERN in May of 2016} that have underscored the need to continue and evolve the study of LLP signatures at the LHC experiments to achieve these goals and this document is a compendium of notes and material from these events, as well as new, complementary studies.

This document is organized as follows: \hyperref[sec:motivation]{Section 2} presents the general theoretical motivations for BSM LLP searches at the LHC and discusses the experimental challenges related to such signatures.  \hyperref[sec:signatures]{Section 3} enumerates the detector signatures the ATLAS, CMS, and LHCb experiments are capable of identifying and discusses in more detail the theoretical motivations for each, as well as a discussion of the existing searches, known gaps in coverage and proposals to improve coverage.  \hyperref[sec:recommendations]{Section 4} contains a set of recommendations for the presentation of LLP search results to ensure that such searches are optimally reinterpretable.  \hyperref[sec:future]{Section 5} presents a discussion of the prospects for LLP searches in Runs 2 and 3 of the LHC and for the High-Luminosity LHC run planned for the future.  Finally, \hyperref[sec:conclusions]{Section 6} concludes.



