Particles in the Standard Model (SM) have lifetimes spanning an enormous range of magnitudes, from the $Z$ boson ($\tau\sim2\times10^{-25}$ s) through to the proton ($\tau\gtrsim10^{34}$ years) and electron (stable). Similarly, models beyond the SM typically predict new particles with a variety of lifetimes \cite{massive-cite-dump}. In particular, new weak-scale particles can easily have lifetimes $\gtrsim10^{-12}$ m for several reasons, including approximate symmetries that stabilize the long-lived particle (LLP), small couplings between the LLP and lighter states, and suppressed phase space available for decays. For particles moving close to the speed of light, this can lead to macroscopic, detectable displacements between the production and decay points of an unstable particle for $c\tau\gtrsim 10\,\,\mu\mathrm{m}$.

The experimental signatures of LLPs are varied, and often different from signals of SM processes:~tracks with unusual ionization and propagation properties; small, localized deposits of energy inside of the calorimeters without associated tracks; stopped particles that decay out of time with collisions; displaced vertices in the inner detector or muon spectrometer; and disappearing, appearing, and kinked tracks, to name a few. Because the long-lived particles of the SM have masses $\lesssim5$ GeV and have well-understood experimental signatures, the unusual signatures of beyond-SM LLPs offer excellent prospects for the discovery of new physics at particle colliders. At the same time, standard reconstruction algorithms may reject events or objects containing LLPs precisely because of their unusual nature, and dedicated searches are needed to uncover LLP signals. These atypical signatures can also resemble noise, pile-up, or mis-reconstructed objects in the detector; due to the rarity of such mis-reconstructions, Monte Carlo (MC) simulations may not accurately model backgrounds for LLP searches, and dedicated methods are needed to assess backgrounds for LLP searches.

There exist many searches for LLPs at the ATLAS, CMS, and LHCb experiments at the Large Hadron Collider (LHC) \cite{another-massive-cite-dump}, and novel methods have been developed for identifying signals of LLPs, and identifying and suppressing the relevant backgrounds. Indeed, in several scenarios searches for LLPs have sensitivities that greatly exceed the search for similar, promptly decaying new particles. The excellent sensitivity of these searches, together with the lack of a definitive signal in any prompt channels at the LHC, have focused attention on other types of LLP signatures that are not currently covered. These include low-mass LLPs that do not pass trigger or selection thresholds of current searches, high multiplicities of LLPs produced in dark-sector showers, or unusual LLP production and decay modes that are not covered by current methods. Given the excellent  sensitivity of LHC detectors to LLPs, along with the potentially large production cross sections of LLPs and the enormous amount of data to be collected in the LHC's high-luminosity running, it is imperative that the space of LLP signatures be explored as thoroughly as possible to ensure that no signals are missed. This is particularly important as decisions are currently being made about detector upgrades for Phase 2 of the LHC, and design decisions should be made to ensure that sensitivity to LLPs is retained through high-luminosity running; indeed, upgrades to the detectors may improve the sensitivity to LLPs over current conditions!

The growing theoretical and experimental interest in LLPs has been mirrored by an increased activity in proposals for LLP searches, new experimental analyses, and meetings to communicate results and discuss new ideas. Workshops focused on LLPs at the University of Massachussetts, Amherst; Fermilab; CERN; and KITP (UCSB), among others, highlighted the need for a community-wide effort to map the current space of theoretical models for LLPs, assess the coverage of current experimental methods to these models, and identify areas where new searches are needed. Additionally, the work presented in these meetings accentuated the need for presenting the results of experimental searches in a manner that allows for their application to different models:~this makes current searches more powerful by increasing their applicability to new scenarios, while reducing redundancies in searches and ensuring that gaps in coverage are identified and addressed. This task extends beyond the purview of any particular theoretical model or experiment, and requires an effort across collaborations to address the needs of the LLP community and illuminate a path forwards. 

This is the work undertaken by the LHC-LLP Community and presented in this document. Based on the most pressing needs identified by the community, we established a set of goals to guide the work of the LHC-LLP Community:
%
\begin{itemize}

\item {\bf Simplified models:}~Identify a minimal (but expandable) set of simplified models that capture the most important LLP signatures motivated by theory and accessible at the LHC. The simplified models approach has been successfully applied to models such as supersymmetry (SUSY) and dark matter, and proposals exist for LLP simplified models in particular contexts. We aim to provide a basis of models that serves as a focal point for the other studies performed by the Community, as well as a library that can be used in simulating LLP signal events, to allow for a common grammar to better understand how current and future searches cover LLP signature space.

\item {\bf Experimental coverage:}~In spite of the many successful LLP searches undertaken by the ATLAS, CMS, and LHCb experiments, there remains a need for a systematic study of the complementary coverage of LLP searches to the parameter spaces of LLP models. Having developed a simplified model basis, we provide a comprehensive overview of the sensitivity of existing searches, highlighting gaps in coverage and high-priority searches to be undertaken in the future.

\item {\bf Upgrades and trigger:}~We discuss the prospects for LLP searches with upgraded detectors for Phase 2 of LHC running, with a focus on how upgrades can offer new sensitivity to LLPs as well as mitigate the effects of pile-up. This is tied to the crucial question of triggers for LLPs; we discuss the performance of current triggers for LLPs (?), as well as the effects of future upgrades to the trigger system. Most importantly, we identify a few concrete upgrade studies that should be performed by the experiments that are of prime importance to the community.

\item {\bf Reinterpretation of LLP searches:}~Due to the non-standard nature of the objects used in analyses, LLP searches are notoriously hard to reinterpret for models beyond those considered by the experimental collaborations. Designing searches and presenting search results in a way that is broadly applicable to current and yet-to-be-developed LLP models is crucial to the impact and legacy of the LLP search program. We summarize the challenges of presenting the results of LLP searches and provide proposals based on the experience of experimentalists and theorists who are actively working on LLP studies.

\item {\bf Dark showers and other new frontiers:}~Current LLP search strategies have limited sensitivity to models where the LLPs are very soft, highly collimated, and come in large multiplicities, as can occur in models of dark-sector showers. We report on recent progress in theoretically parameterizing the space of dark-shower models and signatures, as well as experimental searches to uncover these signals. We also discuss other frontiers in LLP searches where more research and development is needed to ensure comprehensive coverage.

\end{itemize}
%
In addition to these main goals, we provide a summary and analysis of backgrounds for LLPs, which provide insight into the opportunities and challenges of searching for LLP signatures. Finally, we provide information about current and proposed experiments to search for LLPs at the LHC via dedicated detectors:~these include the MoEDAL monopole search, the milliQan milli-charged particle experiment, the MATHUSLA surface detector for ultra-LLPs, the CODEX-b proposal for a new detector near LHCb, and the FASER proposal for a long, narrow detector located in the forward direction well downstream one of the collision points.

%\section{Goals of the White Paper}

%{\bf redundant, probably don't need anymore}

%In recent years, there has been a proliferation of work on LLP both on the theory and experiment side. This is due to the success of experimental techniques to reconstruct LLPs in earlier runs of the LHC and the theoretical development of well-motivated models that give rise to a range of LLP signatures. It is also in part due to the lack of positive signals of new particles in other, more conventional LHC searches, giving heightened urgency and importance in the coverage of LLP signatures.

%To this end, the document serves to assess the current status of LLP searches at the LHC, determining where searches currently have good coverage of LLP searches and where there can be improvements. It also evaluates the impact of possible new triggers or upgrades on the performance of LLP searches in advance of the final decisions for Phase 2 upgrades. Finally, it provides recommendations for the presentation of search results to broaden the impact of existing and planned searches, and to chart a course for the near-future development of the program.

%More concretely,
%
%\begin{enumerate}
%\item We develop a minimal (but expandable) simplified models framework for the LLP program that incorporates the best-motivated LLP theories and models
%\item We use the simplified models framework to assess the sensitivity of current searches and identify the highest priority gaps in coverage
%\item These in turn inform the development of triggers and decisions about upgrade technologies, and we summarize the impact of possible new upgrades (including existing and planned dedicated detectors such as Moedal, MilliQan, MATHUSLA, \dots)
%\item Provide a summary of recommendations for the presentation of new LLP searches
%\item Identify the new frontiers of LLP searches in which more research and development is needed, particularly in the area of high-multiplicity LLP searches (dark showers)
%\end{enumerate}
