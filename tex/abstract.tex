Abstract: Searches for long-lived particles (LLPs) beyond the Standard Model at the Large Hadron Collider --- particles that can have non-negligible lifetimes and decay to SM particles within detectors but substantially displaced from the interaction vertex --- constitute a rich, challenging, and increasingly fascinating avenue via which new physics may be discovered at the LHC.  Members of the ATLAS, CMS, and LHCb experiments in conjunction with theorists, phenomenologists, and those working on dedicated experiments such as Moedal, MilliQan, MATHUSLA, CODEX-b, and FASER, here report upon the state of LLP searches at the LHC; propose a set of simplified models for LLP searches; survey the existing searches, the experimental coverage of LLP signatures, and enumerate gaps in this coverage; identify high-priority studies to be performed by the experimental collaborations to ensure that LLP signatures are not missed in detector upgrades planned for the upcoming high-luminosity era at the LHC; propose recommendations for new triggering strategies for LLPs in ATLAS, CMS, and LHCb; list ideas for new searches for LLPs; propose a set of recommendations for the presentation of search results to ensure future reinterpretation and recasting for LLP searches; discuss new frontiers for LLP searches such as those involving dark sector QCD-like theoretical ideas; and describe the often unexpected experimental challenges inherent in LLP searches, including atypical or non-standard background sources.
