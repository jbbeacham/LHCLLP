%Searches for long-lived particles (LLPs) beyond the Standard Model (SM) at the Large Hadron Collider --- particles that can have non-negligible lifetimes and decay to SM particles within detectors but substantially displaced from the interaction vertex of the primary proton-proton collision --- constitute a rich, challenging, and increasingly fascinating avenue via which new physics may be discovered at the LHC.
%Members of the ATLAS, CMS, and LHCb experiments in conjunction with theorists, phenomenologists, and those interested in or working on dedicated experiments such as MoEDAL, MilliQan, MATHUSLA, CODEX-b, and FASER, here report upon the state of LLP searches at the LHC.
%We propose a set of simplified models for LLP searches; survey the existing searches, the experimental coverage of LLP signatures, and enumerate gaps in this coverage; identify high-priority studies to be performed by the experimental collaborations to ensure that LLP signatures are not missed in detector upgrades planned for the upcoming high-luminosity era at the LHC; propose recommendations for new triggering strategies for LLPs in ATLAS, CMS, and LHCb; list ideas for new searches for LLPs; propose a set of recommendations for the presentation of search results to ensure future reinterpretation and recasting for LLP searches; discuss new frontiers for LLP searches such as those involving dark sector QCD-like theoretical ideas; and describe the often unexpected experimental challenges inherent in LLP searches, including atypical or non-standard background sources.
%
Particles beyond the Standard Model (SM) can generically have lifetimes that are long compared to SM particles at the weak scale.
When produced at experiments such as the Large Hadron Collider (LHC), these long-lived particles (LLPs) can decay far from the interaction vertex of the primary proton-proton collision but still within the acceptance of detectors.
Such LLP signatures are distinctly different from those associated with searches for promptly decaying BSM particles that constitute the majority of searches for new physics at the LHC, often requiring customized analysis techniques to identify, e.g., significantly displaced decay vertices, tracks with atypically large impact parameters, and short track segments.
Given their non-standard nature, a comprehensive overview of LLP signatures at the LHC is beneficial to ensure that possible avenues of the discovery of BSM physics are not overlooked.
Here we report on the joint work of a community of theorists and experimentalists with the ATLAS, CMS, and LHCb experiments --- as well as those working on dedicated experiments such as MoEDAL, MilliQan, MATHUSLA, CODEX-b, and FASER --- to survey the current state of LLP searches at the LHC, and to chart a path for the development of LLP searches into the future.
The work is organized around the current and future potential capabilities of LHC experiments to generally discover BSM LLPs, and treats broad classes of theoretical models that can give rise to LLPs rather than emphasizing any particular model or theory motivation.
To that end, we develop a set of simplified models and tools that can be used to parametrize the space of LLP signatures, and we use these tools to assess the coverage of current LLP searches and identify avenues for improving and extending the existing LLP search program.
We also look to the future, exploring how current searches can apply to new models and providing recommendations for the presentation of search results, while also exploring the capabilities of proposed detector upgrades for LLP searches, highlighting areas where new technologies can have a large impact on sensitivity to LLP signatures.
Finally, we look towards the newest frontiers of LLP searches, namely high-multiplicity or ``dark shower'' LLP signals, elaborating on the theoretical and experimental challenges and opportunities in expanding the LHC reach of these signals.
