Particles beyond the Standard Model (SM) can generically have lifetimes that are long compared to SM particles at the weak scale.
When produced at experiments such as the Large Hadron Collider (LHC) at CERN, these long-lived particles (LLPs) can decay far from the interaction vertex of the primary proton-proton collision.
Such LLP signatures are distinct from those associated with searches for promptly decaying particles, often requiring customized techniques to identify, for example, significantly displaced decay vertices, tracks with atypical properties, and short track segments.
Given their non-standard nature, a comprehensive overview of LLP signatures at the LHC is beneficial to ensure that possible avenues of the discovery of new physics are not overlooked.
Here we report on the joint work of a community of theorists and experimentalists with the ATLAS, CMS, and LHCb experiments --- as well as those working on dedicated experiments such as MoEDAL, MilliQan, MATHUSLA, CODEX-b, and FASER --- to survey the current state of LLP searches at the LHC, and to chart a path for the development of LLP searches into the future, both in the upcoming Run 3 and at the High-Luminosity LHC.
The work is organized around the current and future potential capabilities of LHC experiments to generally discover new LLPs, and takes a signature-based approach to surveying classes of models that give rise to LLPs rather than emphasizing any particular theory motivation.
We develop a set of simplified models; assess the coverage of current searches; document known, often unexpected backgrounds; explore the capabilities of proposed detector upgrades; provide recommendations for the presentation of search results; and look towards the newest frontiers, namely high-multiplicity ``dark showers’', highlighting opportunities for expanding the LHC reach for these signals.
