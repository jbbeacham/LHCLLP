The research program carried out over the first nine years of the Large Hadron Collider (LHC) at CERN has been an unqualified success.
The discovery, in 2012, at a center-of-mass energy of 7 and 8 TeV, of a new particle thus far consistent with the Standard Model (SM) Higgs boson has opened numerous new research directions and has begun to shed light upon the source of electroweak symmetry breaking, vector boson scattering amplitudes, and the origin of particle masses.
And the establishment of a wide range of searches for new physics at 7, 8, and 13 TeV with the ATLAS, CMS, and LHCb detectors — searches thus far consistent with SM expectations — has inspired new ideas and thinking about the most prominent open issues of physics, such as the nature of dark matter, the hierarchy problem, neutrino masses, and the possible existence of supersymmetry.

The overwhelming majority of searches for new physics have been performed under the assumption that the new particles decay promptly, i.e., very close to the proton-proton interaction point (IP), leading to well-defined objects such as jets, leptons, photons, and missing transverse momentum.
Such objects are constructed requiring information from all parts of the detector including hits close to the IP, calorimeter deposits known to be signatures of particles originating from the IP, and muons with tracks that traverse the entirety of the detector, moving out from the IP.
However, given the large range of particle lifetimes in the SM —-- resulting from general concepts such as approximately preserved symmetries, scale hierarchies, or phase space restrictions —-- and the lack of clear, objective motivation related to any particular model or theory beyond the SM, the lifetime of hypothetical new particles is best treated as a free parameter.
This leads to a wide variety of spectacular signatures in the LHC detectors that would evade prompt searches, and which have received modest attention compared to searches for promptly decaying new particles.
Because such signatures require significantly customized analysis techniques and are usually performed by a smaller number of physicists working on the experimental collaborations, a comprehensive overview and critical review of beyond-SM (BSM) LLPs at the LHC has been performed by a community of experimentalists, theorists, and phenomenologists. This effort ensures that such avenues of the possible discovery of new physics at the LHC are not overlooked.
The results of this initiative have been presented in the current document.

We developed a set of simplified models and tools, in Chapter~\ref{sec:simplifiedmodel}, that can be used to parametrize the space of LLP signatures. The simplified models were organized around generic ways that various BSM LLPs can be produced and decay to displaced or non-standard objects in the LHC detectors, rather than emphasizing any one particular theory or physics motivation.
These can serve as a useful grammar by which to compare coverage of LLP signature space and model classes among current and future experiments.

To that end, in Chapter~\ref{sec:experimentcoverage}, we utilized these models and tools to assess the coverage of current LLP searches and we identified multiple avenues for improving and extending the existing LLP search program.
Opportunities for new and improved triggering strategies, searches, and open questions for the experimental collaborations to explore centrally were presented as a list at the end of the chapter.

Moreover, due to the non-standard nature of LLP searches, many of them are performed under very low-background conditions.
As a result, sources of backgrounds largely irrelevant to searches for promptly decaying BSM particles are important for LLP analyses and can be surprising and unexpected.
In Chapter~\ref{sec:backgrounds} we discussed several sources of backgrounds for LLP searches, collecting the knowledge gained, often by trial-and-error, by experimentalists over many years of searches.

Also with an eye to the future, in Chapter~\ref{sec:triggers} we explored the potential for the expanded capabilities of proposed detector upgrades at ATLAS, CMS, LHCb, and related dedicated detectors, highlighting areas where new technologies can have a large impact on sensitivity to LLP signatures and suggesting several studies to be performed by the collaborations to ensure new physics potential is not missed for the upcoming era of the High-Luminosity LHC.

Additionally, to ensure that current and future searches can be maximally useful in the future, in Chapter~\ref{sec:reint} we explored how current searches can apply to new models and performed a comprehensive overview of some of the challenges and pitfalls inherent in attempting to recast existing LLP analyses, leading to recommendations for the presentation of search results in the future.

Finally, in Chapter~\ref{sec:showers}, we looked toward the newest frontiers of LLP searches, namely high-multiplicity or ``dark shower'' LLP signals that can, for example, be signatures of complex hidden sectors with strong dynamics and internal hadronization.
In this chapter, we elaborated on the theoretical and experimental challenges and opportunities in expanding the LHC reach of these signals.
Such dark shower signatures have a high potential for being overlooked with existing triggering strategies and analysis techniques.
Moreover, the dark-QCD-like theoretical models from which they can arise are currently being explored in depth and the resulting LHC phenomenology is in the process of being understood.
We discussed the current state of this work and we anticipate exciting independent developments in the near future.

This document is incomplete by design, since it is a record of the critical thinking and examination of the state of LLP signatures by a large number of independently organized members of the LHC LLP Community as it has evolved from 2016 to 2019, and a major component of the work has been the identification of several open questions and opportunities for discovery in such signatures.
As these questions are addressed and new searches emerge from the experimental collaborations, so, too, will new ideas emerge and evolve from the community.
We expect this document to be followed by future papers to record, review, and summarize the evolution of LLP signatures and searches, always with the intention of more effectively facilitating the discovery of new particles at the LHC and beyond. \\

\vspace{\baselineskip}

\noindent {\bf Acknowledgments:}\\

We are very grateful to Andreas Albert, Tao Huang, Laura Jeanty, Joachim Kopp, Matt LeBlanc, Larry Lee, Haolin Li, Simone Pagan Griso, Michele Papucci, Matt Strassler, and Tien-Tien Yu for helpful conversations and comments on the draft.\\

\paragraph{Editors:}~J. Beacham acknowledges support from the U.S.~Department of Energy (DOE) and the U.S.~National Science Foundation (NSF). G.~Cottin acknowledges support from the Ministry of Science and Technology of Taiwan (MOST) under Grant No.~MOST-107-2811-M-002-3120 and CONICYT-Chile FONDECYT Grant No. 3190051. N.~Desai was supported in part by the OCEVU Labex (ANR-11-LABX-0060) and the A*MIDEX project (ANR-11-IDEX-0001-02) funded by the ``Investissements d'Avenir'' French government program managed by the ANR. J.~A.~Evans is supported by DOE grant DE-SC0011784. S.~Knapen is supported by DOE grant DE-SC0009988. A.~Lessa is supported by the Sao Paulo Research Foundation (FAPESP), project 2015/20570-1. Z.~Liu is supported in part by the NSF under Grant No. PHY-1620074, and by the Maryland Center for Fundamental Physics. S.~Mehlhase is supported by the BMBF, Germany. M.~J.~Ramsey-Musolf is supported by DOE grant DE-SC0011095. The work of P.~Schwaller has been supported by the Cluster of Excellence ``Precision Physics, Fundamental Interactions, and Structure of Matter'' (PRISMA+ EXC 2118/1) funded by the German Research Foundation (DFG) within the German Excellence Strategy (Project ID 39083149). The work of J.~Shelton is supported in part by DOE under grant DE-SC0017840. The work of B.~Shuve is supported by NSF under grant PHY-1820770. X.C.~Vidal is supported by MINECO through the Ramón y Cajal program RYC-2016-20073 and by XuntaGal under the ED431F 2018/01 project. \\

\paragraph{Contributors:}~M.~Adersberger is supported by the BMBF, Germany. The work of C. Alpigiani, A. Kvam, E Torro-Pastor, M. Profit, and G. Watts is supported in part by the NSF. Y.~Cui is supported in part by DOE Grant DE-SC0008541. J.~L.~Feng is supported in part by Simons Investigator Award \#376204 and by NSF Grant No. PHY-1620638. I.~Galon is supported by DOE Grant DE-SC0010008. K.~Hahn is supported by DOE Grant DE-SC0015973. J.~Heisig acknowledges support from the F.R.S.-FNRS, of which he is a postdoctoral researcher. The work of F.~Kling was supported by NSF under Grant No. PHY-1620638. H.~Lubatti thanks the NSF for support. P.~Mermod was supported by Grant PP00P2\_150583 of the Swiss National Science Foundation. S.~Mishra-Sharma is partially supported by the NSF CAREER Grant PHY-1554858 and NSF Grant PHY-1620727. V.~Mitsou acknowledges support by the Generalitat Valenciana (GV) through MoEDAL-supporting agreements and the GV Excellence Project PROMETEO-II/2017/033, by the Spanish MINECO under the project FPA2015-65652-C4-1-R, by the Severo Ochoa Excellence Centre Project SEV-2014-0398 and by a 2017 Leonardo Grant for Researchers and Cultural Creators, BBVA Foundation. J.~Prisciandaro is supported by funding from FNRS. M.~Reece is supported by DOE Grant DE-SC0013607. D. Robinson is supported in part by NSF grant PHY-1720252. D.~Stolarski is supported in part by the Natural Sciences and Engineering Research Council of Canada (NSERC). S.~Trojanowski is supported by Lancaster-Manchester-Sheffield Consortium for Fundamental Physics under STFC grant ST/L000520/1. S.~Xie is supported by the California Institute of Technology High Energy Physics Contract DE-SC0011925 with the DOE. This manuscript has been partially authored by Fermi Research Alliance, LLC under Contract No. DE-AC02-07CH11359 with the U.S. Department of Energy, Office of Science, Office of High Energy Physics.\\



