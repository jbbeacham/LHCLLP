There exist in the literature many, many examples of new particles beyond the SM with long lifetimes. In some cases, experimental searches have targeted these new particles; in others, proposals have been made for new techniques to reconstruct LLPs in particular models. From past experience with SUSY and dark matter searches, it is evident that new dedicated searches are not needed for each and every possible new model:~instead, it is desirable to develop a strategy for performing searches such that experimental search results are as broadly applicable as possible to different types of models of new particles, reduces redundancy among experimental searches, and makes plain the gaps in coverage and areas where new searches are needed. 

The most commonly adopted framework for accomplishing these goals is the simplified models framework. The simplified models framework uses the fact that many searches are only sensitive to a few broad aspects of a particle's signature (such as the production mode, production rate, and decay topology) and not precise details such as the spin of the particle, its angular distribution, etc. With LLP signatures, the simplified models approach is even \emph{more} appropriate:~this is because searches tend to have low backgrounds and relatively inclusive searches can be done, each of which has sensitivity to LLPs with many different types of production and decay modes.

Because LLP particles are often produced and decay (if at all) in physically distinct locations, this allows for a factorization of the production and decay processes\footnote{The only scenario where this is not true is for LLPs with color charge, as these undergo hadronization near the production point. Such theoretical issues have already been discussed in the context of, for example, SUSY $R$-hadrons, and this subtlety is discussed more in the relevant subsection}. We can exploit this to construct a simple basis of LLP production modes and LLP decay modes; we define an {\bf LLP channel} as a combination of a particular production and decay mode, with the lifetime of the LLP taken as a free parameter. We emphasize that the LLP channel as defined here is \emph{not} the same as an experimental signature that manifests in the detector:~a single simplified model channel could give rise to many, many different signatures depending on where the decays in occur in the detector (or outside), while a single experimental search for a particular signature could cover many simplified model channels for particular choices of parameters. We focus here on the theory definition, which provides some control over the range of models that we expect to see at the LHC, while in Section \ref{sec:experimentcoverage} we map our basis of Simplified Models onto existing searches to determine the gaps in coverage and proposals for new searches.

\section{Goals of the Simplified Model Framework}
The purpose of the simplified model framework is to provide a simple, common language that experimentalists and theorists can use to describe LLP theories and the corresponding mapping between models and experimental signatures. We therefore want our simplified model space to be:
%
\begin{enumerate}
\item A minimal set of models that cover the most interesting theories of LLPs without undue redundancy;
\item Easy to map between models and signatures so that we can tell where the current coverage and gaps are;
\item Expandable so that it can be applied to theories and signatures that we are not yet clever enough to have thought of;
\item Provide a concrete method for the Monte Carlo simulation of signal events within a simplified model, and can be used for expressing experimental efficiencies for ready re-interpretation of experimental results to a variety of scenarios. Note that more simplified models may be needed for  re-interpretation than are strictly necessary for discovering a new particle (\emph{i.e.,} we should be mindful both of whether two simplified models share a common signature in a search, and also whether they  look similar enough to have similar reconstruction efficiencies)
\end{enumerate}

\section{Existing Well-Motivated Theories for LLPs}
Here, we categorize a few broad classes of models that have conventionally provided the best motivation for LLPs. We should emphasize that there are many theories beyond these that motivate LLPs; however, many such theories broadly fall under the umbrella of one of the following UV theories and so we can cite them but not give a detailed description.
%
\begin{itemize}
\item {\bf SUSY-like theories:}~these are models with multiple new particles carrying SM gauge charges and a variety of allowed cascade decays. Lifetimes can be long due to approximate symmetries (such as $R$-parity or gauge mediation) and decays mediated by highly off-shell intermediaries (as in split SUSY)
\item {\bf Higgs-portal theories:}~these include scenarios with exotic Higgs decays to low-mass particles  (as in many Hidden Valley scenarios), and is well motivated by the fact that the Higgs can still have a 30\% branching fraction to exotic final states. The Higgs is also special in that it comes with its own set of associated production modes, such as VBF or Higgs-strahlung.
\item {\bf Gauge-portal theories:}~these include scenarios with new vector mediators decaying to exotic LLPs. These are similar to Higgs models, but where the mediator is predominantly produced from $q\bar{q}$-initiated final states without other associated objects.
\item {\bf Dark-matter theories:}~this class focuses on non-SUSY dark matter and hidden-sector scenarios, and encompasses models where dark matter is produced as a final state in the collider process. The main distinguishing feature from the Higgs and gauge scenarios above is that dark matter (missing momentum) is a necessary and irreducible component of each signature.
\item {\bf Heavy neutrinos and friends:}~if there exist new weak-scale states responsible for giving SM neutrinos mass, the new particles can typically be long lived. The signatures tend to be lepton-rich due to the connection with SM neutrino masses.
\end{itemize}
%
As we develop our simplified models framework, we will construct maps between the UV models and the simplified model channels to show the highest priority/best motivated combinations of production and decay modes for LLPs. This will then allow us to focus on the most interesting scenarios and determine their coverage in particular parts of the model parameter space.

\section{The Simplified Model Building Blocks}

Recall that in LLP searches production and decay can be factorized. This allows us to specify the relevant production and decay modes for LLP models separately; we then put them together and map the space of models into the existing motivated theories.

\subsection{Production Modes}
According to the earlier models, we can identify a minimal set of interesting production modes for LLPs:
%
\begin{itemize}
\item {\bf Direct gauge production (GP):}~If the LLP is charged under a SM gauge interaction, it can be directly produced via the corresponding gauge boson. The production cross section is specified by the LLP gauge charge and mass.
\item {\bf Heavy parent (HP):}~If the LLP can be produced in the decay of a heavy parent particle that is itself charged under the SM gauge interactions. The production cross section is essentially a free parameter and is indirectly specified by the charge and mass of the heavy parent. The heavy parent production gives a very different kinematics for the LLP than gauge production.
\item {\bf Higgs (HIG):}~The LLP can be produced in decays of the SM Higgs boson, and therefore has all of the same associated production modes (VBH, VH). Note that if the LLP mass is heavier than $m_h/2$, it can also be produced via the off-shell Higgs portal. The LLP can be pair produced or singly produced (in association with MET). The cross section (or Higgs branching fraction) is a free parameter of the model.
\item {\bf $Z'$ (ZP):}~Similar to the Higgs portal, but the LLP is produced in the decay of an on-shell gauge boson initiated by $q\bar{q}$ initial state. Note that production via off-shell $Z'$ is similar to if the LLP is directly charged under the SM $Z$ and so is included in GP above. As with HIG, the LLP can be pair produced or singly produced (in association with MET).
\item {\bf Charged current (CC):}~In the case of a right-handed neutrino, the LLP can be produced in the leptonic decays of $W/W'$. Single production is favored.
\end{itemize}

\subsection{Decay Modes}
It is important to note that LLP searches are typically fairly inclusive. This is in part due to the fact that particle ID is less possible for decays far in the detector (\emph{e.g.,} for decays inside of the calorimeter, everything looks like a calorimeter deposition). It is also because backgrounds are low enough that tight cuts typically found in exclusive analyses are not needed to suppress backgrounds. For example, ATLAS has a displaced vertex search sensitive to dilepton and multitrack vertices that are relatively agnostic to other objects originating from near the displaced vertex. Similarly, CMS has an analysis sensitive to high-impact-parameter leptons without reconstructing a vertex. Indeed, the most important factor is whether the LLP decays into two SM objects or three, because this determines whether its mass is resonantly reconstructed by looking for two objects a displaced vertex. 

{\bf We therefore emphasize that the following decay modes are intended to cover similar/related decay modes, for example $2j+invisible$ is also a proxy for $3j$ because searches for non-resonant hadronic LLP decays are sensitive to both. It should also be emphasized that searches should not be optimized to the exact, exclusive decay mode because that could suppress sensitivity to related but slightly more complicated models.}

\begin{itemize}
\item {\bf Diphoton decays:}~The LLP can decay resonantly to $\gamma\gamma$ (like in Higgs-portal models) or to $\gamma\gamma+\mathrm{invisible}$ (in dark matter models). This latter mode stands as a proxy for other 3-body decays where you don't explicitly reconstruct the third object.
\item {\bf Single photon decays:}~The LLP decays to $\gamma+\mathrm{invisible}$ (like in SUSY gauge mediation).
\item {\bf Hadronic decays:}~The LLP can decay into two jets (like in Higgs-portal, gauge-portal models, or RPV SUSY) or to two jets + invisible (like in SUSY, dark matter, or neutrino models). Here, a jet means either a light-quark jet, gluon, or $b$.
\item {\bf Semileptonic decays:}~The LLP can decay into a lepton + 2 jets (like in SUSY or neutrino models).
\item {\bf Leptonic decays:}~The LLP can decay into $\ell^+\ell^-$ or into $\ell^+\ell^-+\mathrm{invisible}$ (as in Higgs-portal, gauge-portal, SUSY, or neutrino models). $\ell$ is any flavor of charged lepton, but the decays are lepton-flavor conserving.
\item {\bf Flavored leptonic decays:}~The LLP can decay into $\ell_\alpha^+\ell_\beta^-$ or $\ell_\alpha^+\ell_\beta^-+\mathrm{invisible}$ where flavors $\alpha\neq\beta$ (as in SUSY or neutrino models).
\end{itemize}

In all examples, $c\tau$ is a free parameter. Therefore, stable particle searches are also covered by taking the $c\tau\rightarrow\infty$ limit of any decay mode. The mass of the LLP is also a free parameter, and in the case of an invisible particle in the LLP decay, there are three well-motivated choices of the quantity $\Delta\equiv M_{\rm LLP}-M_{\rm inv}$:~$\Delta=M_{\rm LLP}$ (\emph{i.e.,} massless invisible state like light DM or a neutrino); $\Delta=M_{\rm LLP}/2$, corresponding to a democratic mass hierarchy; $\Delta=\epsilon \ll M_{\rm LLP}$, representing a squeezed spectrum.

\section{A Proposed Basis for Simplified Models}

We separately consider LLPs that are:~(a)~neutral, (b)~electrically charged but color neutral, and (c)~carry color charge. These are considered separately because the relevant production modes are different (the latter two have irreducible production through SM gauge interactions) and their signatures can be different (such as disappearing tracks and hadronized LLPs).

Once again, we emphasize that in spite of the many simplified model channels, there are a small number experimental LLP searches that have excellent coverage to a wide range of channels. The goal is ultimately to identify whether there are other searches that could have a similarly high impact, and where the gaps are. As with any choice of organizing principle, there are also going to be models and signatures that do not fit neatly into the prescription below. We discuss exceptions and caveats in Section \ref{sec:exceptions}, and we encourage the community to see this not as a \emph{rigid} prescription, but as a starting point that can be built upon, adapted, and expanded as the community grows and evolves.

\subsection{Neutral LLPs}

Reminder:~$j=u,d,s,c,b,g$, $\ell=e,\mu,\tau$.
When multiple production modes are specified in one row (typically one or more in parentheses), this means that multiple especially well motivated production channels give rise to similar signatures. Typically only one of these production modes will need to be included in a search, but we include the different production modes to indicate where people's favorite models may lie. $X$ indicates the LLP.

In each entry of the table, we indicate where a particular $(\mathrm{production})\times(\mathrm{decay})$ mode is predicted in the most well-motivated version of the UV theory class. If the UV model is indicated in parentheses, MET is required in the decay.

\begin{center}
\begin{tabular}{ |c|c|c|c|c|c|c| } 
 \hline
Channel & $\gamma\gamma(+\mathrm{inv.})$ & $\gamma+\mathrm{inv.}$ & $jj(+\mathrm{inv.})$ & $jj\ell$ & $\ell^+\ell^-(+\mathrm{inv.})$ & $\ell_\alpha^+\ell_{\beta\neq\alpha}^-(+\mathrm{inv.})$\\
\hline\hline
GP:~sneutrino pair &  & SUSY-like & SUSY-like & SUSY-like & SUSY-like & SUSY-like\\
 \hline
 HP:~$\tilde{q}\rightarrow jX$ &  & SUSY-like & SUSY-like & SUSY-like & SUSY-like & SUSY-like\\
 (or $\tilde g\rightarrow jjX$) &&&&&&\\
 \hline
HP:~$\tilde{\ell}\rightarrow\ell X$ &  & SUSY-like & SUSY-like & SUSY-like & SUSY-like & SUSY-like\\
 \hline 
 HIG:~$h(h')\rightarrow XX$ & Higgs (DM)  &  & Higgs (DM) &  & Higgs (DM) & \\
  (or $\rightarrow XX+\mathrm{inv.}$) &&&&&&\\
 \hline 
 HIG:~$h(h')\rightarrow X+\mathrm{inv.}$ & DM  &  & DM &  & DM & \\
  \hline
   ZP:~$Z(Z')\rightarrow XX$ & $Z'$ (DM)  &  & $Z'$ (DM) &  & $Z'$ (DM) & \\
  (or $\rightarrow XX+\mathrm{inv.}$) &&&&&&\\
 \hline 
 ZP:~$Z(Z')\rightarrow X+\mathrm{inv.}$ & DM  &  & DM &  & DM & \\
  \hline
   CC:~$W(W')\rightarrow \ell X$ &   &  & (RH$\nu$) & RH$\nu$ & (RH$\nu$) & (RH$\nu$) \\
  \hline
\end{tabular}
\end{center}

\subsection{Electrically Charged LLPs:~$|Q|=1$}

Here, we need \emph{far fewer} production modes because of the irreducible gauge production associated with the electric charge. We still consider one heavy parent scenario where the heavy parent has a QCD charge, as this could potentially dominate the production cross section. Similarly, there are fewer decay modes because of the requirement of charge conservation.

Note we lump all resonant production into the $Z'$ simplified model. The reason is that the SM Higgs cannot decay into two charged particles due to the model-independent limits from LEP on charged particles masses $M\gtrsim75$ GeV. It is true that a heavy scalar could lead to LLP pair production, but due to the irreducible gauge production cross section, we do not need to rely on associated production modes and so the scalar and $Z'$ simplified models can be combined for simplicity.

For concreteness, we recommend using $Q=1$ as a benchmark for charged LLPs for the purpose of determining allowed decay modes. Because other values of $Q$ are also possible, it is worth still treating the production cross section as a free parameter. We note that there are dedicated searches for heavy quasi-stable charged particles with either $Q\gg1$ or $Q\ll1$; because those searches are by construction not intended to be sensitive to the decays of the LLP, the existing models are sufficient for characterizing these signatures and they do not need to be additionally included in our framework.

\begin{center}
\begin{tabular}{ |c|c|c|c|} 
 \hline
Channel & $\ell+\mathrm{inv.}$ &  $jj(+\mathrm{inv.})$ & $jj\ell$ \\
\hline\hline
GP:~chargino pair & SUSY-like & SUSY-like & SUSY-like \\
(or slepton pair) & & &\\
\hline
HP:~$\tilde{q}\rightarrow j X$ & SUSY-like & SUSY-like & SUSY-like \\
\hline
ZP:~$Z'\rightarrow XX$ & Higgs/Z'/DM & Higgs/Z'/DM & Higgs/Z'/DM \\
\hline
CC:~$W'\rightarrow X+\mathrm{inv.}$ & DM & DM & DM\\
\hline
\end{tabular}
\end{center}

\subsection{LLPs with Color Charge}

Because QCD is a non-Abelian group, the gauge pair production cross section of the LLP is specified by the LLP mass and its representation under $\mathrm{SU}(3)$. 

A complication of the QCD-charged LLP is that the LLP hadronizes prior to the decay. While the hadronization will not affect any hard kinematic features of its decay, it can result in interesting phenomena such as stopping in the detector, charge flipping, etc. These have been greatly explored in the context of $R$-hadrons in SUSY and we refer those interested in performing such searches to the relevant literature. {\bf [BS:~What more do we want to say about this?]}

\begin{center}
\begin{tabular}{ |c|c|c|c|} 
 \hline
Channel & $j+\mathrm{inv.}$ &  $jj(+\mathrm{inv.})$ & $j\ell^+\ell^-$ \\
\hline\hline
GP:~squark pair & SUSY-like & SUSY-like & SUSY-like \\
(or gluino pair) & & &\\
\hline
\end{tabular}
\end{center}

\section{Conventions for the Simplified Models Library}
Can have library of FeynRules UFO models for production \& decay modes, separately stitch these together. {\bf [BS:~Would folks rather have LHE files or the relevant cards for running MG? Need to figure out.]}

\section{Exceptions and Caveats to the Simplified Models} \label{sec:exceptions}
\begin{itemize}
\item Focused on the simplest and best-motivated models for now. Already many channels to consider! But this can be expanded as more work is done to fill in gaps
\item Focused on low-multiplicity signatures (so does not include dark showers, part of ongoing working group)
\item Cannot cover every single model! Some searches may need to be sensitive to exact features of a specific model, particularly as we push to be more background-dominated search regions.
\end{itemize}
