[Here the theory community will present a discussion of how particles obtain LHC-detector-visible (and beyond) lifetimes in BSM models and describe curated classes of models (including exhaustive citations) that highlight the ways in which the lifetime frontier is an ideal place where new physics could be hiding.]

\section{General Motivations From Theory}

General principles for why we expect long-lived particles to appear in beyond-the-SM physics. Examples include:
%
\begin{itemize}

\item Approximate symmetries (like RPV in SUSY)
\item Split spectra where decays are mediated by off-shell heavy particles (like in split SUSY)
\item Small mixings with SM particles or small couplings (like in hidden sectors/hidden valleys, neutrinos, dark matter, baryogenesis etc.)
\item Degeneracies enforced by approximate symmetries (nearly degenerate higgsinos/winos in SUSY or new electroweak multiplets)

\end{itemize}

From the theory perspective, the lifetime $c\tau$ in most theory models is a quasi-free parameter, as is the mass of the new particle. It therefore makes sense to organize models according to production and decay modes, and there are a specific subset that are well-motivated and appear in most classes of LLP models. We discuss these in the next section. However, we recognize that changing the mass and lifetime can have dramatic effects on the experimental signatures, and this discussion is undertaken in Chapter \ref{sec:experimentmotivation}.

\section{Simplified Models for Long-Lived Particle Production}

Simplified models are characterized by the LLP, $X$, and potentially a parent particle, $P$, such that $X$ is produced in cascade decays of $P$. 

\subsection{LLP Pair Production}

In many models, $X$ is pair produced. The reason is that LLPs are typically stabilized by an approximate symmetry, and so single production is often (although not always) suppressed by the small coupling that gives it a long lifetime. In contrast, pair production of $X$ is often allowed by the symmetry and the rates can be large.

The most important pair-production modes are outlined below. [Organize in a list instead? Don't know if there are too many sub-headings]

\subsubsection{SM Gauge Interactions}

If $X$ has electroweak or strong charges, it can be produced through the SM gauge interactions. Note that charge conservation also then has implications for the decay modes (for example, if $X$ carries color charge, then its decays necessarily include jets).

\subsubsection{Resonant Production of Parent Particle}

The parent particle can cascade decay to $X$, for example $P\rightarrow XX$. $P$ could be a SM particle ($Z$, $H$, or a charged particle) or a BSM particle ($Z'$, $H'$, or a charged particle, etc.). If sufficiently light, the parent particle can also be produced in rare meson decays. The resonant production also gives a {\bf characteristic mass scale} for the process, namely that $X$ carries away a fraction of $M_P$ in momentum.

Depending on the nature of $P$, there may be associated prompt objects. For example, if we take a Higgs portal production, there can be associated forward VBF jets or vector boson production.

\subsubsection{Non-Resonant Production of Parent Particle}

The parent can also be produced off-shell, leading to pair-production of $X$ without resonance. This is most common in models where $X$ is heavier than $P$, for example in a Higgs portal model where $M_X > M_h/2$. Such rates are typically tiny and so LLP searches are the best shot at discovering off-shell portals. In this case, the typical momentum scale of the event is set by $M_X$ rather than $M_P$.

Depending on the nature of $P$, there may be associated prompt objects. For example, if we take a Higgs portal production, there can be associated forward VBF jets or vector boson production.

\subsubsection{Parent Particle Pair Production}

$P$ can itself be pair-produced, decaying to a single $X$ each. This is common in SUSY scenarios, with pair production of superpartners that cascade to a long-lived (N)LSP. In this case, the typical momentum scale of $X$ is again set by the parent mass, but there may also be associated prompt objects associated with the prompt decay (for example, $\tilde{g}\rightarrow j j \tilde{\chi}_1$ (prompt), $\tilde{\chi}_1\rightarrow j j j$ (displaced).


\subsection{LLP Single Production}
Single production of a LLP can happen in scenarios where there are two new BSM particles that can be pair-produced in association with one another, but one escapes the detector invisibly while the other undergoes a displaced decay. There are also a few situations where the LLP has a sufficiently large production portal that it is possible to singly create it through the same small coupling that mediates decay.

\subsubsection{Electroweak Associated Production}
This is common in SUSY and other models with new electroweak multiplets. In this case, we can have associated chargino-neutralino production, or $\tilde{\chi}_1-\tilde{\chi}_2$ production. In each case, the heavier particle can decay with a long lifetime (either the chargino or the $\tilde{\chi}_2$), while the lighter particle escapes as missing momentum.

\subsubsection{Associated Production of Hidden-Sector States}
This is common in models of hidden sectors that contain a dark matter particle plus one or more additional particles. Examples include inelastic dark matter, where DM is produced in association with a heavier particle that decays with a long lifetime into DM + SM states, or DM plus a dark photon (where $M_{A'} < 2M_{\rm DM}$), such that $A'$ can be radiated off of DM when produced but decays at a displaced vertex.

\subsubsection{New Neutrino States}
If there exist a right-handed neutrino within kinematic reach of the LHC, it is singly produced via its mixing with the left-handed neutrino. For neutrino masses below the weak scale, the decay is mediated by off-shell weak gauge bosons and the lifetime can be naturally long (particularly if the mixing is small).

\subsubsection{Production in Rare Meson Decays}
[This kind of overlaps with neutrinos]

Because of the large numbers of mesons, it is possible that even particles with small couplings can be produced in rare meson decays. The LLP decay is typically suppressed by ratios of small masses to the weak scale, making it even more displaced.

\subsection{Multiple Production of Long-Lived Particles}

[THIS SECTION NEEDS WORK]

Models with new confining sectors (or large gauge couplings) can result in large amounts of hidden-sector radiation, some of which can decay back into SM particles with displaced vertices. The classic hidden valley signature is a large multiplicity of (typically soft) long-lived particles, potentially associated with prompt objects.

\subsubsection{Dark Shower}
If the hidden sector contains a new confining group, the hidden sector states shower and hadronize. Some hidden-sector hadrons may be stable, but others can mix with SM states such as the Higgs and pions, leading to long lifetimes. The challenge in these models is finding an adequate parameterization that is sufficiently general to capture most of the interesting dynamics without having too many free parameters. [Topic of discussion for workshop?]


[Do we want to include ``soft clouds'' or ``soft soup'' or whatever we were calling them in this section?]

\subsubsection{Quirks}

In this case, the quarks of the new confining group are much heavier than the confinement scale. The result is that the resulting ``quirks'' travel apart in the detector and are pulled back together, resulting in large amounts of radiation and non-trivial tracking patterns. If the quirks annihilate in the detector, this can also give rise to other striking signatures that are potentially out-of-time.

\subsection{Rare Meson Decays}

[I've actually folded this into the earlier discussions because I don't think this is really a separate category]

\section{Long-Lived Particle Decay Modes}

Determined in part by charge of LLP (for example, a strongly interacting LLP will necessarily decay to at least one jet). However, most of the decay modes can apply to any LLP. The most important part is the \emph{visible} part of each decay mode, although the invisible final-states can give rise to missing momentum and mean that the vertex is not fully reconstructible.

\begin{itemize}
\item Jets
\item Electrons and muons
\item Taus
\item SM gauge bosons (on- or off-shell)
\item Higgs bosons
\item Photons
\item Missing momentum
\end{itemize}


