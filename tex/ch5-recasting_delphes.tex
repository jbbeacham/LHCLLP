
% Tools
\newcommand{\MG}{\textsc{MadGraph}~5\_aMC@NLO}
\newcommand{\PY}{\textsc{Pythia}~8}
\newcommand{\MA}{\textsc{MadAnalysis}~5}
\newcommand{\MW}{\textsc{MadWidth}}
\newcommand{\MAnorm}{{MadAnalysis}~5}
\newcommand{\FJ}{\textsc{FastJet}}
\newcommand{\DEL}{\textsc{Delphes}}
\newcommand{\ROOT}{\textsc{Root}}

%latin
%\newcommand{\eg}{\textit{e.g.}}
%\newcommand{\ie}{\textit{i.e.}}

% environments
%\def\be{\begin{equation}}
%\def\ee{\end{equation}}
%\def\bsp#1\esp{\begin{split}#1\end{split}}


\section{Handling long-lived particles in \DEL-based detector simulations}
\label{sec:ch5-recastingDelphes}

\subsection{Long-lived particle simulation in \DEL~3.4.1}

The \DEL\ package~\cite{deFavereau:2013fsa} allows for the generic simulation of
the response of a typical detector used in high-energy physics experiments. It
is widely used for simulating the effects of the ATLAS and CMS detectors or the
hypothetical detectors that could be used for the future FCC and CLIC projects.
The architecture of \DEL\ is composed of distinct
and specialized modules that interact with each other. The detector is described
by the user through an input card, where the modules to be used in the
simulation are sequentially enumerated and their input parameters are specified.

The detector simulation relies on a mix of parametric and algorithmic
modules. More precisely, tracking is simulated through efficiency and smearing
functions that are applied to the properties of the electrically-charged stable
particles. The particles are then propagated to the calorimeters and dedicated
modules simulate the energy deposits in the electromagnetic and hadronic
calorimeters. The output of such a step consists in a list of calorimetric towers.
Moreover, \DEL\ includes a particle-flow like
algorithm that combines tracking and calorimetric data in order to
improve the identification of the final-state objects and the resolution on
their reconstructed momenta.

Jet clustering is performed via an internal call to the \FJ\
package~\cite{Cacciari:2011ma}, which takes as input the
list of calorimeter towers or, alternatively, the particle-flow candidates and
outputs jet objects. Lepton and photon isolation is then handled through a
specific isolation module. Finally, \DEL\ takes care of removing the
double-counting of objects that could be simultaneously identified as element of
different collections. The final output is stored in a \ROOT\ file.

In addition, \DEL\ allows for the simulation of pile-up effects by
superimposing minimum-bias events attached to displaced interaction vertices
along the beam direction
to the hard scattering event. Procedures mimicking pile-up
removal can then be configured in the input card. The subtraction of the charged
particles belonging to pile-up vertices is performed at the tracker level.
Neutral particles are removed by applying the jet area
method~\cite{Cacciari:2007fd} supplied within the \FJ\ package. More advanced methods are also available in \DEL, such as the
{\sc Puppi}~\cite{Bertolini:2014bba} or {\sc SoftKiller}~\cite{Cacciari:2014gra}
techniques, and they can easily be added to the input card. The loss of
performance originating from pile-up, in particular relative to the isolation, is
automatically accounted for.

The official \DEL\ package (version 3.4.1) with the default detector cards needs
to be adapted for the proper handling of LLPs.
By default, the decay products of a long-lived particle enter the
simulation as if the corresponding decay would have occurred within the
tracker volume. However, the user has the possibility to define a volume in
which the particle can decay and still be detected outside the tracker volume. This is achieved in
practice by making use of the \verb+RadiusMax+ parameter of the
\verb+ParticlePropagator+ module, that is by default set to the tracker radius
stored in the \verb+Radius+ parameter. When setting the \verb+Radius+ and
\verb+RadiusMax+ parameters to different values, the particles decaying outside
the tracker volume, but inside the `decay volume' of radius \verb+RadiusMax+, are
included in the collection of stable particles stored in the output \ROOT\ file.
They can in this way be used for an offline, more correct, treatment.

Moreover, several modules that are not used in the default ATLAS and CMS cards
could serve for a better simulation of the long-lived particles in \DEL . For
instance, the \verb+TrackSmearing+ module allows the user to smear the track momentum
according to the impact parameter in the transverse plane ({\it i.e.}, the $d_0$
parameter) and in the longitudinal plane ({\it i.e.}, the $d_z$ parameter).

By default, the detector simulation in \DEL\ totally ignores the presence of any
LLP. While this is convenient for neutral particles like a
neutralino which could be considered as invisible from the detector standpoint,
a charged particle leaves tracks in the tracker and would interact with the
calorimeters if its lifetime is large enough.
In this case, if the long-lived particle decays inside the tracker, its
trajectory is properly propagated to the calorimeters and the displaced vertex
is correctly accounted for. However, if the long-lived particle decays outside
the tracker, its decay products are ignored in \DEL, unless the \verb+RadiusMax+
parameter has been specified to be larger than the tracker \verb+Radius+
parameter. In this case the decay products can be found in the
\verb+Delphes/stableParticles+ output collection and
treated with adequate smearing functions and efficiency directly from the \DEL\
output.

Finally, disappearing tracks are simply treated as missing energy in \DEL. Emerging tracks or tracks containing kinks are not treated appropriately, in the sense that the parameterizations required for a proper description of such signatures has not been implemented yet. Also, \DEL\ does not include any trigger simulation, and the latter is in general complex in the case of LLPs.

\subsection{Displaced tracks with the \MA\ tune of \DEL}

The so-called \DEL-LLP package can be installed from the version v1.6 of
\MA~\cite{Conte:2012fm,Conte:2014zja} and contains improvements of \DEL\
specific to LLPs. It
leads to new possibilities for phenomenological investigations of long-lived
particles and the recasting of related LHC analyses. More information is
available on the web page:\\ \noindent \url{https://madanalysis.irmp.ucl.ac.be/wiki/MA5LongLivedParticle}

This new package was designed to handle neutral LLPs that decay
into leptons within the tracker volume. Realistic efficiencies are applied to
the displaced tracks and several parameters specific to this kind of analysis
have been made available within \MA. An extension to the case of neutral
LLPs decaying into muons outside the tracker volume can be easily
implemented, the muons being thus reconstructed only through their hits in the
muon chambers. The simulation of the displaced leptons is performed through efficiencies and resolution functions to be specified by the user. Furthermore, another
extension allowing the user to handle long-lived charged particles that decay into
leptons could be implemented.

The \DEL-LLP package contains a new module called \verb+MA5EfficiencyD0+ that
allows for the definition of a track reconstruction efficiency parameterized as a function of the $|d_0|$ and $d_z$ parameters (named \verb+d0+ and \verb+dz+ in the \DEL\ input
card). The default efficiency function, specified via a \verb+DelphesFormula+ is taken from the
8~TeV tracking performance of CMS~\cite{Khachatryan:2014mea},\\ 
\noindent \url{https://twiki.cern.ch/twiki/bin/view/CMSPublic/DisplacedSusyParametrisationStudyForUser},\\
\begin{verbatim}
 set EfficiencyFormula {
   (d0<=20) * (-5.06107e-7 * d0**6 + 0.0000272756 * d0**5 - 0.00049321 * d0**4
      + 0.00287189 * d0**3 + 0.00522007 * d0**2 - 0.0917957 * d0 +  0.924921) +
   (d0> 20) * (0.00)
 }
\end{verbatim}
In addition, the data-format of \DEL\ has been extended so that the \verb+Muon+ and
\verb+Electron+ classes now include the transverse ($|d_0|$) and longitudinal ($d_z$) impact parameters relative to the closest approach point (encoded
in the \verb+d0+ and \verb+dz+ variables), the coordinates of the closest
approach point $(x_d, y_d, z_d)$ (encoded in the \verb+xd+, \verb+yd+ and
\verb+zd+ variables) and the four-vector of the vertex from which the lepton is
originating from $(t_p, x_p, y_p, z_p)$ (encoded within the \verb+tp+,
\verb+xp+, \verb+yp+ and \verb+zp+ variables), the latter quantity being
evaluated from Monte Carlo information.

%\renewcommand{\arraystretch}{1.2}
%\setlength{\tabcolsep}{12pt}
\begin{table}[t]
\begin{center}
  \begin{tabular}{l || l | c  c | c }
    Region & $c\tau_{\tilde t}$ [cm] & MA5 & CMS& Difference [\%] \\ \hline\hline
%    \multirow{4}{*}{\bf SR-1}& 0.1 & 3.89   & 3.8  & 2\\
    \multirow{4}{*}{\bf SR-1}& 0.1 & 3.89   & 3.8  & 2\\
                             & 1   & 4.44   & 5.2  & 15\\
                             & 10  & 0.697  & 0.8  & 15\\
                             & 100 & 0.0610 & 0.009& $> 100\%$\\ \hline
    \multirow{4}{*}{\bf SR-2}& 0.1 & 0.924  & 0.94 & 2\\
                             & 1   & 3.87   & 4.1  & 5\\
                             & 10  & 0.854  & 1.0  & 15\\
                             & 100 & 0.0662 & 0.03 & $\sim 100\%$\\ \hline
    \multirow{4}{*}{\bf SR-3}& 0.1 & 0.139  & 0.16 & 15\\
                             & 1   & 6.19   & 7.0  & 10\\
                             & 10  & 4.45   & 5.8  & 25\\
                             & 100 & 0.497  & 0.27 & 85\\
  \end{tabular}
  \caption{Number of events populating the three signal regions (SR-1, SR-2 and SR-3) of Ref.~\cite{CMS-PAS-EXO-16-022}
   for different stop decay lengths ($c\tau_{\tilde t}$).
    We compare the CMS and \MA~(MA5) results in the second
    and third column of the table, respectively, and the difference taken relatively to CMS is
    shown in the last column.}
  \label{tab:cms_exo}
\end{center}
\end{table}
\renewcommand{\arraystretch}{1}


Consequently, the data-format of \MA\ has been extended, so that the \verb+Muon+ and
\verb+Electron+ classes now contains the \verb+d0()+ and \verb+dz()+ methods allowing to
access the value of the $|d_0|$ and $d_z$ parameters, the \verb+closestPoint()+
method that returns the coordinates of the closest approach point (through the
\verb+X()+, \verb+Y()+ and \verb+Z()+ daughter methods) and the
\verb+vertexProd()+ method that returns coordinates of the displaced vertex from
which the lepton originates from (through the \verb+X()+, \verb+Y()+ and
\verb+Z()+ daughter methods). An analysis example \cite{MA5:longlivedleptons} can be found in the public analysis database of \MA,\\
\noindent \url{http://madanalysis.irmp.ucl.ac.be/wiki/PublicAnalysisDatabase},\\
\noindent where information about the re-implementation in \MA\ of Ref.~\cite{CMS-PAS-EXO-16-022}  
analysis of 2.6~fb$^{-1}$ of 13~TeV LHC data is available. This is a search for long-lived particles decaying into electrons and muons, where signal
events are selected by requiring the presence of either an electron or a muon
whose transverse impact parameter lies between 200~$\mu$m and 10~cm. For a given
benchmark signal where a pair of long-lived stops is produced through usual QCD
interactions and where each stop further decays into a displaced $b$-jet and a
displaced lepton. In Table~\ref{tab:cms_exo}, we present the number of events
surviving the selection of the three different signal regions of the CMS analysis.
Our event generation has been preformed with {\sc Pythia} 8~\cite{Sjostrand:2007gs}
and for the benchmark
scenario SPS 1a.
We observe that a good agreement is obtained, except in the case of a long stop
lifetimes, $c\tau \gtrsim 1$~m. 
This is however the region in which no public information on
the CMS reconstruction efficiencies is available.

The \MA\ tune of \DEL\ has neither been designed for disappearing
(or appearing) tracks nor for track kinks. Concerning the disappearing (or appearing)
tracks, the only missing experimental ingredients are the track reconstruction efficiency
and resolution as a function of the number of missing hits in the (inner) outer
layers of the tracker. There is to this date no public material on the tracking
performance description related to track kinks.

\subsection{What about other LLP signatures?}

In this section, we briefly discuss how \DEL\ could be improved for a better
handling of LLP signatures.

\subsubsection{Displaced jets}
\label{sec:dispjets}

Displaced jets are jets that are reconstructed either from stand-alone calorimeter
information, or from the particle-flow input with a minimum requirement on the
multiplicity of tracks with high transverse displacement (see
Ref.~\cite{CMS:2014wda}). Conceptually, such jets can be handled in \DEL\
provided that the displaced tracks are properly parametrized. As described
above, a module designed to smear the full set of track properties, including
their transverse and longitudinal displacement, exists ({\it i.e.}, the
\verb+TrackSmearing+ module). In addition, efficiencies based on displacement
parameters have already been implemented in \MA\ (see above) and a module that
performs the matching of an existing jet collection with a track collection
based on track displacements is very similar to the already existing
\verb+TrackCountingBTagging+ module. Minor modifications to this module are
hence needed to be able to select tracks based on an absolute displacement
instead of the impact parameter significance. Finally, in order to be able to
perform a displaced jet selection, one would need a (not-yet existing) module
that performs jet clustering on the basis of the secondary vertices based and
the displaced tracks matched with these jets. Alternatively, a module that
includes a vertex reconstruction efficiency and a module including a vertex
position smearing could be implemented.


\subsubsection{Displaced vertices}

The missing modules described in Section~\ref{sec:dispjets} could perfectly
serve the purpose of a displaced vertex analysis. Provided that tracking
efficiencies and resolutions are available as a function of the full set of
tracking parameters ($d_0$, $d_z$, $p_T$, $\phi$, and $\theta$ ) and,
eventually, of the Monte Carlo truth vertex position, a simple vertexing
algorithm can be implemented in \DEL.

\subsubsection{Discussion: \DEL\ versus a specific parametric simulation}

In a fully parametric simulation of the detector, the detector effects are encoded in
terms of efficiencies and resolution functions. Such a simulation is
typically very fast, but is likely to suffer from a lack of accuracy in the
modeling of complex observables such as jet properties and missing energy. The
\DEL\ simulation is an admixture of such a parametric simulation, and of an
algorithmic one. This is slower, but has the clear advantage of correctly
treating in particular the reconstruction of the jets and the missing energy.

In order to be able to answer whether \DEL\ should be used in place of a fully
parametric simulation in recasting  LLP analyses,
further studies are needed. In the meantime, the following guidelines could be
used. If the signal selection is based only on displaced tracks, a simple
parametric simulation should in principle be sufficient. This simulation could encapsulate the
track reconstruction efficiency and resolution, including pile-up effects. \DEL\
could then optionally be used to mix the resulting ``reconstructed" tracks with the additional tracks originating from the pile-up vertices.
On the other hand, if the analysis under consideration additionally uses
calorimetric information ({\it i.e}, jets or missing energy), \DEL\ should be
preferred to a fully parametric framework. However, a precise
quantification of these effects cannot be assessed without detailed comparative
studies between the two approaches. Finally, it should be pointed out that
neither of these techniques can be used to correctly simulate the instrumental background,
which is challenging to simulate and is discussed in Chapter \ref{sec:backgrounds}.


