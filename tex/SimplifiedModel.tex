\noindent {\bf Chapter editors:}~James Beacham, Giovanna Cottin, David Curtin, Jared Evans, Zhen Liu, Michael Ramsey-Musolf, Jessie Shelton, Brian Shuve \\
\text{ \; }\\
\noindent {\bf Contributors:}~Andreas Albert, Oliver Buchmueller, Alessandro Davoli, Andrea De Simone, Kristian Hahn, Jan Heisig, Thomas Jacques, Matthew McCullough, Stephen Mrenna, Hiaolin Li, Marco Trovato, Jiang-Hao Yu
\text{ \; }\\
\text{ \; }\\

%%%%%%%%%%%%%%%%
%% JS: this new introduction is aimed at helping this document stand alone; I'd imagine in the final white paper a lot of these ideas would be introduced in the overall introduction.

\noindent Long-lived particles (LLPs) arise in many well-motivated theories of physics beyond the SM, ranging from  heavily studied scenarios such as the minimal supersymmetric SM (MSSM)~\cite{Fayet:1976et,Fayet:1977yc,Farrar:1978xj,Fayet:1979sa,Dimopoulos:1981zb} to newer theoretical frameworks such as neutral naturalness~\cite{Chacko:2005pe,Burdman:2006tz,Cai:2008au} and hidden sector dark matter~\cite{Boehm:2002yz,Boehm:2003ha,Pospelov:2007mp,Pospelov:2008zw,ArkaniHamed:2008qn,Pospelov:2008jd}.
Macroscopic decay lengths of new particles naturally arise from the presence and breaking of symmetries, which can be motivated by cosmology (such as dark matter and baryogenesis)~\cite{Bouquet:1986mq,Campbell:1990fa,Cui:2012jh,Barry:2013nva,Cui:2014twa,Ipek:2016bpf,Feng:2008ya,Baumgart:2009tn,Kaplan:2009ag,Chan:2011aa,Dienes:2011ja,Dienes:2012yz,Kim:2013ivd}, neutrino masses~\cite{Helo:2013esa,Antusch:2016vyf,Graesser:2007yj,Graesser:2007pc,Izaguirre:2015pga,Maiezza:2015lza,Batell:2016zod,Cottin:2018kmq,Nemevsek:2018bbt}, as well as solutions to the hierarchy problem~\cite{Giudice:1998bp,Burdman:2006tz,Cai:2008au,Chacko:2005pe,Fan:2011yu,Barbier:2004ez,Csaki:2013jza,Arvanitaki:2012ps,ArkaniHamed:2012gw}; indeed, LLPs are generically a prediction of new hidden sectors at and below the weak scale~\cite{Chen:1995yu,Thomas:1998wy,Feng:1999fu,Strassler:2006im,Strassler:2006ri,Strassler:2006qa,Han:2007ae,Strassler:2008bv,Strassler:2008fv}. An extensive and encyclopedic compilation of theoretical motivations for LLPs has already been performed for the physics case of the proposed MATHUSLA experiment~\cite{Curtin:2018mvb}, and we refer the reader to this document and the references therein for an in-depth discussion of theoretical motivations for LLPs.
Given the large number of theories predicting LLPs, however, it is clear that a comprehensive search program for LLPs is critical to fully leverage the LHC's immense capability to illuminate the physics of the weak scale and beyond.

The simplified model framework has proven to be a highly successful approach to characterizing signals of beyond the SM (BSM) physics.
Simplified models have driven the development of searches for new signatures at the LHC and allowed existing searches to be reinterpreted for many models beyond the one(s) initially targeted in the analysis.
Comprehensive simplified model programs exist for scenarios featuring prompt decays of new particles~\cite{ArkaniHamed:2005px,Knuteson:2006ha,ArkaniHamed:2007fw,Aaltonen:2007dg,Alwall:2008ag,Alwall:2008va,Alves:2011wf} or dark matter produced at colliders~\cite{Petriello:2008pu,Dudas:2009uq,Goodman:2011jq,An:2012va,Frandsen:2012rk,Dreiner:2013vla,Cotta:2013jna,Abdallah:2015ter,Abercrombie:2015wmb,Boveia:2016mrp}.
Simplified models are so successful because the majority of search sensitivity is driven by only a few broad aspects of a given BSM signature, such as the production process, overall production rate, and decay topology.
Meanwhile, the sensitivity of searches is typically insensitive to other properties such as the spin of the particles involved~\cite{Edelhauser:2015ksa,Edelhauser:2014ena,Arina:2015uea,Kraml:2016eti}.

To extend the simplified model approach to LLP signatures in a systematic way, we develop a proposal for a set of simplified models which aims to ensure that experimental results can be characterized as follows:~(i) {\em powerful}, covering as much territory in model space as possible; (ii) {\em efficient}, reducing unnecessary redundancy among searches; (iii) {\em flexible}, so that they are broadly applicable to different types of models; and (iv) {\em durable}, providing a common framework for Monte Carlo (MC) simulation of signals and facilitating the communication of results of LLP searches so that they may be applied to new models for years to come.
We elaborate on these goals in Section~\ref{sec:goals}.
This framework helps illuminate gaps in coverage and highlight areas where new searches are needed, and we undertake such a study in Chapter~\ref{sec:experimentcoverage}. Our efforts build on earlier work proposing simplified model programs for LLPs motivated by particular considerations such as SUSY or dark matter (DM)~\cite{Heisig:2012zq,Liu:2015bma,Heisig:2015yla,Khoze:2017ixx,Mahbubani:2017gjh,Buchmueller:2017uqu}.

In our work, we concentrate on establishing an initial basis of simplified models representative of theories giving rise to final states with one or two LLPs~\footnote{Some models predict moderately higher LLP multiplicities, but the coverage of such signatures from 1-2 LLP searches is good provided the LLPs do not overlap in the detector. Our proposed simplified models are not, however, representative of high-multiplicity signatures such as dark showers (see Section~\ref{sec:simplified_future} and Chapter~\ref{sec:showers}).}.
The simplified model approach is very powerful for LLP signatures:~the typically lower backgrounds for displaced signatures allow searches to be highly inclusive with respect to other objects in the event or the identification of objects originating from the decay of an LLP.
This enables a single analysis to have sensitivity to a wide variety of models for LLP production and decay.


We organize our simplified models in terms of {\bf LLP channels} characterized by a combination of a particular LLP production mode with a particular decay mode.
Because the production and decay positions of LLPs are physically distinct~\footnote{Indeed, the decay position may be so far from the collision point that external detectors can also be used to search for ultra-long-lived neutral or milli-charged particles~\cite{Pinfold:2009oia,Haas:2014dda,Chou:2016lxi,Gligorov:2017nwh,Feng:2017uoz}. }, it is often possible to factorize and consider separately their production and decay~\footnote{In addition to production and decay, a third consideration is the propagation of particles through the detector. While neutral LLPs undergo straightforward propagation, states with electric or color charge (\emph{e.g.}, SUSY $R$-hadrons), or particles with exotic charges such as magnetic monopoles or quirks, typically engage in a more complicated and often very uncertain traverse through the detector. This spoils the factorization of LLP production and decay.  The subtleties related to LLPs with electric or color charge is discussed more in Section~\ref{sec:coloredLLPs}. A trickier question is how to best simulate such states:~since LLPs with electric or color charge interact with the detector material, there must be an interface between the detector simulation software and the program implementing decay. This is discussed further in Section~\ref{sec:geant}.}.
For each LLP channel, the lifetime of the LLP is taken to be a free parameter.
We emphasize that the LLP channel defined here is \emph{not} the same as an experimental signature that manifests in the detector:~a single channel can give rise to many different signatures depending on where (or whether)~\footnote{The case of detector-stable particles is understood to be included in the simplified models by setting $c\tau\to\infty$. In this case there is manifestly no dependence on the decay mode. See Section~\ref{sec:decmodes} for further details.} the LLP decays occur inside the detector, while a single experimental search for a particular signature could potentially cover many simplified model channels.
In this chapter, we focus on the construction and simulation of a concrete basis of LLP simplified model channels; a partial mapping of existing searches into our basis of simplified models is discussed in Chapter~\ref{sec:experimentcoverage}, along with the highest-priority gaps in current coverage and proposals for new searches.

As discussed in the existing simplified model literature, simplified models have their own limited range of applicability~\cite{Alves:2011wf,Abdallah:2015ter,Abercrombie:2015wmb,Boveia:2016mrp,Ambrogi:2017lov}.
For example, the presentation of search results in terms of simplified models often assume 100\% branching fractions into particular final states.
In a UV model where the LLP decays in a very large number of ways, none of the individual simplified model searches may be sufficient to constrain it.
Similarly, if the LLP is produced in a UV model with other associated objects that spoil the signal efficiency (for example, the production of energetic, prompt objects collimated with the LLP such that the signal fails isolation or displacement criteria; this is particularly important for high-multiplicity or dark-shower scenarios, as discussed in Chapter~\ref{sec:showers}), then the simplified model result does not apply and a more targeted analysis is required to cover the model.
Nevertheless, the simplified models framework allows us to organize possible production modes and signatures in a systematic way and identify if there are any interesting signals or parts of parameter space that are missed by current searches.Therefore, we present a proposal for simplified models here with the understanding that there exist scenarios where UV models remain important for developing searches and presenting results.

The basis of simplified models presented here is a starting point, rather than a final statement.
The present goal is to provide a set of simplified models that covers the majority of the best-motivated and simplest UV models predicting LLPs, which we outline in Section~\ref{sec:motivated_theories}. Many of these contain singly and doubly produced LLPs (or in some cases, three-to-four relatively isolated LLPs, which are typically covered well by searches for 1$-$2 LLPs) and so we restrict our simplified model proposal to cover these multiplicities.
By design, simplified models do not include all of the specific details and subtle features that may be found in a given complete model.
Therefore, the provided list is meant to be expanded to cover new or more refined models as the LLP-search program develops.
For instance, extending the simplified model framework to separately treat final states with heavy-flavor particles is of great interest (in analogy with the prompt case~\cite{Essig:2011qg,Brust:2011tb,Papucci:2011wy}); see Section~\ref{sec:simplified_future} for a discussion of this and other limitations of the current framework along with future opportunities for expansion.
High-multiplicity signatures such as dark showers or emerging jets present different experimental and theoretical issues, which are discussed in Chapter~\ref{sec:showers}.
Finally, a broader set of simplified models may be needed to present the results of experimental searches and to allow ready application of experimental results to UV models of interest (see Chapter~\ref{sec:reint}).

%%%%%%%%%%%%%%%%%%%%%%%%%%%%%%%
\section{Goals of the Present Simplified Model Framework}\label{sec:goals}
%%%%%%%%%%%%%%%%%%%%%%%%%%%%%%%

The purpose of the simplified model framework is to provide a simple, common language that experimentalists and theorists can use to describe theories of LLPs and the corresponding mapping between models and experimental signatures.
We therefore want our simplified model space to:
%
\begin{enumerate}
%
\item Use a minimal but sufficient set of models to cover a wide range of the best-motivated theories of LLPs;
\item Furnish a simple map between models and signatures to enable a clear assessment of existing search coverage and possible gaps; 
\item Expand flexibly when needed to incorporate theories and signatures not yet proposed;
\item Provide a concrete MC signal event generation framework for signals;
\item Facilitate the reinterpretation of searches by supplying a sufficiently varied set of standard benchmark models for which experimental efficiencies can be provided for validation purposes.
\end{enumerate}
%
Note that points \#1 and \#5 are somewhat in tension with one another:~we wish to have a compact set of models that can be the subject of systematic study in terms of experimental signatures, but expressing experimental results in terms of only this set of simplified models may make it challenging to reinterpret experimental searches for UV models that are not precisely described by one of the simplified models.
In this section, we prioritize having a minimal set of simplified models for the purpose of studying experimental coverages and generating new search ideas, while we defer a discussion of simplified models in the presentation and
reinterpretation of search results to Chapter~\ref{sec:reint}.~\footnote{We note that, in general, more benchmark models may be needed for enabling reliable reinterpretation than the minimal set discussed here. An example where an extended set of simplified models is used can be seen in the heavy stable charged particle (HSCP) reinterpretation in Section~\ref{sec:ch5-smsHSCP} (Table~\ref{tab:defModels}).}

In the remainder of this chapter, we construct a proposal for a minimal basis of simplified models for events with one or two LLPs.
We begin with a discussion of the well-motivated UV theories that predict the existence of LLPs, and identify a set of umbrella models that yield LLPs in Section~\ref{sec:motivated_theories}.
We next identify the relevant (simplified) production and decay modes for LLPs in Section~\ref{sec:building_blocks}, emphasizing that each channel for production and decay has a characteristic set of predictions for the number and nature of {\em prompt} accompanying objects (AOs) producing along with the LLP.
In Section~\ref{sec:proposal}, we combine these production and decay modes into our simplified model basis set and highlight how different umbrella models naturally populate the various LLP channels.
Section~\ref{sec:library} and Appendix~\ref{sec:library_more} present a framework and instructions for how the best-motivated simplified model channels can be simulated in Monte Carlo (MC) using a new model library provided in Appendix~\ref{sec:library_more}.
Finally, limitations of the existing framework, along with opportunities for its further development are outlined in Section~\ref{sec:simplified_future}.

%%%%%%%%%%%%%%%%%%%%%%%%%%%%%%%%%%%%%
\section{Existing Well-Motivated Theories for LLPs}\label{sec:motivated_theories}
%%%%%%%%%%%%%%%%%%%%%%%%%%%%%%%%%%%%%

Here we provide a brief distillation of many of the best-motivated theories with LLPs into five over-arching categories, focusing in particular on those that give rise to single and double production of LLPs at colliders.
We emphasize that each of these categories is a broad umbrella containing many different individual models containing LLPs; in many cases, the motivations and model details among theories within a particular category may be very different, but tend to predict similar types of LLPs. Additionally, the categories are not mutually exclusive, with several examples of UV models falling into one or more category.
In all cases, long lifetimes typically arise from some combination of hierarchies of scales in interactions that mediate decays; small couplings; and phase space considerations (such as small mass splittings between particles or large multiplicities of final-state particles in a decay).

The UV umbrella models we consider are:
%
\begin{itemize}

\item {\bf Supersymmetry-like theories (SUSY).}~This category contains models with multiple new particles carrying SM gauge charges and a variety of allowed cascade decays.
Here LLPs can arise as a result of approximate symmetries (such as $R$-parity~\cite{Barbier:2004ez} or indeed SUSY itself in the case of gauge mediation~\cite{Dimopoulos:1996vz}) or through a hierarchy of mass scales (such as highly off-shell intermediaries in split SUSY~\cite{ArkaniHamed:2004fb}, or nearly-degenerate multiplets~\cite{Chen:1995yu,Thomas:1998wy,Byrne:2003sa}, as in anomaly-mediated SUSY breaking~\cite{Feng:1999fu}).  Finally, models of SUSY hidden sectors such as Stealth SUSY~\cite{Fan:2011yu} generically lead to LLPs. 
Our terminology classifies any non-SUSY models with new SM gauge-charged particles, such as composite Higgs or extra-dimensional models, under the SUSY-like umbrella because of the prediction of new particles above the weak scale with SM gauge charges.  In this category, LLP production is typically dominated by SM gauge interactions, whether of the LLP itself or of a heavy parent particle that decays to LLPs.

\item {\bf Higgs-portal theories (Higgs).}~In this category, LLPs couple predominantly to the SM-like Higgs boson.
This possibility is well motivated because the SM Higgs field provides one of the leading renormalizable portals for new gauge-singlet particles to couple to the SM, and the experimental characterization of the Higgs boson leaves much scope for couplings of the Higgs to BSM physics~\cite{Khachatryan:2014jba,Aad:2015pla}.
The most striking signatures here are exotic Higgs decays to low-mass particles~\cite{Curtin:2013fra} (as in many Hidden Valley scenarios~\cite{Strassler:2006im,Strassler:2006ri}), which can arise in models of neutral naturalness~\cite{Chacko:2005pe,Burdman:2006tz,Craig:2015pha} and DM~\cite{Silveira:1985rk}, as well as in more exotic scenarios such as relaxion models~\cite{Beauchesne:2017ukw}.
The Higgs is also special in that it comes with a rich set of associated production modes in addition to the dominant gluon-fusion process, with vector-boson fusion (VBF) and Higgs-strahlung (VH) production modes allowing novel opportunities for triggering on and suppressing backgrounds to Higgs-portal LLP signatures.
Indeed, in many scenarios where LLPs are produced in exotic Higgs decays, associated-production modes can be the only way of triggering on the event.

\item {\bf Gauge-portal theories (ZP).}~This category contains scenarios where new vector mediators can produce LLPs.
These are similar to Higgs models, although here the vector mediator is predominantly produced from $q\bar{q}$ initial  states without other associated objects except for gluon initial-state radiation (ISR).
Examples include models where both SM fermions and LLPs carry a charge associated with a new $Z'$ (for a review, see Ref.~\cite{Langacker:2008yv}), as well as either Abelian or non-Abelian dark photon or dark $Z$ models~\cite{Holdom:1985ag} in which the couplings of new vector bosons to the SM are mediated by kinetic mixing.
Scenarios with LLPs coupled to new gauge bosons are well motivated by theories of DM, particularly models with significant self-interactions~\cite{Feng:2009hw,Buckley:2009in,Tulin:2012wi} and/or sub-weak mass scales~\cite{Boehm:2003hm,Boehm:2003ha,Pospelov:2007mp,ArkaniHamed:2008qp,ArkaniHamed:2008qn}.

\item {\bf Dark-matter theories (DM):}~Non-SUSY and hidden-sector DM scenarios are collected in this category, which encompasses models where the cosmological DM is produced as a final state in the collider process.
Examples of multi-component DM theories include models of new electroweak multiplets~\cite{Thomas:1998wy,Cirelli:2005uq,Cirelli:2009uv,FileviezPerez:2008bj}, strongly interacting massive particles (SIMPs)~\cite{Hochberg:2015vrg}, inelastic dark matter~\cite{TuckerSmith:2001hy}, and models with DM coannihilation partners~\cite{Griest:1990kh,Baker:2015qna,Khoze:2017ixx}.
In many of these models, the collider phenomenology and LLP lifetime can be tied to the DM relic abundance.
The main feature distinguishing this category from the Higgs and gauge scenarios above is that an explicit detector-level signature of a dark matter candidate, \emph{i.e.,} missing energy ($\slashed{E}_{\rm T}$), is a necessary and irreducible component~\cite{Strassler:2006im,Strassler:2006ri,Baumgart:2009tn,Falkowski:2010cm,Bai:2011jg,Primulando:2015lfa,Bai:2015nfa,Izaguirre:2015zva,Khoze:2017ixx,Buchmueller:2017uqu,Garny:2017rxs,Davoli:2017swj}.

\item {\bf Heavy neutrino theories (RH$\nu$):}~The see-saw mechanism of SM neutrino mass generation predicts new right-handed neutrino (RHN) states~\cite{Minkowski:1977sc,Yanagida:1979as,Mohapatra:1979ia,Glashow:1979nm,Mohapatra:1986bd}.
If the RHNs have masses in the GeV to TeV range, they typically have a long lifetime and can be probed at the LHC~\cite{Keung:1983uu,Ferrari:2000sp,Basso:2008iv,Atre:2009rg,Perez:2009mu,Nemevsek:2011hz,Helo:2013esa,Izaguirre:2015pga,Maiezza:2015lza,Batell:2016zod,Nemevsek:2016enw,Accomando:2016rpc,Accomando:2016sge,Caputo:2017pit,Accomando:2017qcs,Cottin:2018kmq,Nemevsek:2018bbt,Helo:2018qej}.
Examples of well-motivated, UV-complete models with RHNs include the neutrino minimal SM ($\nu$MSM)~\cite{Asaka:2005an,Asaka:2005pn} and the left$-$right symmetric model~\cite{Pati:1974yy,Mohapatra:1974gc,Senjanovic:1975rk,Senjanovic:1978ev}.
Characteristic features of models in this category are LLPs produced singly via SM neutral- and charged-current interactions, and lepton-rich signatures in terms of prompt and displaced objects (often in association with quarks).
For example, in extended scenarios like left$-$right symmetric models, production through new right-handed $W$ and $Z$ bosons can result in between one and four LLPs, and cascade decays between RHNs can lead to phenomena such as doubly displaced decays.
Additionally, RHNs can be produced via Higgs decays~\cite{Graesser:2007yj, Graesser:2007pc,Maiezza:2015lza,Accomando:2016rpc,Das:2017rsu,Caputo:2017pit}.

\end{itemize}
%
It is possible for a given model to fit into two or more of the umbrella UV model categories.
For example, a SUSY theory with a stable lightest SUSY particle (LSP) could have the LSP serve as a dark matter candidate, while alternatively DM could be a new electroweak multiplet, giving rise to SUSY-like signatures~\cite{Thomas:1998wy,Cirelli:2005uq,Cirelli:2009uv,FileviezPerez:2008bj}.  In other models featuring  particles charged under a confining gauge group (such as ``quirks''~\cite{Kang:2008ea}), there can exist many production possibilities for the LLPs, including via the Higgs portal and the annihilation of new TeV-scale states (see, for example, Ref.~\cite{Chacko:2015fbc}).
Thus, the umbrella models should not be considered as exclusive categories, but rather as over-arching scenarios that motivate particular classes of signatures (such as new SM gauge-charged particles in the SUSY-like category, or presence of $\slashed{E}_{\rm T}$ in DM models).

%
In developing our simplified model framework below, we construct maps between these UV model categories and the simplified model channels to illuminate some of the best-motivated combinations of production and decay modes for LLPs.
This allows us  to focus on the most interesting channels and assess their coverage in Chapter~\ref{sec:experimentcoverage}.

%%%%%%%%%%%%%%%%%%%%%%%%%%%%%%%%%%%%%%%%%%%%%%
\section{The Simplified Model Building Blocks}\label{sec:building_blocks}
%%%%%%%%%%%%%%%%%%%%%%%%%%%%%%%%%%%%%%%%%%%%%%

As discussed above, production and decay can largely be factorized in LLP searches~\footnote{Once again, we comment that non-factorization of production and decay due to LLP interaction with the detector material and non-trivial propagation effects arise in models with LLPs with electric or color charge, and we discuss these subtleties further in Section~\ref{sec:geant}}.
This allows us to specify the relevant production and decay modes for LLP models separately; we then put them together and map the space of models into the umbrella categories of motivated theories.

%%%%%%%%%%%%%%%%%%%%%%%%%%%%%%%%%%%
\subsection{Production Modes}
\label{SM:secProduction_modes}
%%%%%%%%%%%%%%%%%%%%%%%%%%%%%%%%%%%

Motivated by our over-arching UV frameworks, we can identify a minimal set of interesting production modes for LLPs. Schematic diagrams for each production mode are shown in Fig.~\ref{fig:feyndiagram}.
These production modes determine LLP signal rates both by relating the LLP production cross section to meaningful theory parameters such as gauge charges or Higgs couplings, and by determining the kinematic distribution of the LLP.
Additionally, a given production mechanism also makes clear predictions for the number and type of {\em prompt} objects accompanying the LLP(s).
These prompt AOs can be important for both triggering on events with LLPs and for background rejection, particularly when the LLP has a low mass or decays purely hadronically, and they can be either SM states (leptons, $\slashed{E}_{\rm T}$, tagging jets) or BSM objects such as $Z'$ or dark photons~\cite{Bai:2015nfa,Autran:2015mfa,Blinov:2017dtk}.


%%%%%%%%%%%%%%%%%%%%%%%%
\begin{figure}[t]
\centerline{\includegraphics[width=\textwidth]{figures/feyndiagram_row1.pdf}}
\vspace{0.8cm}
\centerline{\includegraphics[width=\textwidth]{figures/feyndiagram_row2.pdf}}
\vspace{0.8cm}
\centerline{\includegraphics[width=\textwidth]{figures/feyndiagram_row3.pdf}}
  \caption{Schematic illustrations of LLP production modes in our simplified model framework. From top to bottom and left to right:~direct pair production (DPP); heavy parent (HP); Higgs modes (HIG), including gluon fusion and VBF production (not shown here is $VH$ production); heavy resonance (RES); charged current (CC).}
  \label{fig:feyndiagram}
\end{figure}
%%%%%%%%%%%%%%%%%%%%%%%%%

%
\begin{itemize}

\item {\bf Direct-Pair Production (DPP):}~Here the LLP is dominantly pair-produced non-resonantly from SM initial states.
This is most straightforwardly obtained when the LLP is charged under a SM gauge interaction.  
In this case, an irreducible production cross section is then specified by the LLP gauge charge and mass.
Such continuum DPP can also occur in the presence of a (heavy, virtual) mediator (\emph{e.g.,} an initial quark$-$antiquark pair may exchange a virtual squark to pair produce bino-like neutralinos); in this case the production cross section is essentially a free parameter, as it is determined by the unknown heavy mediator masses and couplings.

\item {\bf Heavy parent (HP):}~The LLP is produced in the decays of on-shell heavy-parent particles that are themselves pair produced from the $pp$ initial state.
The production cross section is essentially a free parameter, and is indirectly specified by the gauge charges and masses of the heavy parent particles.
Heavy-parent production gives very different kinematics for the LLP than DPP, and  often produces additional prompt AOs in the rapid cascade decays of the parents.

\item {\bf Higgs (HIG):}~Here the LLP is produced through its couplings to the SM-like Higgs boson.
This case has an interesting interplay of possible production modes.
The dominant production is via gluon fusion, which features no AOs beyond gluon ISR.
Owing to its role in electroweak symmetry breaking, however, the Higgs has associated production modes (VBF, $VH$), each with its own characteristic features.
The best prospects for discovery are for LLP masses below $m_h/2$, in which case the LLPs can be in decays of the on-shell SM-like Higgs boson.
Higher-mass LLPs can still be produced via an off-shell Higgs, albeit at substantially lower rates \cite{Cui:2014twa,Craig:2014lda}.
The LLP can be pair produced or singly produced through the Higgs portal depending on the model; an LLP $X$ can also be produced in association with $\slashed{E}_{\rm T}$ via $h\rightarrow XX+\slashed{E}_{\rm T}$ or $h\rightarrow X+\slashed{E}_{\rm T}$.
The cross section (or, equivalently, the Higgs branching fraction into the LLP) is a free parameter of the model.
The Higgs mass can also be taken as a free parameter:~there exist many theories that predict new exotic scalar states (such as the singlet-scalar extension to the SM~\cite{Silveira:1985rk}), and these new scalars can be produced in the same manner as the SM Higgs.

\item {\bf Heavy resonance (RES):}~Here the LLP is produced in the decay of an on-shell resonance, such as a heavy $Z'$ gauge boson initiated by $q\bar{q}$ initial state.
Note that production via an off-shell resonance is kinematically similar to the DPP mode.
As with HIG, the LLP can be pair produced or singly produced (potentially in association with $\slashed{E}_{\rm T}$).
In RES models, ISR is the dominant source of prompt AOs.
Models with new heavy scalars could conceivably fall into either RES or HIG; the main determining factor according to our organizational scheme is whether the scalar possesses Higgs-like production modes such as VBF and $VH$.
Note that heavy resonance decays to SM particles also occur in these models, and searches for such resonances~\cite{Khachatryan:2016zqb,Sirunyan:2016iap,Aaboud:2017yvp,Sirunyan:2017dnz,Aaboud:2017buh,Aaboud:2018juj,Aaboud:2018zba} may complement the sensitivity for decays to LLPs.

\item {\bf Charged current (CC):}~In models with weak-scale right-handed neutrinos, the LLP can be produced in the leptonic decays of $W/W'$.
Single production is favored.
Prompt charged leptons from the charged-current interaction are typical prompt AOs.
\end{itemize}
%

It is important to note that each of the above production mechanisms has its own ``natural'' set of triggers to record the signal.
For example, HIG production can be accompanied by forward jets or leptons that are characteristic of VBF or $VH$ production.
Similarly, CC production often results in prompt charged leptons, while HP production comes with AOs from the heavy-parent decay.
However, the reader should be cautioned that this does not necessarily mean that the ``natural'' trigger is \emph{optimal} for a particular signal.
For example, the HIG modes suggest the use of VBF- or $VH$-based triggers, but if the LLP decays leptonically, it might be more efficient to trigger on the lepton from the LLP decay.
Thus, the final word on which trigger is most effective for a given simplified model depends on the production mode as well as the nature and kinematics of the LLP decay.
The prompt AOs of each production mode could still, however, be used to extend sensitivity to the model (see Sec.~\ref{sec:ch5-smsReinterpretations}).

We also comment that some models may span several production modes.
For example, a charged LLP that is part of an electroweak multiplet and nearly degenerate with a stable, neutral component~\cite{Chen:1995yu,Thomas:1998wy,Feng:1999fu,Cirelli:2005uq,Ibe:2006de,Cirelli:2009uv,FileviezPerez:2008bj,Buckley:2009kv,Mahbubani:2017gjh} gives both DPP signatures (via $pp\rightarrow \chi^+\chi^-$) and CC production (via associated production $pp\rightarrow\chi^\pm\chi^0$).
Comprehensive coverage of each of the above production modes will allow for a conservative determination of sensitivity for models that span many production modes.

%%%%%%%%%%%%%%%%%%%%%%%%%%
\subsection{Decay Modes}\label{sec:decmodes}
%%%%%%%%%%%%%%%%%%%%%%%%%%

We now list a characteristic set of LLP decay modes.
As we attempt to construct a minimal, manageable set of decay-mode building blocks, it is important to bear in mind that a given experimental search for LLPs can frequently be sensitive to a variety of possible LLP decay modes.
As a result, it is not always necessary to perform separate searches for each possible decay mode as might otherwise be needed for prompt signatures.

The fact that LLP searches can be sensitive to many LLP decay modes is, in part, because LLPs that decay far from the collision point offer fewer avenues for particle identification.
For example, for an LLP decaying inside of the calorimeter, most decay products are reconstructed as missing energy, or an energy deposition in the calorimeter.
Consequently, particle identification criteria are typically relaxed in comparison to requirements on searches without displaced objects.
Indeed, these ``loose'' collider objects can differ significantly from the corresponding ``tight'', prompt objects.
This leads to more inclusive analyses that can cover a wider range of signatures with a single search.

Additionally, backgrounds for LLP searches are often small; for a comprehensive discussion of backgrounds to LLP searches, see Chapter~\ref{sec:backgrounds}.
As a result, tight identification and/or reconstruction criteria typically found in exclusive prompt analyses are  no longer needed to suppress backgrounds.
For example, ATLAS has a displaced vertex search sensitive to di-lepton and multi-track vertices that is relatively inclusive with respect to other objects originating from near the displaced vertex~\cite{Aad:2015rba}.
Similarly, CMS has an analysis sensitive to events with one each of a high-impact-parameter muon and electron without reconstructing a vertex or any other objects~\cite{CMS-PAS-EXO-16-022}.
For these examples, the backgrounds are sufficiently low that other requirements may be relaxed and the specific decay mode of the LLP may not be too important so long as certain objects (such as muons) are present or the decay occurs in a specific location.
An even more extreme example in this regard is the search for highly-ionizing tracks sensitive to electrically and color-charged LLPs.
While the searches are primarily targeted to detector-stable particles (heavy stable charged particles or $R$-hadrons) they can also be used to probe intermediate lifetimes for which only a certain fraction of LLPs traverse the tracker before decaying (see \emph{e.g.}~\cite{Garny:2017rxs}).
Both because of low backgrounds as well as modified particle identification criteria compared to prompt searches, LLP searches can often be inclusive and therefore covered by a more limited range of simplified models.

In some cases, however, the topology of a decay does matter.
One potentially important factor that influences the sensitivity of a search to a particular model is whether the LLP decays into two SM objects vs.~three, because the kinematics of multi-body decay are distinct from two-body decay and this may affect the acceptance of particular search strategies.
An additional simplified model featuring a three-body decay of the LLP may consequently be needed to span the space of signatures.
  
Below, we describe an irreducible set of decay modes that can be used to characterize LLP signatures for various LLP charges (including neutral, electrically charged, and color charged).
For each, we also provide an explicit example for how the decay would appear in a particular UV model.
{\bf We emphasize that the following decay modes are  loosely defined with the understanding that their signatures are also representative of similar, related decay modes; for example $2j$ or $2j+\slashed{E}_{\rm T}$ can also be proxies for $3j$ because searches for multi-body hadronic LLP decays can be sensitive to both and typically do not require reconstruction of a third jet.}
It should also be noted that we are not recommending searches to be optimized to the exact, exclusive decay mode because that could suppress sensitivity to related but slightly more complicated LLP decays.

\begin{itemize}
\item {\bf Di-photon decays:}~The LLP can decay resonantly to $\gamma\gamma$ (like in Higgs-portal models or left$-$right symmetric models~\cite{Dev:2016vle}) or to $\gamma\gamma+\mathrm{invisible}$ (in DM models).
This latter mode stands as a proxy for other $\gamma\gamma+X$ decays where the third object is not explicitly reconstructed, although whether $X$ is truly invisible can influence the triggers used.
\emph{Example:~a singlino decaying to a singlet (which decays to $\gamma\gamma$) and a gravitino in Stealth SUSY~\cite{Fan:2011yu}.}

\item {\bf Single-photon decays:}~The LLP decays to $\gamma+\mathrm{invisible}$ (like in SUSY models).
The SUSY model mandates a near-massless invisible particle, while other models (such as DM theories~\cite{Weiner:2012cb,Primulando:2015lfa}) allow for a heavy invisible particle.
\emph{Example:~a bino decaying to photon plus gravitino in gauge-mediated models of SUSY breaking~\cite{Dimopoulos:1996yq}.}
%dark matter framework can come with a heavy final state particle.

\item {\bf Hadronic decays:}~The LLP can decay into two jets ($jj$) (like in Higgs and gauge-portal models, or RPV SUSY), $jj$ + invisible (SUSY, dark matter, or neutrino models), or $j$ + invisible (SUSY).
Here, ``jet'' ($j$) means either a light-quark parton, gluon, or $b$-quark.
This category also encompasses decays directly into hadrons (for example, LLP decay into $\pi^+$ plus an invisible particle~\cite{Chen:1995yu,Thomas:1998wy,Feng:1999fu}).
\emph{Example:~a scalar LLP decaying to $b\bar{b}$ due to mixing with the SM Higgs boson, as in models of neutral naturalness~\cite{Chacko:2005pe,Burdman:2006tz,Craig:2015pha}}.

\item {\bf Semi-leptonic decays:}~The LLP can decay into a lepton + 1 jet (such as in leptoquark models) or 2 jets (like in SUSY or neutrino models).
\emph{Example:~a right-handed neutrino decaying to a left-handed lepton and an on- or off-shell hadronically decaying $W$ boson (or $W'$ boson in a left$-$right symmetric model)~\cite{Keung:1983uu}. }

\item {\bf Leptonic decays:}~The LLP can decay into $\ell^+\ell^-(+\mathrm{invisible})$, or $\ell^\pm+\mathrm{invisible}$ (as in Higgs-portal, gauge-portal, SUSY, or neutrino models).
Here the symbol $\ell$ may be any flavor of charged lepton, but the decays are lepton flavor-universal and (for $\ell^+\ell^-$ decays) flavor-conserving.
\emph{Example:~a wino decaying to a neutralino and an on- or off-shell leptonic Z boson in SUSY~\cite{Barbier:2004ez}.}

\item {\bf Flavored leptonic decays:}~The LLP can decay into $\ell_\alpha+\mathrm{invisible}$, $\ell_\alpha^+\ell_\beta^-$ or $\ell_\alpha^+\ell_\beta^-+\mathrm{invisible}$ where flavors $\alpha\neq\beta$ (as in SUSY or neutrino models).
\emph{Example:~a neutralino decaying to two leptons and a neutrino in $R$-parity-violating SUSY~\cite{Barbier:2004ez}; or a right-handed neutrino decaying to two leptons and a neutrino~\cite{delAguila:2008cj}.}
\end{itemize}

In all cases, both the LLP mass and proper lifetime are free parameters.
Therefore, the case of detector-stable particles is automatically included by taking any of the above decay modes and taking the lifetime to infinity~\footnote{As mentioned earlier, in the $c\tau\rightarrow\infty$ limit the decay mode becomes irrelevant. However, an exception is the search for particles that are stopped inside the detector material and decay out of time, which are discussed in Sec.~\ref{sec:outoftime}.}.
We emphasize that, depending on the location of the LLP within the detector, these decay modes may or may not be individually distinguishable:~a displaced di-jet decay will look very different from a displaced di-photon decay in the tracker, but nearly identical if the decay occurs in the calorimeter.
The goal here is to identify  promising channels (as distinct from detector signatures). 

As an example of how the above-listed decay modes cover the most important experimental signatures, we consider a scenario of an LLP decaying to top quarks.
This scenario is very well motivated (for instance, with long-lived stops in SUSY) and might appear to merit its own decay category of an LLP decaying to one or more top quarks.
However, the top quark immediately decays to final states that \emph{are} covered in the above list, giving an effective semileptonic decay mode ($t\rightarrow b\ell^+\nu$) and a hadronic decay mode ($t\rightarrow bjj$) of the LLP.
Similarly, LLP decays to four or more final states are typically covered by the above inclusive definitions of decay modes; this provides motivation not to over-optimize experimental searches to the specific, exclusive features of a particular decay mode.

While it would be ideal to have separate experimental searches for each of the above decay modes (when distinguishable), it is rare for specific models to allow the LLP to decay in only one manner; as in the  example of an LLP decaying to a top quark, a number of decay modes typically occur with specific predictions for the branching fractions.
As another example, if the LLP couples to the SM via mixing with the SM Higgs boson, then the LLP decays via mass-proportional couplings giving rise to $b$- and $\tau$-rich signatures.
If, instead, the LLP decays through a kinetic mixing as in the case of dark photons or $Z$ bosons, then the LLP can decay to any particle charged under the weak interactions, giving rise to a relatively large leptonic branching fraction in addition to hadronic decay modes.
This allows some level of prioritization of decay modes based on motivated UV-complete models; for example, the Higgs-portal model prioritizes searches for heavy-flavor quarks and leptons in LLP decay, while the gauge-portal model prioritizes searches for electrons and muons in LLP decay.
Ultimately, however, it is desirable to retain independent sensitivity to each individual decay mode as much as possible.
Indeed, for each decay mode listed above, models exist for which the given decay mode would be the main discovery channel. 
\linebreak

\noindent {\bf Invisible Final-State Particles:}~Where invisible particles appear as products of LLP decays, additional model dependence arises from the unknown mass of the invisible particle.
The invisible particle could be a SM neutrino, DM, an LSP in SUSY, or another BSM particle.
The phenomenology depends strongly on the mass splitting, $\Delta \equiv M_{\rm LLP}-M_{\rm invisible}$.
If $\Delta \ll M_{\rm LLP}$ (\emph{i.e.,} $M_{\rm LLP}\sim M_{\rm invisible}$), the spectrum is compressed and the visible decay products of the LLP are soft.
This could, for instance, lead to signatures such as disappearing tracks or necessitate the use of ISR jets to trigger on the LLP signature.
If the mass splitting is large, $M_{\rm invisible}\ll M_{\rm LLP}$, then the signatures lose their dependence on the invisible particle mass.

We suggest three possible benchmarks:~a compressed spectrum with $\Delta \ll M_{\rm LLP}$ (example:~a nearly degenerate chargino-neutralino pair, giving rise to soft leptons or disappearing tracks~\cite{Chen:1995yu,Thomas:1998wy,Feng:1999fu,Cirelli:2005uq,Ibe:2006de,Cirelli:2009uv,FileviezPerez:2008bj,Buckley:2009kv,Mahbubani:2017gjh}); a massless invisible state, $\Delta = M_{\rm LLP}$ (example:~a next-to-lightest SUSY particle (NLSP) decaying to SM particles and a massless gravitino in gauge-mediated SUSY breaking~\cite{Dimopoulos:1996vz,Ambrosanio:1997rv,Delgado:2007rz,Meade:2010ji,Allanach:2015cia,Evans:2016zau,Allanach:2016pam}); and an intermediate splitting corresponding to a democratic mass hierarchy, $\Delta \approx M_{\rm LLP}/2$ (example:~NLSPs in mini-split SUSY~\cite{Arvanitaki:2012ps,ArkaniHamed:2012gw,Liu:2015bma}).

%%%%%%%%%%%%%%%%%%%%%%%%%%%%%%%%%%%%
\section{A Simplified Model Proposal}\label{sec:proposal}
%%%%%%%%%%%%%%%%%%%%%%%%%%%%%%%%%%%%

In this section, we present a compact set of simplified model channels that, broadly speaking, covers the space of theoretical models in order to motivate new experimental searches.
Such a minimal, compact set may not be optimal for reinterpretation of results (where variations on our listed production and decay modes may influence signal efficiencies and cross-section sensitivities), but rather provides a convenient characterization of possible signals to ensure that no major discovery mode is missed.
These models may therefore serve as a starting point for systematically understanding experimental coverage of LLP signatures and devising new searches, but may need to be extended in future for the purposes of facilitating reinterpretation.
We undertake an in-depth discussion of these topics in Section~\ref{sec:reint}.

We classify LLPs according to their SM gauge charges, as these dictate the dominant or allowed LLP production and decay modes, and can give rise to different signatures (for example, disappearing tracks and hadronized LLPs).
We separately consider LLPs that are:~(a)~neutral; (b)~electrically charged but color neutral; and (c)~color charged.
In the latter case, it is important to distinguish between the long-lived parton (which carries a charge under quantum chromodynamics, QCD) that hadronizes prior to decay, and the physical LLP, which is a color-singlet ``$R$-hadron'' (using the standard nomenclature inspired by SUSY).
The decays of the $R$-hadron are still dominated by the parton-level processes.

All of the following models have the LLP mass and lifetime as free parameters.
For heavy-parent (HP) production, the parent mass is an additional parameter, while for invisible decays, several different benchmarks for mass splittings between LLP and invisible final state may have to be separately considered as described in Section~\ref{sec:decmodes}.
The cross section may have a theoretically well-motivated target value depending on UV-model parameters, but phenomenologically can generally be taken as a free parameter.

We emphasize that in spite of the many simplified model channels proposed below, a small number of experimental LLP searches can have excellent coverage over a wide range of channels (at least for certain lifetime ranges).
The list is intended to be comprehensive in order to identify whether there are new searches that could have a similarly high impact on the space of simplified models, and identify where the gaps in coverage are.

%%%%%%
\subsection{Neutral LLPs}\label{sec:proposal_neutralLLP}
%%%%%%

The simplified model channels for neutral LLPs are shown in Table~\ref{tab:neutral_LLP}, where $X$ indicates the LLP.

In our initial proposal, which is the first iteration of the simplified model framework, it is sufficient to consider as ``jets'' all of the following:~$j=u,d,s,c,b,g$.
It is worth commenting that $b$-quarks pose unique challenges and opportunities.
Since $b$-quarks are themselves LLPs, they appear with an additional displacement relative to the LLP decay location.
They also often give rise to soft muons in their decays, which could in principle lead to additional trigger or selection possibilities.
However, these subtleties can be addressed in further refinements of the simplified models; we discuss this further in Section~\ref{sec:simplified_future}.
Similarly, we consider $e$, $\mu$, and $\tau$ to be included in the broad category of ``leptons'', with the proviso that
searches should be designed where possible with sensitivity to each.
 
When multiple production modes are specified in one row of the table, this means that multiple especially well-motivated production channels give rise to similar signatures.
Typically only one of these simplified model production modes will actually need to be included when developing and assessing sensitivity of an experimental search, but we sometimes include multiple different production modes as individuals may variously prefer one over the other.

In each entry of the table, we indicate which umbrella category of well-motivated UV models (Section~\ref{sec:motivated_theories}) can predict a particular $(\mathrm{production})\times(\mathrm{decay})$ mode.
An asterisk (*) on the umbrella model indicates that $\slashed{E}_{\rm T}$ is required in the decay.
A dagger (${}^\dagger$) indicates that this particle production $\times$ decay scenario is not present in the \emph{simplest and most minimal} implementations of the umbrella model, but could be present in extensions of the minimal models.
While the HIG production signatures are best-motivated for the SM-like 125 GeV Higgs, exotic Higgses of other masses can still have the same production modes and so $m_H$ can be taken as a free parameter.
%
\begin{table}[t]
\begin{center}
\begin{tabular}{ |c|c|c|c|c|c|c| } 
 \hline
\backslashbox{Production}{Decay} & $\gamma\gamma(+\mathrm{inv.})$ & $\gamma+\mathrm{inv.}$ & $jj(+\mathrm{inv.})$ & $jj\ell$ & $\ell^+\ell^-(+\mathrm{inv.})$ & $\ell_\alpha^+\ell_{\beta\neq\alpha}^-(+\mathrm{inv.})$\\
 \hline\hline
 DPP:~sneutrino pair & ${}^\dagger$ & SUSY & SUSY & SUSY & SUSY & SUSY\\
 \hline
 HP:~squark pair, $\tilde{q}\rightarrow jX$ & $ {}^\dagger$  & SUSY & SUSY & SUSY & SUSY & SUSY\\
 or gluino pair $\tilde g\rightarrow jjX$ &&&&&&\\
 \hline
 HP:~slepton pair, $\tilde{\ell}\rightarrow\ell X$ & ${}^\dagger$ & SUSY & SUSY & SUSY & SUSY & SUSY\\
 or chargino pair, $\tilde{\chi}\rightarrow WX$ &&&&&&\\
 \hline 
% HIG:~$h(h')\rightarrow XX$ & Higgs (DM)  &  & Higgs (DM) &  & Higgs (DM) & \\
 HIG:~$h\rightarrow XX$ & Higgs, DM*  & ${}^\dagger$ & Higgs, DM* & RH$\nu$ & Higgs, DM* &RH$\nu$* \\
  or $\rightarrow XX+\mathrm{inv.}$ &&&&& RH$\nu$* &\\
 \hline 
 %HIG:~$h(h')\rightarrow X+\mathrm{inv.}$ & DM  &  & DM &  & DM & \\
 HIG:~$h\rightarrow X+\mathrm{inv.}$ & DM*, RH$\nu$  & ${}^\dagger$ & DM* & RH$\nu$ & DM* &${}^\dagger$ \\
 \hline
 RES:~$Z(Z')\rightarrow XX$ & $Z'$, DM*  & ${}^\dagger$ & $Z'$, DM* & RH$\nu$ & $Z'$, DM* &$ {}^\dagger$\\
 or $\rightarrow XX+\mathrm{inv.}$ &&&&&&\\
 \hline 
 RES:~$Z(Z')\rightarrow X+\mathrm{inv.}$ & DM  & ${}^\dagger$ & DM &  RH$\nu$ & DM & ${}^\dagger$\\
 \hline
  CC:~$W(W')\rightarrow \ell X$ & ${}^\dagger$  & ${}^\dagger$ & RH$\nu$* & RH$\nu$ & RH$\nu$* & RH$\nu$* \\
 \hline
\end{tabular}
%
\end{center}
\caption{{\bf Simplified model channels for neutral LLPs.} The LLP is indicated by $X$.
Each row shows a separate production mode and each column shows a separate possible decay mode, and therefore every cell in the table corresponds to a different simplified model channel of (production)$\times$(decay).
We have cross-referenced the UV models from Section~\ref{sec:motivated_theories} with cells in the table to show how the most common signatures of complete models populate the simplified model space.
The asterisk (*) shows that the model definitively predicts missing energy in the LLP decay.
A dagger (${}^\dagger$) indicates that this particle production $\times$ decay scenario is not present in the \emph{simplest and most minimal} implementations of the umbrella model, but could be present in extensions of the minimal models.
When two production modes are provided (with an ``or''), either simplified model can be used to simulate the same simplified model channel.}\label{tab:neutral_LLP}
\end{table}

We remind the reader that the production modes listed in Table~\ref{tab:neutral_LLP} encompass also the associated production of characteristic prompt objects.
For example, the Higgs production modes not only proceed through gluon fusion, but also through VBF and $VH$ production, each of which results in associated prompt objects such as forward jets in VBF, and leptons or $\slashed{E}_{\rm T}$ in $VH$.
All of the production modes listed in Table~\ref{tab:neutral_LLP} could be accompanied by ISR jets that aid in triggering or identifying signal events.
It is therefore important that searches are designed to exploit such prompt AOs whenever they can improve signal sensitivity, especially with regard to triggering.

To demonstrate how to map full models onto the list of simplified models (and vice-versa), we consider a few concrete cases.
For instance, if we consider a model of neutral naturalness where $X$ is a long-lived scalar that decays via Higgs mixing (for instance, $X$ could be the lightest quasi-stable glueball), then the process where the SM Higgs $h$ decays via  $h\rightarrow XX$, $X\rightarrow b\bar{b}$ would be covered with the HIG production mechanism and a di-jet
decay.
Entirely unrelated models, such as the case where $X$ is a bino-like neutralino with RPV decays $h\rightarrow XX$, $X\rightarrow j jj $ could be covered with the same simplified model because most hadronic LLP searches do not have exclusive requirements on jet multiplicity.
Similarly, a hidden-sector model with a dark photon, $A'$, produced in $h\rightarrow A'A'$, $A'\rightarrow f\bar{f}$ would also give rise to the di-jet signature when $f$ is a quark, whereas it would populate the $\ell^+\ell^-$
column if $f$ is a lepton.
Finally, a scenario with multiple hidden-sector states $X_1$ and $X_2$, in which $X_2$ is an LLP and $X_1$ is a stable, invisible particle, could give rise to signatures like $h\rightarrow X_2 X_2$, $X_2\rightarrow X_1jj$ that would be covered by the same HIG production, hadronic-decay simplified model; however, we see how  $\slashed{E}_{\rm T}$ can easily appear in the final state, and that the LLP decay products may not be entirely hadronic.
Therefore, the simplified models in Table~\ref{tab:neutral_LLP} can cover an incredibly broad range of signatures, but only if searches are not overly optimized to particular features such as $\slashed{E}_{\rm T}$ and LLPs decaying entirely visibly (which would allow reconstruction of the LLP mass)~\footnote{This should not, of course, be interpreted as saying that searches shouldn't be done that exploit these features.
Instead, our position is that experiments should bear in mind the range of topologies and models covered by each cell in Table~\ref{tab:neutral_LLP} when designing searches, and that some more inclusive signal regions should be established where possible.}.

\subsection{Electrically Charged LLPs:~$|Q|=1$}\label{sec:EMcharge}

For an electrically charged LLP, we need to consider far fewer production modes because of the irreducible gauge production associated with the electric charge.
We still consider the additional possibility of a HP scenario where the parent has a QCD charge, as this could potentially dominate the production cross section, see \emph{e.g.}, Ref.~\cite{Heisig:2012zq}.
We summarize our proposals in Table~\ref{tab:charged_LLP}.

Note that we group all resonant production into the $Z'$ simplified model.
The reason is that the SM Higgs cannot decay into two on-shell charged particles due to the model-independent limits from LEP on charged particles masses, $M\gtrsim$75$-$90 GeV (see, for example, Ref.~\cite{Abbiendi:2003yd}); because of this lower limit on the LLP mass, it is less important to use AOs for triggering and reconstructing charged LLP signatures than for neutral LLPs.
Additionally, there are fewer allowed decay modes because of the requirement of charge conservation. 

For concreteness, we recommend using $|Q|=1$ as a benchmark for charged LLPs for the purpose of determining allowed decay modes.
Although other values of $Q$ are possible, these often result in cosmologically stable charged relics or necessitate different decay modes than those listed here.
Additionally, LLPs with $|Q|=1$ are motivated within SUSY~\cite{Chen:1995yu,Thomas:1998wy,Feng:1999fu} and within Type-III seesaw models of neutrino masses~\cite{Bajc:2006ia,Bajc:2007zf,Franceschini:2008pz,Arhrib:2009mz}.
We note that there exist already dedicated searches for heavy quasi-stable charged particles with non-standard charges~\cite{Aad:2015kta,Khachatryan:2016sfv}.
Because such searches are by construction not intended to be sensitive to the decays of the LLP, the existing models are sufficient for characterizing these signatures and they do not need to be additionally included in our framework.

For massive particles with $|Q|=1$ with intermediate or large lifetimes such that the LLP traverses a significant part (or all) of the tracker, the highly ionizing track of the LLP provides a prominent signature.
This can be exploited for an efficient suppression of backgrounds while keeping identification and/or reconstruction criteria as loose and, hence, as inclusive as possible.
In particular, for decay-lengths of the order of or larger than the detector size, the signature of highly ionizing tracks and anomalous time-of-flight (\emph{i.e.,}~searches for heavy stable charged particles; see Sections~\ref{subsec:funnytracks} and~\ref{sec:ch5-HSCPs}) constitute an important search strategy covering a large range of lifetimes present in the parameter space of theoretically motivated models.
While the searches for heavy stable charged particles are largely inclusive with respect to additional objects in the event, they depend strongly on the velocity of the LLP\@.
For $\beta\to1$ one loses the discriminating power against minimally ionizing particles, while for small velocities, $\beta\lesssim0.5$, the reconstruction becomes increasingly difficult due to timing issues.
It is therefore important to include the heavy parent production scenario which covers a much larger kinematic range than direct production alone and which may feature a much wider range of signal efficiencies than the DPP scenario~\cite{Heisig:2015yla}.

\begin{table}[t]
\begin{center}
\begin{tabular}{ |c|c|c|c|c|} 
 \hline
\backslashbox{Production}{Decay} & $\ell+\mathrm{inv.}$ &  $jj(+\mathrm{inv.})$ & $jj\ell$ & $\ell\gamma$ \\
\hline\hline
DPP:~chargino pair & SUSY & SUSY & SUSY & ${}^\dagger$ \\
or slepton pair & DM* & DM* & &\\
\hline
HP:~$\tilde{q}\rightarrow j X$ & SUSY & SUSY & SUSY &${}^\dagger$ \\
& DM* & DM* & &\\
\hline
RES:~$Z'\rightarrow XX$ & Z', DM*& Z', DM* & Z'  &${}^\dagger$ \\
\hline
CC:~$W'\rightarrow X+\mathrm{inv.}$ & DM* & DM* & RH$\nu$ &${}^\dagger$\\
\hline
\end{tabular}
\end{center}
\caption{{\bf Simplified model channels for electrically charged LLPs} such that $|Q| = 1$.
The LLP is indicated by $X$.
Each row shows a separate production mode and each column shows a separate possible decay mode, and therefore every cell in the table corresponds to a different simplified model channel of (production)$\times$(decay).
We have cross-referenced the UV models from Section~\ref{sec:motivated_theories} with cells in the table to show how the most common signatures of complete models populate the simplified model space.
The asterisk (*) shows that the model definitively predicts missing energy in the LLP decay.
A dagger (${}^\dagger$) indicates that this particle production $\times$ decay scenario is not present in the \emph{simplest and most minimal} implementations of the umbrella model, but could be present in extensions of the minimal models.
When two production modes are provided (with an ``or''), both production simplified models can be used to cover the same experimental signatures.}\label{tab:charged_LLP}
\end{table}

While the signatures in Table~\ref{tab:charged_LLP} form a minimal set, they also encompass some scenarios that merit special comment.
One of these is the disappearing track signature~\cite{Chen:1995yu,Thomas:1998wy,Feng:1999fu,Cirelli:2005uq,Ibe:2006de,Cirelli:2009uv,FileviezPerez:2008bj,Buckley:2009kv,Mahbubani:2017gjh}, in which a charged LLP decays to a nearly degenerate neutral particle.
The lifetime is long in this scenario due to the tiny mass splitting between the two states.
Formally, these are included in the chargino or slepton DPP modes in Table~\ref{tab:charged_LLP} with decays to $\ell+\mathrm{inv.}$ or $q\bar{q}'+\mathrm{inv.}$ taken in the limit where the splitting between the charged LLP and the invisible final state is of $\mathcal{O}(200\,\,\mathrm{MeV})$.
In the case of a hadronic decay, $X$ decays to a soft pion that is very challenging to reconstruct and so the track simply disappears.
This is an important scenario that is already the topic of existing searches~\cite{CMS:2014gxa,Aaboud:2017mpt}.
As the degeneracy between the charged LLP and the neutral state is relaxed, other signatures are possible; this parameter range is well motivated both by SUSY and DM models with coannihilation~\cite{Griest:1990kh,Baker:2015qna,Khoze:2017ixx}.

Finally, we comment on the challenges of simulating the charged LLP simplified models.
Because the LLP bends and interacts with detector material prior to its decay, the simulation of the LLP propagation is important in correctly modeling the experimental signature.
The subsequent decay of the LLP must either be hard-coded into the detector simulation, or allow for an interface with programs such as Pythia 8 to implement the decays.
We discuss the challenges of simulating signals for LLPs with electric or color charge in Section~\ref{sec:geant}.

\subsection{LLPs with Color Charge}
\label{sec:coloredLLPs}

An LLP charged under QCD is more constrained than even electrically charged LLPs.
Because of the non-Abelian nature of the strong interactions, the gauge pair-production cross section of the LLP is specified by the LLP mass and its representation under the color group, $\mathrm{SU}(3)_{\rm C}$.
We do not consider LLP production via a heavy parent particle  because that cross section is unlikely to dominate the total production rate at the LHC relative to DPP. The simplified model channels are provided in Table~\ref{tab:color_LLP}.

\begin{table}[t]
\begin{center}
\begin{tabular}{ |c|c|c|c|c|}
 \hline
\backslashbox{Production}{Decay} & $j+\mathrm{inv.}$ &  $jj(+\mathrm{inv.})$ & $j\ell$ & $j\gamma$ \\
\hline\hline
DPP:~squark pair & SUSY & SUSY & SUSY &${}^\dagger$ \\
or gluino pair & & & &\\
\hline
\end{tabular}
\end{center}
\caption{{\bf Simplified model channels for LLPs with color charge.} The LLP is indicated by $X$.
Each row shows a separate production mode and each column shows a separate possible decay mode, and therefore every cell in the table corresponds to a different simplified model channel of (production)$\times$(decay).
We have cross-referenced the UV models from Section~\ref{sec:motivated_theories} with cells in the table to show how the most common signatures of complete models populate the simplified model space.
A dagger (${}^\dagger$) indicates that this particle production $\times$ decay scenario is not present in the \emph{simplest and most minimal} implementations of the umbrella model, but could be present in extensions of the minimal models.
When two production modes are provided (with an ``or''), both production simplified models can be used to cover the same experimental signatures.}\label{tab:color_LLP}
\end{table}

A complication of the QCD-charged LLP is that the LLP hadronizes prior to its decay, forming an $R$-hadron bound state.
The modeling of hadronization and subsequent propagation is directly related to many properties of the long-lived parton, such as electric charge, flavor, and spin. Event generators such as Pythia 8 have routines~\cite{Sjostrand:2007gs,Sjostrand:2014zea} to simulate LLP hadronization, although it is unclear how precise these predictions are.
For a point of comparison, using the default settings of Pythia 8 yields an estimate of the neutral $R$-hadron fraction from a gluino (color-octet fermion, $\tilde g$) of approximately 54\%, while the neutral $R$-hadron fraction for a stop (scalar top partner) is estimated to be 44\%~\cite{Liu:2015bma}. After hadronization, the charge of the $R$-hadron may change as it passes through the detector. For instance, some estimates~\cite{Buccella:1985cs,Farrar:2010ps} suggest that heavy, color-octet gluinos $\tilde g$ would predominantly form mesons (\emph{e.g.,}  $(u \tilde g \bar d)$) at first.
They eventually drop to the lower-energy neutral singlet baryon $\tilde \Lambda = (\tilde g u d s)$ state when interacting with the protons and neutrons within the calorimeters.

The modeling of LLP hadronization and propagation is crucial to designing searches for color-charged LLPs and assessing their sensitivity. For example,  only the charged $R$-hadrons can be found in heavy stable charged particle search; if the LLP charge changes as it passes through the detector, heavy stable charged particle searches may have limited sensitivity. To take this into account, the experimental searches  include both tracker-only or tracker$+$calorimeter signal regions~\cite{Aaboud:2016uth,CMS:2016ybj}, which enhances sensitivity to the scenario in which $R$-hadrons lose their charge by the time they reach the calorimeters. 

Because no $R$-hadrons have been discovered to date and hence their properties cannot be directly measured, $R$-hadron modeling in detector simulations is challenging. We discuss  the challenges of simulating the propagation and decays of  LLPs with color charge in Sec.~\ref{sec:geant}.




%[\textcolor{red}{JB: Fix this whole section, particularly the next couple of paragraphs.}]




\section{Proposal for a Simplified Model Library}\label{sec:library}

The simplified models outlined in the above sections provide a common language for theorists and experimentalists to study the sensitivity of existing searches, propose new search ideas, and interpret results in terms of UV models.
Each of these activities demands a simple framework for the simulation of signal events that can be used to evaluate signal efficiencies of different search strategies and map these back onto model parameters.
Requiring individual users to create their own MC models for each simplified model is impractical, redundant, and invites the introduction of errors into the analysis process.

In this section, we propose and provide a draft version of a \emph{simplified model library} consisting of model files and MC generator cards that can be used to generate events for various simplified models in a straightforward fashion.
Because each experiment uses slightly different MC generators and settings, this allows each collaboration (as well as theorists) to generate events for each simplified model based on the provided files.
Depending on how the LLP program expands and develops over the next few years, it may become expedient to expand the simplified model library to include sets of events in a standard format (such as the Les Houches format~\cite{Alwall:2006yp}) that can be directly fed into event-generator and detector-simulation programs.
Given the factorization of production and decay of LLPs that is valid for neutral LLPs, this could involve two mini-libraries:~a set of production events for LLPs and a set of decay events for LLPs, along with a protocol for ``stitching'' the events together.

The current version of the library can be found here:~{\bf [provide link]}.
In Appendix~\ref{sec:library_more}, we also provide tables that list how to simulate each LLP simplified model channel with one of the specified base models.
These proposals are based on the models outlined in Section~\ref{sec:base} and often match the best-motivated simplified models from Section~\ref{sec:proposal}, and also building on the DM-inspired LLP simplified models proposed and detailed in Ref.~\cite{Buchmueller:2017uqu}.
The library currently focuses on models of neutral LLPs; simulating the propagation of charged LLPs along with the full range of decays listed in Sections~\ref{sec:EMcharge}$-$\ref{sec:coloredLLPs} requires more careful collaboration with detector simulation and other MC programs to ensure that they can practically be used in experimental studies.

We provide model files in the popular Universal Feynrules Output (UFO) format~\cite{Degrande:2011ua}, which is designed to interface easily with parton-level simulation programs such as MadGraph5\_aMC@NLO~\cite{Alwall:2014hca}.
The goal is to cover as many of the simplified models of Section~\ref{sec:proposal} with as few UFO models as possible; this limits the amount of upkeep needed to maintain the library and develops familiarity with the few UFO models needed to simulate the LLP simplified models.
We provide specific instructions for how to simulate each simplified LLP channel along with the UFO models. 

\subsection{Base Models for Library}\label{sec:base}

In order to reproduce the simplified model channels of Section~\ref{sec:building_blocks}, we need a collection of models that:
%
\begin{itemize}
\item Includes additional gauge bosons and scalars to allow vector- and scalar-portal production of LLPs (RES and HIG);
\item Includes new gauge-charged fermions and scalars to cover direct and simple cascade production modes of LLPs (DPP and HP); 
\item Includes a RHN-like state with couplings to SM neutrinos and leptons (CC);
\item \emph{Recommended, but optional:}~Allows for the decays of the LLP particle through all of the decay modes listed in Section~\ref{sec:building_blocks}, either through renormalizable or higher-dimensional couplings.
If couplings that allow LLP decay are included in the UFO model, then the decays can be performed directly at the matrix-element level in programs such as MadGraph5\_aMC@NLO~\cite{Alwall:2014hca} and accompanying packages such as MadSpin~\cite{Artoisenet:2012st}.
If the couplings needed for LLP are not in the UFO model, then LLPs can be left stable at the matrix-element level and decays implemented via Pythia 8~\cite{Sjostrand:2007gs,Sjostrand:2014zea}, which allows for the straightforward implementation of decays according to a phase-space model, but does not correctly model the angular distribution of decay products.
Instructions for implementing decays in Pythia are included with the model library files.
\end{itemize}

Fortunately, an extensive set of UFO models is already available for simulating the production of BSM particles.
We note that extensions or generalizations of only three already-available UFO models are needed at the present time; the SUSY models
in particular can cover many of the simplified models since they contain an enormous collection of new fermions and scalars.
We also provide an optional
fourth model, the Hidden Abelian Higgs Model, that can be helpful to simulate HIG and ZP theories.

\begin{enumerate}

\item {\bf The Minimally Supersymmetric SM (MSSM):}~The use of this model is motivated by and allows for the simulation of SUSY-like theories.
The model contains a whole host of new particles with various gauge charges and spins.
Therefore, an MSSM-based model allows for the simulation of many of the simplified model channels.
In particular, we note that existing UFO variants of the MSSM that include gauge-mediated supersymmetry breaking (GMSB) couplings (including decays to light gravitinos), $R$-parity violation (to allow for the decay of otherwise stable LSPs), and the phenomenological MSSM (pMSSM)~\cite{Djouadi:1998di,Berger:2008cq} already cover most of the SUSY-motivated LLP scenarios. In some cases, the model is modified to give direct couplings between the Higgs states and gluons/photons.



\item {\bf The Left$-$Right Symmetric Model (LRSM):}~This UFO model is best for simulating UV theories with right-handed neutrinos (RH$\nu$).
The UFO model supplements the SM by an additional $\mathrm{SU}(2)_{\rm R}$ symmetry, which gives additional charged and neutral gauge bosons.
The model is available in the simplified models library and contains a right-handed neutrino which is the typical LLP candidate, which can be produced via SM $W$, $Z$, or the new gauge bosons~\footnote{Additional LRSM tools are available at \url{https://sites.google.com/site/leftrighthep/}.}. 

\item {\bf Dark-Matter Simplified Models (DMSM)}:~These UFO models are best for simulating UV theories in the DM class.
These UFO models have been created by the LHC DM working group~\cite{Abdallah:2015ter}.
They typically consist of a new BSM mediator particle (such as a scalar of a $Z'$) coupled to invisible DM particles.
The UFO models can either be modified to include an unstable LLP, or else the otherwise stable ``DM'' particle can be decayed via Pythia.
The utility and applicability of the DM simplified model framework to LLPs has already been demonstrated with a detailed proposal and study of classes of DM simplified models for LLPs~\cite{Buchmueller:2017uqu}.
These models are particularly good for simulating LLP production via a heavy resonance (RES), and can also simulate continuum production of LLPs in the limit where the mediator is taken to be light and off-shell (DPP).

\item{\bf (optional) The Hidden Abelian Higgs Model (HAHM):}~This UFO model contains new scalars and gauge bosons and so can be used to simulate both Higgs-portal and gauge-portal (ZP) theories.
The model consists of the SM supplemented by a ``hidden sector'' consisting of a new $\mathrm{U}(1)$ gauge boson and a corresponding Higgs field.
The physical gauge and Higgs bosons couple to the SM via kinetic and mass mixing, respectively.
The HAHM allows for straightforward simulation of Higgs-portal production of LLPs, as well as $Z'$ models and many hidden sector scenarios.
The UFO implementation is from Ref.~\cite{Curtin:2014cca}.

\end{enumerate}
%
If additional decay modes are needed beyond those in the specified simplified models, then the library can be updated to include the new couplings mediating the decay.
Alternatively, the LLPs can be left stable at parton level and decayed in event generators such as Pythia.


A detailed list of processes that can be used to simulate each simplified model channel is provided in Appendix~\ref{sec:library_more}.
The primary purpose of the library is to be used to simulate events for determining acceptances, and, as a result, the signal cross section is not important.
Thus, for example, SM gauge interactions can be used as proxies for much weaker exotic interactions.
Similarly, the spins of the particles are generally of subdominant importance:~replacing the direct production of a fermion with the direct production of a scalar will not fundamentally alter the signature.
As long as results are expressed in terms of sensitivity to cross sections and not couplings, the results can be qualitatively (and in many cases, quantitatively) applied to any similar production mode regardless of spin.
However, we caution the reader that changing the spin of the LLP (or its parent) can change the angular distribution, and since in some cases LLP searches are typically more sensitive to aspects of event geometry than prompt searches, the second-order effects of spin could have more of an effect than for prompt simplified models.

\subsection{LLP Propagation and Interaction with Detector Material}\label{sec:geant}

Long-lived particles with electric or QCD charges interact with the detector material prior to decay, and their propagation through the detector must be correctly modeled.
The propagation of both LLPs with color charge (in the form of $R$-hadrons) and electrically charged LLPs can be implemented in the Geant4 (G4) toolkit~\cite{Agostinelli:2002hh}.
For example, routines exist to simulate the propagation of color-charged LLPs~\cite{Mackeprang:2006gx,Mackeprang:2009ad}.
G4 also includes routines that can implement $N$-body decays of LLPs using a phase-space model.
This works fine for decays of LLPs to leptons, photons, invisible particles such as neutrinos, as well as exclusive hadronic decays. 

However, G4 cannot implement decays to partons that subsequently shower and hadronize.
One solution to this limitation is employed by CMS~\cite{Khachatryan:2015jha,Sirunyan:2017sbs} and ATLAS~\cite{Aad:2013gva} in their searches for stopped LLPs.
In these analyses, the signal simulation proceeds in two stages.
During the first stage, the production of the LLP and its subsequent interactions with the detector are simulated.
Once the stopping point of the LLP is determined, a new event is simulated including the LLP decay; the LLP decay products are then manually moved to the stopping point from the first stage.
G4 is then run a second time to determine the efficiency for reconstructing the LLP decay signal.

It would be preferable to fully automate the simulation of decays of charged LLPs after propagation in G4.
There exists in G4 a class called G4ExtDecayer, which can be used to implement decays by interfacing with an external generator.
This class has been used to interface G4 with Pythia 6~\footnote{See \url{http://geant4-userdoc.web.cern.ch/geant4-userdoc/Doxygen/examples_doc/html/Exampledecayer6.html}}.
The interface with Pythia 6 has been used most recently to model LLP gluino propagation and decay in a search for displaced vertices and missing energy in ATLAS~\cite{ATLAS2017a}.
Work is ongoing to extend this functionality to Pythia 8 and to simplify the interface.

An additional challenge of simulating LLP decays is that, if the LLP undergoes a multi-body decay, generators such as Pythia use a phase-space model to implement the decays. If more accuracy is required, it may be preferable to use the full matrix element via generators such as MadGraph5~\cite{Alwall:2011uj,Alwall:2014hca}. If the matrix element is important for computing the decay of the LLP, then either an interface with MadGraph is needed to implement the decay prior to passing the vertex back to Pythia 8 for showering and hadronization, or matrix-element-based methods within the event generator itself must be used.

Because of the need to interface with G4 in simulating the decays of LLPs with electric or color charges, we do not at this point include such decay modes in our simplified model library.
The decays of such LLPs will be most easily simulated via an interface with Pythia 8 once it is finalized.

Finally, we comment that LLPs can have even stranger propagation properties than  LLPs with electric or color charges.
For example, quirks are LLPs that are charged under a hidden-sector gauge interaction that confines at macroscopic scales \cite{Kang:2008ea}.
Because the confinement scale can be just about any distance, quirks can have very unusual properties; as a specific example, if electrically charged quirk-antiquirk pairs are bound on the millimeter or centimeter level, they behave as an electric dipole and therefore do not leave conventional tracks that bend in the magnetic field.
Other confinement scales give rise to different behaviors, such as meta-stable heavy charged particles and non-helical tracks~\cite{Farina:2017cts,Knapen:2017kly}.
In scenarios where the quirks carry color charge, the quirks hadronize and can undergo charge-flipping interactions as they move through the detector.
These quirk scenarios can be challenging to model, and no public code exists that allows for the propagation and interaction of quirks with the detector material; we encourage the collaborations to validate and release any internal software they may have to study the propagation of quirks (for more discussion, see the discussion of quirks  in Section~\ref{subsec:funnytracks})~\footnote{Ideally, this software would be well-documented to facilitate sharing between experiments.
A successful example of readily shareable software between experiments is the G4 package for $R$-hadrons and other particles' interaction with matter, found at~\url{http://r-hadrons.web.cern.ch/r-hadrons/}}.




\section{Limitations of Simplified Models \& Future Opportunities}\label{sec:simplified_future}

We conclude our discussion of simplified models with a more extensive discussion of the limitations of the current simplified model proposal in its application to models of various types, along with opportunities for future development.
The presented framework is only the first step of a simplified model program that is comprehensive in terms of generating LHC signatures and  allowing straightforward reinterpretation of experimental results for UV models.
The framework we have developed with separate, modular components for LLP production and decay is amenable to expansion, and we encourage members of the theory and experimental communities to continue to do so over the coming years to ensure maximal utility of the simplified models framework.

One significant simplification we have undertaken in our framework is to define a ``jet'' as any of $j=u,c,d,s,b,g$.
In reality, different partons give rise to different signatures, especially when one of the ``jets'' is a heavy-flavor quark.
Jets initiated by $b$ and $c$ quarks have some useful distinguishing features, such as the fact that the underlying heavy-flavor meson decays at a distance slightly displaced from the proton interaction vertex and that there are often associated soft leptons resulting from meson decays.
In particular, it is possible that the soft muons associated with $B$-meson decays could be used to enhance trigger and reconstruction prospects for LLPs decaying to $b$-jets~\cite{Aad:2013txa}.
However, heavy quarks also constitute an important backgrounds for LLP searches, and so LLPs decaying to $b$- and $c$-jets may necessitate dedicated treatment in future.
Similarly, LLP decays to $\tau$ leptons may merit further specialized studies.

Another property of the current framework is that it is restricted to LLP signatures of low multiplicity.
By ``low multiplicity'', we mean collider signatures with one or two LLPs.
Searches inspired by these models are also suitable for many scenarios with three or four LLPs per event (which include models with dark-Higgs decays into lepton-jets~\cite{Falkowski:2010cm}, or left$-$right symmetric models~\cite{Nemevsek:2016enw}), since the LLP signatures are generally extremely rare and so only one or two typically need to be identified in a given event to greatly suppress backgrounds.
Thus, as long as the search is inclusive with respect to possible additional displaced objects, the signature can be covered with low-multiplicity strategies.
As the LLP multiplicity grows, however, the simplified model space we have presented requires modification.
This is both because the individual LLPs grow softer, making them harder to reconstruct on an individual level, and they become less separated in the detector, which makes isolation and identification of signal a challenge.
On the other hand, the high LLP multiplicity may allow for new handles for further rejecting backgrounds, and the kinematics can vary widely based on the model (for example, in some ``quirky'' scenarios, LLPs can be produced in a variety of ways with different kinematic distributions~\cite{Beauchesne:2017ukw}). 
In extreme cases, signals can even mimic pile-up~\cite{Knapen:2016hky}.
High-multiplicity signatures therefore require dedicated modeling, and we defer the study of these signatures to Chapter~\ref{sec:showers}.


Finally, we conclude by noting that simplified models are intended to provide a general framework to cover a broad swath of models.
Any simplified model set-up, however, cannot cover every single UV model without becoming as complex as the UV model space itself. As with the case of promptly decaying new particles, care must also be taken in the interpretation of simplified models~\cite{Alves:2011wf,Abdallah:2015ter,Abercrombie:2015wmb,Boveia:2016mrp,Ambrogi:2017lov}:~for example, constraints on simplified models assuming 100\% branching fractions of LLPs to a particular final state may not accurately represent the constraint on a full model due to the large multiplicity of possible decay modes.
There will additionally always be very well-motivated models that predict specific signatures that are challenging to incorporate into the simplified model framework outlined here.
Experimental searches for these signatures should still be done where possible, but we encourage theorists and experimentalists alike to think carefully about how to design such searches so as to retain maximal sensitivity to simplified models that may give rise to similar signatures.
