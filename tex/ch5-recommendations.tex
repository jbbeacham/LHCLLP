\section{Our Proposals for the Presentation of Results}
\label{sec:ch5-rec_summary}

Here we summarize the recommendations for the presentation of searches involving long-lived
particles. These recommendations follow from the detailed
examples presented in Sections~\ref{sec:ch5-smsReinterpretations}
and~\ref{sec:ch5-recastExamples}.

Our primary recommendation is that the experiments provide as detailed
information as possible to make a general recasting feasible.
We therefore encourage the experiments to:
\begin{description*}
  \item[A.1.] Provide LLP reconstruction and selection efficiencies at the signature or object level. Although the parametrization of efficiencies is strongly analysis dependent, it is advantageous if they are given as a function of model-independent variables (such as functions of displaced vertex $d_0$, $\pt$, $\eta$, etc.), so
  they do not rely on a specific LLP decay or production mode;
  \item[A.2.] Present results for at least two distinct benchmark
  models, with different event topologies, since it greatly helps to validate the recasting. For clarity, the input cards for the benchmark points should also be provided;
  \item[A.3.] Present cut-flow tables, for both the signal
    benchmarks and the background, since these are  very useful for
    validating the recasting;
  \item[A.4.]  When an analysis is superseded, differences and commonalities with previous versions of the same analysis should be made clear, especially if the amount of information presented in both analyses differs. The understanding as the extent to which the information presented in an old version can be used directly in a later version greatly helps the recasting procedure, and also highlights ways in which the new search gains or loses sensitivity relative to the superseded analysis;
  \item[A.5.] Provide all this material in numerical form, preferably on HEPdata, or on the collaboration wiki page. A very useful resource we also highly encourage is a truth-code snippet illustrating the event and object selections, such as the one from the ATLAS disappearing-track search~\cite{Aaboud:2017mpt} provided in HEPdata under ``Common Resources".
\end{description*}

\noindent
%
We realize that implementing the above recommendations requires an enormous amount of time and effort by
the collaborations and may not always be feasible to the full extent.
However, good examples of presentations are already available, such as
the parametrized efficiencies provided by the ATLAS 13 TeV displaced
vertex~\cite{Aaboud:2017iio} (see the auxiliary material to Ref.~\cite{SUSY-2016-08}) and the CMS 8 TeV heavy stable
charged particle~\cite{Khachatryan:2015lla} analyses.

When the object- or signature-level efficiency maps are not feasible, providing efficiencies for an extensive, diverse array of simplified models can be useful for reinterpretation.
Concerning simplified-model results, we recommend that the experiments:
\begin{description*}
  \item[B.1.] Provide signal efficiencies (acceptance times efficiency) for
  simplified models and not only upper limits or exclusion curves.
 % Note that to be useful for reinterpretation, efficiencies
  %for all signal regions, not only the best one, are needed;
  Note that efficiencies for {\it{all signal regions}}, not only the best one, are necessary for reinterpretation;
  \item[B.2.] Present efficiency maps as a function of the relevant simplified-model parameters, such as the LLP mass and lifetime, with sufficient coverage of the simplified-model parameter space. While for direct production of LLPs the parameter space is 2-dimensional
  (LLP mass and lifetime), simplified models with cascade decays have a higher-dimensional parameter space. In these cases we strongly recommend
  efficiencies to be provided for a significant range of {\it all} the
  parameters;
  \item[B.3.] Release the efficiencies in digital format (on HEPdata or the collaboration wiki page), going beyond the 2-dimensional parameterization suitable for paper plots whenever necessary. In particular, for auxiliary material, we recommend multidimensional data tables instead of a proliferation of 2-dimensional projections
  of the parameter space;
  \item[B.4.] Consider in each analysis a range of simplified models which aim to encompass:
%  \begin{enumerate}
\begin{description*}
    \item[(a)] Different decay modes, including distinct final-state particles and multiplicities;
    \item[(b)] Different LLP boosts (for example, provide efficiencies and limits
        for distinct parent particle masses, which decay to the LLP).
% \end{enumerate}
\end{description*}
\end{description*}

\noindent
Although extensive, the above recommendations for the choice of simplified
models allow for a thorough comparison between the range of validity of
the LLP analyses and a detailed test of recasting methods.
Furthermore, when an MC-based recasting is not available, one
can use the ``nearest'' simplified model or a combination of them to
estimate the constraints on a theory of interest.
Finally, if a sufficiently broad spectrum of simplified models is covered, this can
be useful for quickly testing complex models which feature a large variety of signatures,
and rapidly finding the interesting region in a model scan before going to more precise but
computationally more expensive MC simulation.

We hope that our recommendations, in particular, points A.1--A.5, will serve as a guide for best practices and help establish a reliable and robust reinterpretation of LLP searches.
The added value for the experiments and the whole HEP community will be the immediate and more precise feedback on the implications of the LLP results for a broad range of theoretical scenarios, including gaps in coverage.


% \vspace{\baselineskip}

% \noindent {\bf Acknowledgements:}

% \vspace{\baselineskip}

% \noindent We thank Ben Allanach for comments. G.C. acknowledges support by the Ministry of Science and
%   Technology of Taiwan under grant No. MOST-106-2811-M-002-035. N.D. is supported by OCEVU Labex (ANR-11-LABX-0060) and the A*MIDEX project (ANR-11- IDEX-0001-02) funded by the ”Investissements d’Avenir” French government program managed by the ANR.
