\section{Our Proposals for the Presentation of Results}
\label{sec:ch5-rec_summary}

Here we summarize the recommendations for the presentation of long-lived
particles search results. These recommendations follow from the detailed
examples presented in Sections~\ref{sec:ch5-smsReinterpretations} 
and~\ref{sec:ch5-recastExamples}.

Our primary recommendation is that the experiments provide as detailed
information as possible to make a generic recasting possible. 
We therefore encourage the experiments to: 
\begin{description*}
  \item[A.1.] Provide LLP reconstruction and selection efficiencies
  at the signature or object level. Although the parametrization
  of efficiencies is strongly analysis dependent,
  it is of advantage if they are given as a function of model independent 
  variables (such as function of displaced vertex $p_T$, $\eta$, etc.), so 
  they do not rely on a specific LLP decay or production mode;
  \item[A.2.] Present results for at least two distinct benchmark
  models, with distinct event topologies, since it greatly helps 
  validating the recasting. For clarity, the input cards for the benchmark 
  points should also be provided;  
  \item[A.3.] Present cut-flow tables, for both the signal
    benchmarks and the background, since these are  very useful for
    validating the recasting;
  \item[A.4.]  When an analysis is superseded, differences and commonalities with previous
   versions of the same analysis should be made clear, especially if the amount of information 
   presented in both analysis differs. The understanding to which extent 
   the information presented in an old version can be used directly in a later version greatly 
   helps the recasting procedure;
  \item[A.5.] Provide all this material in numerical form, preferably on HEPdata, or on the 
  collaboration wiki page.
\end{description*}

\noindent 
We realize that the above requires an enormous amount of time and effort by 
the collaborations and may not be always feasible to full extent. 
However, good examples of presentations are already available, such as
the parametrized efficiencies provided by the ATLAS 13 TeV displaced
vertex~\cite{Aaboud:2017iio} (see auxiliary material on~\cite{SUSY-2016-08}) and the CMS 8 TeV heavy stable
charged particle~\cite{Khachatryan:2015lla} analyses.

When object- or signature-level efficiency maps are not feasible, providing
efficiencies for a large, diverse array of simplified models can be 
useful for re-interpretation. 
With respect to simplified-model results we recommend that the experiments: 
\begin{description*}
  \item[B.1.] Provide signal efficiencies  (acceptance times efficiency) for 
  simplified models and not only upper limits or exclusion curves;
  \item[B.2.] Present efficiency maps as a function of the relevant simplified model
  parameters, such as the LLP mass and lifetime, with  
  sufficient coverage of the simplified model parameter space. While for 
  direct production of LLPs the parameter space is 2-dimensional
  (LLP mass and lifetime), simplified models with cascade decays have
  a higher dimensional parameter space. In these cases we strongly recommend
  efficiencies to be provided for a significant range of {\it all} the
  parameters; 
  \item[B.3.] Release the efficiencies in digital format (on HEPdata or on the 
  collaboration wiki page), going 
  beyond the 2-dimensional parameterization suitable for paper plots whenever necessary;
  \item[B.4.] Consider a range of simplified models which aim to encompass:
%  \begin{enumerate}
\begin{description*}
    \item[(a)] Different multiplicities of final-state particles, to allow ready
        re-interpretation of the limits for both 2-body and 3-body decays, which
        are qualitatively different;
    \item[(b)] Different LLP boosts (for example, provide efficiencies and limits 
        for distinct parent particle masses, which decay to the LLP);
% \end{enumerate}  
\end{description*}
\end{description*}

\noindent 
Although extensive, the above recommendations for the choice of simplified
models allow for a thorough comparison between the range of validity of
the LLP analyses and a detailed test of recasting methods.
Furthermore, when a Monte Carlo based recasting is not available, one
can use the ``nearest'' simplified model or a combination of them to 
estimate the constraints on the theory of interest.
Finally, if a large enough spectrum of simplified models is covered, this can 
be useful for fast testing of complex models which feature a large variety of signatures, 
and quickly finding the interesting region in a model scan before going to more precise but 
computationally more expensive Monte Carlo simulation. 

We hope that our recommendations, in particular points A.1--A.5, will serve as a
guide for best practice and help establish a reliable and robust re-interpretation of LLP searches.   
The added value for the experiments, and the whole HEP community, will be a fast 
and more precise feedback on the implications of the LLP results for a broad range of
theoretical scenarios.  


%This can help simplify the comparison between the reported efficiencies.
% 
% Finally, we also comment on how the presentation on the information publicly
% available by the experimental collaborations can be improved for a more
% straightforward reinterpretation. Specifically;
% 
% \begin{itemize}
% \item When an analysis is superseded by a conference note, often the older
% analysis has more information for recasting.  It would be useful to know if this
% information can be used (i.e. differences and commonalities with previous
% versions of same analysis should be made clear).
% \item Efficiency tables should be clearly accessible (ideally from the
% conference note or paper itself). In general, if recasting a search warrants
% supplemental information beyond what is provided in the main document, links to
% this additional material should still be provided within the text.
% \end{itemize}
% 
% It is our hope that, with these recommendations, experimental analysis at the
% LHC can present their long-lived particles search results in an optimal way, so
% they can be re-used. This will ensure a long-term impact of the LHC legacy.
% 
