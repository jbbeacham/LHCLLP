LLP searches present an interesting experimental challenge, because detectors were designed to look for SM particles (that typically decay promptly or travel most of the way through the detector). However, it is also an exciting possibility for looking for long-lived BSM physics, because a spectacular new-physics signature may have significantly less background than the corresponding prompt search.

\section{Using the LHC Detectors to Look for LLPs}

In this section, we review the different ways in which the LHC detectors can be used to reconstruct long-lived objects. In some sense, this is a list of signatures that are currently possible to reconstruct that look different from prompt SM physics. This list will necessarily change as the detectors are upgraded or new clever uses are found. In each section, we include a short summary and references to documents that provide more details on reconstruction (where available).


\subsection{Inner Detector Tracking}

\begin{itemize}
\item Non-pointing inner detector tracks (large impact parameters)
\item Appearing inner detector tracks (missing hits in innermost layers)
\item Disappearing inner detector tracks (missing hits in outermost layers)
\item Highly $dE/dx$ tracks (good for highly charged or slow particles)
\item Low $dE/dx$ tracks (good for fractionally charged particles)
\item Kinked tracks (disappearing track linked to appearing track, not pointing in same direction)
\item Displaced multitrack vertices (vertex location far from beamspot, and/or direction not pointing to beamspot)
\end{itemize}

\subsection{Calorimetry}

\begin{itemize}
\item Anomalous HCAL vs. ECAL depositions
\item Timing information (consistent with slow, non-pointing, or out-of-time particles)
\item Pointing information from depth segmentation and shower shape variables
\item Anomalous shower shapes (good for monopoles)
\end{itemize}

\subsection{Muon Spectrometer}
\begin{itemize}
\item Short, not necessarily pointing tracks
\item Anomalous $dE/dx$
\item Timing information (to identify slow particles, distinguish outward vs.~inward tracks)
\item Dimuon vertices from MS-only muon tracks
\item Multitrack vertices from short tracks
\end{itemize}

\subsection{RICH Detectors (LHCb)}

Can be used to reconstructed charged LLPs with mass well-separated from pions, kaons and protons. Can cover momentum range $\sim1-150$ GeV and $\beta>0.6$.

[More details needed here]

\subsection{Combinations of Main Detector Systems}
\subsubsection{Lepton ID Algorithms}

\begin{itemize}
\item Electron = inner track + EM cluster
\item Muon = inner track + track in muon system
\item Hadronic tau = inner track(s) + calo clusters
\end{itemize}

\subsubsection{Jets}

Typically a combination of tracks + calo clusters

\begin{itemize}
\item Multiple displaced vertices within a jet (\emph{e.g.,} emerging jets)
\item Displaced vertex followed by a jet
\end{itemize}

\subsubsection{Displaced lepton jets}

A lepton jet is a collimated collection of muons, electrons, and pions. Perhaps also include other collimated, displaced objects here?

\begin{itemize}
\item Combination of low-EMF jet + muon-system only muon tracks ($ee+\mu\mu$)
\item low-EMF jet ($ee$)
\item Collimated, muon-system only muon tracks
\end{itemize}

\subsubsection{Combination of LHCb subsystems}

Combination of vertex locator + tracking stations (LHCb) + PID detectors can be sensitive to charged LLP decaying to charged particles (used to reconstruction $K+\rightarrow 3\pi$)

\subsubsection{Other Combinations}

\begin{itemize}
\item Calorimeter clusters or tracks in muon system that lack inner-detector tracks
\item High $dE/dx$ track + time-of-flight measurement in calorimeter or muon system
\item Multitrack vertex in muon system with no associated inner-detector tracks 
\end{itemize}

\subsection{Non-Presence of $pp$ Collisions}

Use forward detectors or knowledge of LHC bunch-filling scheme

\section{Summary of Existing LLP Searches}

Here, we include a table of existing LLP searches, with references to the search and hyperlinked to more detailed discussion in Chapter \ref{sec:signatures} [maybe also include links to theory and experiment motivation chapters?]


